\documentclass[11pt]{amsart}
\usepackage[english]{babel}
\usepackage{amsmath}
\usepackage{amsfonts}
\usepackage{amssymb}
%\usepackage{showlabels}
\usepackage{amsthm}
\usepackage{marginnote}

\usepackage[english]{babel}
\usepackage{yfonts}
\usepackage[T1]{fontenc}
\usepackage[utf8x]{inputenc}
\usepackage{enumerate}
\usepackage{verbatim}
\usepackage{graphicx}
\usepackage{verbatim}
\usepackage{faktor}
\usepackage{xcolor}
\usepackage{xfrac}
\usepackage{tikz,tikz-cd}
\usepackage[all]{xy}
\usepackage{bbm}

\newcommand{\M}[4]{\overline{\mathcal M}_{#1,#2}(#3,#4)}
\newcommand{\Q}[4]{\overline{\mathcal Q}_{#1,#2}(#3,#4)}
\newcommand{\Qt}[4]{\widetilde{\mathcal Q}_{#1,#2}(#3,#4)}
\newcommand{\PP}{\mathbb P}
\newcommand{\Z}{\mathbb{Z}}
\newcommand{\N}{\mathbb{N}}
\newcommand{\OO}{\mathcal{O}}
\renewcommand{\to}{\rightarrow}
\newcommand{\A}{\mathcal A}
\newcommand{\B}{\mathcal B}
\newcommand{\C}{\mathfrak C}
\renewcommand{\L}{\mathcal L}
\newcommand{\MM}{\mathfrak M}
\newcommand{\Aaff}{\mathbb{A}}
\newcommand{\kfield}{\mathbb{K}}
\newcommand{\comp}{\chi}
\newcommand{\sst}{\sigma^{ss}}
\newcommand{\Pic}{\operatorname{Pic}}
\newcommand{\Hom}{\operatorname{Hom}}
\newcommand{\Gm}{\mathbb{G}_m}
\newcommand{\virt}[1]{[#1]^{\operatorname{virt}}}
\newcommand{\Id}{\operatorname{Id}}

\newcommand{\bq}{\begin{equation}}
\newcommand{\eq}{\end{equation}}
\newcommand{\ba}{\begin{aligned}}
\newcommand{\ea}{\end{aligned}}
\newcommand{\be}{\begin{enumerate}}
\newcommand{\ee}{\end{enumerate}}
\newcommand{\bsm}{\left(\begin{smallmatrix}}
\newcommand{\esm}{\end{smallmatrix}\right)}                   
\newcommand{\bpm}{\begin{pmatrix}}
\newcommand{\epm}{\end{pmatrix}}
\newcommand{\barr}{\begin{displaymath}\begin{array}{cccc}}
\newcommand{\earr}{\end{array}\end{displaymath}}
\newcommand{\barrl}{\begin{displaymath}\begin{array}{lcl}}
\newcommand{\earrl}{\end{array}\end{displaymath}}
\newcommand{\barl}{\begin{displaymath}\begin{array}{l}}
\newcommand{\earl}{\end{array}\end{displaymath}}
\newcommand{\bxym}{ \begin{displaymath}\xymatrix }
\newcommand{\exym}{\end{displaymath}}
\newcommand{\bcd}{\begin{center}\begin{tikzcd}}
\newcommand{\ecd}{\end{tikzcd}\end{center}}
\newcommand{\R}{\operatorname{R}^{\bullet}}

\newcommand{\tr}{{\rm tr}}
\newcommand{\Isom}{\text{Isom}}
\newcommand{\Spec}{\operatorname{Spec}}
\newcommand{\pr}{\operatorname{pr}}
\newcommand{\ev}{\operatorname{ev}}
\newcommand{\codim}{\operatorname{codim}}
\newcommand{\vdim}{\operatorname{vdim}}
\newcommand{\ildef}[1]{\textbf{\textsc{#1}}}
\newcommand{\om}[1]{\overline{\mathcal{#1}}}
\newcommand{\h}{\operatorname{h}}

\theoremstyle{plain}
\newtheorem{thm}{Theorem}[section]
\newtheorem{lem}[thm]{Lemma}
\newtheorem{lemma}[thm]{Lemma}
\newtheorem{prop}[thm]{Proposition}
\newtheorem{cor}[thm]{Corollary}
\newtheorem*{teo*}{Theorem}
\newtheorem{ipotesi}{ipotesi}
\newtheorem*{nota}{Nota}
\newtheorem{claim}{Claim}
\newtheorem{question}[thm]{Question}
\newtheorem{conj}[thm]{Conjecture}

\theoremstyle{definition}
\newtheorem{example}[thm]{Example}
\newtheorem{ex}[thm]{Example}
\newtheorem{dfn}[thm]{Definition}
\newtheorem{aside}[thm]{Aside}
\newtheorem{remark}[thm]{Remark}
\newtheorem{com}[thm]{Comment}
\newtheorem{num}{Number}
\newtheorem*{sketch}{Sketch}
\newtheorem*{rem}{Remark}


\newcommand{\todo}[1]{\vspace{5mm}\par \noindent
\framebox{\begin{minipage}[c]{0.95 \textwidth} \tt #1\end{minipage}} \vspace{5mm} \par}

\def\ti{-\allowhyphens}
\newcommand{\thismonth}{\ifcase\month % case 0 --- impossible!
  \or January\or February\or March\or April\or May\or June%
  \or July\or August\or September\or October\or November%
  \or December\fi}
\newcommand{\thismonthyear}{{\thismonth} {\number\year}}
\newcommand{\thisdaymonthyear}{{\number\day} {\thismonth} {\number\year}}

\usepackage[T1]{fontenc}
\usepackage{newpxtext,newpxmath}

\title{Relative quasimaps \`a la Gathmann}
\author{Luca Battistella and Navid Nabijou}
\begin{document}
\maketitle
\begin{abstract}
Relative quasimaps are good.
\end{abstract}

\tableofcontents

\section{Introduction}

\section{Stable Quasimaps and their Invariants}

\section{Quasimaps to $\PP^r$ Relative to a Hyperplane}

We first deal with genus 0 quasimaps to projective space, relative to a hyperplane. We give a Gathmann-like description of the space of relative quasimaps as a closed substack of the moduli space of (absolute) quasimaps to $\PP^r$; it turns out to be irreducible of the expected dimension. Finally, we retrieve a Gathmann-type formula by pushforward along the comparison morphism $\comp\colon \M{0}{n}{\PP^r}{d}\to\Q{0}{n}{\PP^r}{d}$.

Fix coordinates on $\PP^r$ such that the hyperplane $H$ is $\{x_0=0\}$. Let $\alpha=(\alpha_1,\ldots,\alpha_n)$ be an $n$-tuple of nonnegative integers. Consider the following locus $\Qt{0}{\alpha}{\PP^r|H}{d}$ inside $\Q{0}{n}{\PP^r}{d}$: the quasimaps $(C,x_1,\ldots,x_n,L,u_0,\ldots,u_r)$ such that, if $Z$ is a connected component of the vanishing locus of $u_0$ in $C$, then one of the following holds:

\begin{enumerate}
\item $Z$ is a point, either unmarked, or one of the $x_i$'s, and in this case $u_0$ vanishes at $Z$ with multiplicity at least $\alpha_i$.
\item $Z$ is a curve (\emph{internal}); letting $C^{(1)},\ldots,C^{(k)}$ be the (\emph{external}) irreducible components adjacent to $Z$, with nodes $q_i=Z\cap C^{(i)}$, and $m^{(i)}$ the order of vanishing of $u_{0|C^{(i)}}$ at $q_i$, we must have
\[
\deg(L_{|Z})+\sum_{i=1}^k m^{(i)}\geq\sum_{x_j\in Z} \alpha_j
\]
\end{enumerate}

On the other hand, denote by $\mathcal Q_{0,\alpha}(\PP^r|H,d)$ the \emph{nice locus}, consisting of actual maps from an irreducible curve (i.e. $\PP^1$) and with specified tangency condition $\alpha$ at the markings $\mathbf x$. Notice that this is an irreducible, locally closed substack of $\Q{0}{n}{\PP^r}{d}$, by pretty much the same argument as in \cite[Lemma 1.8]{Ga}; it has codimension $\sum\alpha$. In fact it is isomorphic to the nice locus inside stable maps, that Gathmann denotes by $\mathcal M_{0,\alpha}(\PP^r|H,d)$ \cite[Def. 1.6]{Ga} (the stricter stability condition has no effect when the source curve is irreducible, of course provided $n\geq2$); hence:

\begin{lem}\label{lem:comparison}
The comparison morphism restricts to a birational morphism $\M{0}{\alpha}{\PP^r|H}{d}\to \Qt{0}{\alpha}{\PP^r|H}{d}$.
\end{lem}
\begin{proof}
The contraction of a rational tail $R$ always happens far away from the markings, hence the only care we need to take is when the one component touching $R$ is internal (call it $Z$); in this case, observe that $m^{(R)}\leq\deg(f_{|R})$ and the quasimap resulting from the contraction of $R$ has $\deg(L_{|Z})=\deg(f_{|Z})+\deg(f_{|R})$, so the corresponding term only moves around the LHS of the $\alpha$-tangency condition nr. 2.

Birationality follows from the fact that the comparison morphism restricts to give an isomorphism between the nice loci.
\end{proof}

\begin{lem}
With notations as above (with $\sum\alpha\leq d$), $\Qt{0}{\alpha}{\PP^r|H}{d}$ is the closure of the nice locus $\mathcal Q_{0,\alpha}(\PP^r|H,d)$ inside $\Q{0}{n}{\PP^r}{d}$. 
\end{lem}
\begin{proof}
$\Qt{0}{\alpha}{\PP^r|H}{d}\subseteq\overline{\mathcal Q_{0,\alpha}(\PP^r|H,d)}$: we show that, given any quasimap satisfying the $\alpha$-tangency conditions spelled above, it can be (infinitesimally) deformed to a stable \emph{map} satisfying Gathmann's conditions \cite[Def. 1.1 and Rmk. 1.4]{Ga}, and then appeal to \cite[Prop. 1.14]{Ga}.

We induct on the number of components containing at least one base-point. If this number is zero, we're done (because quasimap stability is stronger than map stability); otherwise, pick such a component $C_0$, with base-points $p_1,\ldots,p_h$ and adjacent rational trees $R_1,\ldots,R_k$, joined to $C_0$ at the nodes $q_1,\ldots,q_k$. Since there are base-points but the quasimap respects the nondegeneracy condition, $\deg(L_{|C_0})>0$, and since $C_0\simeq\PP^1$ we can find a section $w$ of $L_{|C_0}\simeq\mathcal O_{\PP^1}(d_0)$ not vanishing at any of the base-points $p_i$'s; then it is enough to deform the section $u_{r|C_0}$ to $u_{r|C_0}+\epsilon w$ (and keep the other sections the same) in order to delete the base-points belonging to $C_0$. Notice that $u_{0|C_0}$ is unchanged, so the deformation still respects $\alpha$-tangency at the markings lying on $C_0$ (whether the latter is an internal or an external component). We need to check that such a deformation can be extended to the whole curve $C$ without changing the vanishing conditions on $u_0$. Notice that the action of $PGL_{r+1}$ on $\PP^r$ extends to an action of the group on the space of quasimaps; we can apply the matrix
\[
\begin{bmatrix}
1 & & \\
 & \ddots & \\
 & \epsilon \frac{w(q_i)}{u_j(q_i)} & 1
\end{bmatrix}
\]
to the restriction of the original quasimap to $R_i$, where $j$ is any index s.t. $u_j(q_i)\neq 0$ (one such must exist because the node is not allowed to be a base-point), and by doing this separately to every rational tree springing from $C_0$ we get a deformation of the original quasimap that still has $\alpha$-tangency with the hyperplane $H$ ($u_0$ hasn't been touched at all), but the base-points on $C_0$ have been eliminated.

$\overline{\mathcal Q_{0,\alpha}(\PP^r|H,d)}\subseteq\Qt{0}{\alpha}{\PP^r|H}{d}$: consider a family of relative quasimaps over a smooth curve $S$, such that the generic fiber lies in the nice locus. Then we may blow-up the source curve (which is a fibered surface) in the base-points of the quasimap (that are finitely many smooth points of the central fiber) in order to get an actual morphism to $\mathbb P^r$; we may as well suppose that the central fiber of the new family is stable. Notice that the central fiber actually belongs to Gathmann's space $\M{0}{\alpha}{\PP^r|H}{d}$: we have just introduced some rational tails away from the markings, hence the only thing we have to check is, when we blow-up a base-point on an internal component, the rational tail will again be internal ($u_0\equiv 0$ in a neighborhood of the base-point), so it will contribute to the LHS of the $\alpha$-tangency condition nr. 2 in the very same way. We may now invoke \cite[Lemma 1.9]{Ga} and the quasimap case follows from Lemma \ref{lem:comparison}.
\end{proof}
From now on we shall denote this closed substack by $\Q{0}{\alpha}{\PP^r|H}{d}$.

\medskip

Increasing the multiplicity can be naively performed in the very same way as Gathmann did:
\[
\sigma^m_k:=x_k^*d^m_{\mathcal C/\overline{\mathcal Q}}(u_0)\in H^0(\overline{\mathcal Q},x_k^*\mathcal P^m_{\mathcal C/\overline{\mathcal Q}}(\mathcal L))
\]
with $m=\alpha_k+1$ cuts $\Q{0}{\alpha+e_k}{\PP^r|H}{d}$ inside $\Q{0}{\alpha}{\PP^r|H}{d}$, together with a bunch of degenerate contributions from quasimaps where the component on which $x_k$ lies is internal (call it $Z$) and (notice the equality sign!)
\[
\deg(L_{|Z})+\sum m^{(i)}=\sum_{x_j\in Z}\alpha_j.
\]
Of course, quasimap stability forces these degenerate contributions not to have any rational tail; this is really the only difference with the case of stable maps, and indeed we can pushforward Gathmann's formula along the comparison morphism $\comp\colon \M{0}{n}{\PP^r}{d}\to\Q{0}{n}{\PP^r}{d}$ and the only terms that are going to change are the degenerate ones with rational tails (in fact they disappear, since the restriction of the comparison map has positive dimensional fibers there). With an eye to the future, we remark that these contributions do matter when computing GW invariants of a CY hypersurface in projective space, and may well account for the divergence between GW and quasimap invariants in the CY case \cite[Rmk. 1.6]{Ga-MF}.

\begin{lem}\label{lem:compare_psi}
$\comp^*(\psi_k)=\psi_k$ and $\comp^*(x_k^*\mathcal L)=\ev_k^*(\mathcal O_{\mathbb P^r}(H))$.
\end{lem}
\begin{proof}
Recall that $\psi_k=c_1(x_k^*\omega_{\mathcal C/\mathcal M})$ and contemplate the following diagram
\bcd
& & & \PP^r & \\
\mathcal C_{\overline {\mathcal M}}\ar[rr,"\sst"]\ar[rd] \ar[urrr,"f"] & & \comp^*\mathcal C_{\overline {\mathcal Q}} \ar[ld]\ar[rr]\ar[ur,dashed] & & \mathcal C_{\overline {\mathcal Q}} \ar[d]\ar[ul,dashed] \\
& \M{0}{n}{\PP^r}{d} \ar[rrr,"\comp"] \ar[ul,bend left,"x_k"]\ar[ur,bend right,"x_k"right=.2cm]& & & \Q{0}{n}{\PP^r}{d}\ar[u,bend right, "x_k"right]
\ecd
where $\sst$ is the strong stabilisation map, i.e. contracting the rational tails, which is an isomorphism near the markings.
\end{proof}

\begin{lem}\label{lem:posdimfiber}
$\dim(\M{0}{(m^{(i)})}{\PP^r|H}{d}\cap \ev_1^*(p))>0$ everytime $rd>1$, where $p$ is a point of $H$, so the pushforward along $\comp$ of a degenerate locus with rational tails is 0.
\end{lem}
\begin{proof}
$\dim(\M{0}{(m^{(i)})}{\PP^r|H}{d}\cap \ev_1^*(p))=(r-3)+(1-m^{(i)})+d(r+1)-(r-1)=(rd-1)+(d-m^{(i)})$.
\end{proof}

\begin{prop}
Denote by $[D^\mathcal{Q}_{\alpha,k}(\PP^r|H,d)]$ the sum of the (product) fundamental classes of
\[
\Q{0}{\alpha^{(0)}\cup {(0,\ldots,0)}}{H}{d_0}\times_{(\PP^r)^k}\prod_{i=1}^k \Q{0}{(m^{(i)})\cup\alpha^{(i)}}{\PP^r|H}{d_i}
\]
with coefficient $\frac{m^{(1)}\ldots m^{(k)}}{k!}$, where the sum runs over all splittings $d=\sum d_i$ and $\alpha=\bigcup \alpha^{(i)}$ such that the above spaces are well-defined, in particular $|\alpha^{(0)}|+k$ and $|\alpha^{(i)}|+1$ are all $\geq 2$, and such that
\[
d_0+\sum_{i=1}^k m^{(i)}=\sum \alpha^{(0)}
\]

The following formula holds
\[
(\alpha_k\psi_k+x_k^*\mathcal L)\cdot[\Q{0}{\alpha}{\PP^r|H}{d}]=[\Q{0}{\alpha+e_k}{\PP^r|H}{d}]+[D^\mathcal{Q}_{\alpha,k}(\PP^r|H,d)].
\]
\end{prop}
\begin{proof}
Follows from \cite[Thm. 2.6]{Ga} by pushforward along $\comp\colon \M{0}{n}{\PP^r}{d}\to\Q{0}{n}{\PP^r}{d}$, using the projection fomula and Lemmas \ref{lem:comparison}, \ref{lem:compare_psi} and \ref{lem:posdimfiber}.
\end{proof}

\section{Comparison with the GIT Construction}
Let $X$ be a hypersurface of degree $a$ in $\PP^r$. In the preceding sections, we have put a virtual class on $\Q{g}{n}{X}{d}$ by way of the following Cartesian diagram:

\bcd
\Q{g}{n}{X}{d}\ar[d]\ar[r] & \Q{g}{n}{\PP^r}{d}\ar[d,"\nu_a"] \\
\Q{g}{n}{H}{ad}\ar[r] & \Q{g}{n}{\PP^N}{ad}
\ecd

where $N={{r+a}\choose{a}}-1$ and $\nu_a$ is the Veronese embedding. In fact, $\Q{g}{n}{X}{d}$ is thought of as representing stable quasimaps to $\PP^r$ such that the corresponding sections satisfy the equation for $X$ inside $\PP^r$, that is a homogeneous polynomial $Q$ of degree $a$, i.e. gives a section of $L^{\otimes a}$ on the source curve $C$.

We wish to compare this with the GIT approach of \cite{CFKM}. Here $X$ is seen as the GIT quotient of the affine cone $C_X\subseteq \mathbb A^{r+1}$ with respect to the diagonal $\mathbb G_m$-action. Objects of $\Q{g}{n}{X}{d}^\text{GIT}$ are diagrams of the form

\bcd
P\ar[d,"\mathbb G_m"]\ar[r] & C_X & \text{or, equivalently,} & P\times_{\mathbb G_m} C_X \ar[d,"\rho" left] \\
C & & & C \ar[u,bend right,"u"right]
\ecd
and the dual perfect obstruction theory with respect to $\mathfrak{Bun}_{\mathbb G_m}$ is given by $R^\bullet\pi_*(u^*\mathbb T^\bullet_\rho$), where $\pi\colon \mathcal C_{\mathfrak{Bun}}\to\mathfrak{Bun}_{\mathbb G_m}$ is the universal curve.

Notice that $\mathfrak{Bun}_{\mathbb G_m}\simeq \mathfrak{Pic}$ by taking the line bundle $L=P\times_{\mathbb G_m}\mathbb A^1\to C$ associated to the $\mathbb G_m$-torsor $P\to C$. Furthermore, the $\mathbb G_m$-equivariant embedding in a smooth stack
\bcd
P\times_{\mathbb G_m} C_X \ar[r,hook]\ar[d,"\rho" left] & P\times_{\mathbb G_m}\mathbb A^{r+1}\simeq L^{\oplus r+1}\ar[dl]\\
C \ar[u,bend right,"u"right] &
\ecd
gives us $u^*T^\bullet_\rho\simeq[L^{\oplus r+1}\to L^{\otimes a}]$, where the arrow is induced by $Q$, and shows that both the modular interpretation and the obstruction theory coincide.











\newpage

\section{Recursion Formula in the General Case}
We now move on to the general case. Let $X$ be an arbitrary toric variety (smooth and proper) and $Y \subseteq X$ a very ample hypersurface (not necessarily toric). The complete linear system associated to $\OO(Y)$ defines an embedding $i : X \hookrightarrow \PP^N$ such that $i^{-1}(H) = Y$. By the functoriality property of quasimap spaces (see Appendix \ref{Functoriality of Quasimap Spaces Section}) we have a map:
\begin{equation*} k := i_* : \Q{0}{n}{X}{\beta} \to \Q{0}{n}{\PP^N}{d} \end{equation*}
where $d=i_*\beta$. Furthermore since $i$ is a closed embedding it follows that $i_*$ is as well. It is easy to show that $i_*$ restricts to a morphism between the relative spaces, and thus we have a commuting diagram of embeddings:
\bcd
\Q{0}{\alpha}{X|Y}{\beta} \ar[d, "f", hook] \ar[r, "g", hook] \ar[dr, phantom, "\square"] & \Q{0}{\alpha}{\PP^N|H}{d} \ar[d, "j", hook] \\
\Q{0}{n}{X}{\beta}  \ar[r, "k", hook] & \Q{0}{n}{\PP^N}{d}
\ecd
Furthermore one can show that this is cartesian. As such we can define a virtual class on $\Q{0}{\alpha}{X|Y}{\beta}$ by pullback; that is, we consider the cartesian diagram
\bcd
\Q{0}{\alpha}{X|Y}{\beta} \ar[r, "f \times g", hook] \ar[d, hook]  \ar[rd, phantom, "\square"] & \Q{0}{n}{X}{\beta} \times \Q{0}{\alpha}{\PP^N|H}{d} \ar[d,"k \times j"] \\
\Q{0}{n}{\PP^N}{d} \ar[r, "\Delta"] & \Q{0}{n}{\PP^N}{d} \times \Q{0}{n}{\PP^N}{d}
\ecd
which is equivalent to the previous one. The morphism $\Delta$ is a regular embedding because $\Q{0}{n}{\PP^N}{d}$ is smooth, and thus we can define:
\begin{equation*} \virt{\Q{0}{\alpha}{X|Y}{\beta}} := \Delta^! \left( \virt{\Q{0}{n}{X}{\beta}} \times [\Q{0}{\alpha}{\PP^N|H}{d}] \right) \end{equation*}
The idea is to prove the recursion formula for $(X,Y)$ by pulling back the formula for $(\PP^N,H)$ along $k = i_*$. In order to do this, we need to understand how the various virtual classes involved in the formula pull back along this map. This is the technical heart of the proof.

But before getting into any of this, we need to explain what we mean by ``pulling back along $k$.'' Indeed, $k$ is not necessarily a regular embedding, so the Gysin map in the sense of \cite{FUL} does not necessarily exist.

In \cite{Manolache-Pull} a generalisation of the Gysin map (called the \ildef{virtual pull-back}) is defined for morphisms endowed with a relative perfect obstruction theory. Moreover, a sufficient condition is given (Corollary 4.9) for this map to respect the virtual classes.

\begin{lem} \label{Exists relative POT} There exists a relative perfect obstruction theory $E_k$ for the morphism
\begin{equation*} k : \Q{0}{n}{X}{\beta} \to \Q{0}{n}{\PP^N}{d} \end{equation*}
and hence there exists a virtual pull-back morphism $k^!_{\text{v}}$. Moreover, $E_k$ fits into a compatible triple with the standard obstruction theories for the quasimap spaces over $\MM_{0,n}$, so that:
\begin{equation*} k^!_{\text{v}} [ \Q{0}{n}{\PP^N}{d} ] = \virt{\Q{0}{n}{X}{\beta}} \end{equation*} \end{lem}

\begin{proof} Note first that since $k$ does not change the source curve of a quasimap we indeed have a commuting triangle:
\bcd
\Q{0}{n}{X}{\beta} \ar[rr,"k"] \ar[rd] & & \Q{0}{n}{\PP^N}{d} \ar[ld] \\
& \MM_{0,n} & 
\ecd
We have perfect obstruction theories $E_{\overline{\mathcal{Q}}(X)/\MM}$ and $E_{\overline{\mathcal{Q}}(\PP^N)/\MM}$ and we want to find a perfect obstruction theory $E_k$. Consider the diagram of universal curves
\bcd
\mathcal{C}_X \ar[r,"\alpha"] \ar[d,"\pi"] \ar[rd,phantom,"\square" right] & \mathcal{C}_{\PP^N} \ar[d,"\rho"] \\
\Q{0}{n}{X}{\beta} \ar[r,"k"] & \Q{0}{n}{\PP^N}{d}
\ecd
which is cartesian because $k$ does not alter the source curve of any quasimap. We have sheaves $\mathcal{F}_X$ and $\mathcal{F}_{\PP^N}$ on $\mathcal{C}_X$ and $\mathcal{C}_{\PP^N}$ respectively such that:
\begin{align*} E_{\overline{\mathcal{Q}}(X)/\MM}^\vee & = \R \pi_* \mathcal{F}_X \\
E_{\overline{\mathcal{Q}}(\PP^N)/\MM}^\vee & = \R \rho_* \mathcal{F}_{\PP^N} \end{align*}
It follows that when we pull back the latter obstruction theory to $\om{Q}(X)$ we obtain:
\begin{equation*} k^* E_{\overline{\mathcal{Q}}(\PP^N)/\MM}^\vee = \R \pi_* \alpha^* \mathcal{F}_{\PP^N} \end{equation*}
To construct a compatible triple, we require a morphism $k^* E_{\om{Q}(\PP^N)/\MM} \to E_{\om{Q}(X)/\MM}$. It is therefore enough to construct a morphism of sheaves on $\mathcal{C}_X$
\begin{equation*} \mathcal{F}_X \to \alpha^* \mathcal{F}_{\PP^N} \end{equation*}
and then apply $\R \pi_*$. This is analogous to the morphism $f^* T_X \to f^* T_{\PP^N}|_X$ which is used in the stable maps setting. Here, however, the construction requires a little more ingenuity, because we do not have acess to a universal map $f$.

The sheaf $\mathcal{F}_X$ is defined on $\mathcal{C}_X$ by the short exact sequence
\begin{equation*} 0 \to \OO_{\mathcal{C}_X}^{\oplus r} \to \oplus_{\rho} \mathcal{L}_\rho \to \mathcal{F}_X \to 0 \end{equation*}
where $r = \operatorname{rk} \Pic X$ (implicitly we have chosen a basis for this $\Z$-module) and similarly $\mathcal{F}_{\PP^N}$ is defined on $\mathcal{C}_{\PP^N}$ by:
\begin{equation*} 0 \to \OO_{\mathcal{C}_{\PP^N}} \to \mathcal{L}^{\oplus (N+1)} \to \mathcal{F}_{\PP^N} \to 0 \end{equation*}
We will construct our morphism by first constructing a morhism:
\begin{equation*} \oplus_{\rho} \mathcal{L}_\rho \to \alpha^* \mathcal{L}^{\oplus(N+1)} \end{equation*}
Recall that $i : X \hookrightarrow \PP^N$ is given by homogeneous polynomials
\begin{equation*} P_0 , \ldots, P_N \in S^X_\gamma \subset S^X = k[z_\rho : \rho \in \Sigma_X(1)] \end{equation*}
in the Cox ring of $X$, where $\gamma \in \Pic X$ is some divisor class. We can write this (non-uniquely) as:
\begin{equation*} \gamma = \sum_\rho c_\rho D_\rho \end{equation*}
Then given $i \in \{0,\ldots,N\}$ and sections $v_\rho$ of $\mathcal{L}_\rho$ for all $\rho \in \Sigma_X(1)$ we can use the same trick as in the proof of functoriality (see Appendix \ref{Functoriality of Quasimap Spaces Section}) to view $P_i(v_\rho)$ as a section of:
\begin{equation*} \otimes_\rho \mathcal{L}_\rho^{\otimes c_\rho} = \alpha^* \mathcal{L} \end{equation*}
Thus we obtain a morphism
\begin{equation*} (P_0, \ldots, P_N) : \oplus_\rho \mathcal{L}_\rho \to \alpha^* \mathcal{L}^{\oplus (N+1)} \end{equation*}
as required. On the other hand if we let $M_1, \ldots, M_r$ denote the chosen generators for $\Pic X$ then there exist unique numbers $b_1, \ldots, b_r$ such that:
\begin{equation*} i^* \OO_{\PP^N}(1) = \otimes_{i=1}^r M_i^{\otimes b_i} \end{equation*}
These define a map $\OO_{C_X}^{\oplus r} \to \OO_{C_X}$ and so we obtain a digram
\bcd
0 \ar[r] & \OO_{C_X}^{\oplus r} \ar[r] \ar[d] & \oplus_\rho \mathcal{L}_\rho \ar[r] \ar[d] & \mathcal{F}_X \ar[r] & 0 \\
& \OO_{C_X} \ar[r] & \alpha^* \mathcal{L}^{\oplus (N+1)} \ar[r] & \alpha^* \mathcal{F}_{\PP^N} \ar[r] & 0
\ecd
which one can check is commutative. By exactness of the first row, our morphism descends to a morphism $\mathcal{F}_X \to \alpha^* \mathcal{F}_{\PP^N}$ if the composition
\begin{equation*} \OO_{C_X}^{\oplus r} \to \oplus_\rho \mathcal{L}_\rho \to \alpha^* \mathcal{L}^{\oplus (N+1)} \to \alpha^* \mathcal{F}_{\PP^N} \end{equation*}
is zero. But this follows from the commutativity of the left square and the fact that the bottom row is a complex. Hence we obtain a morphism:
\begin{equation*} \mathcal{F}_X \to \alpha^* \mathcal{F}_{\PP^N} \end{equation*}
Applying $\R \pi_*$ and dualising we obtain a morphism between the obstruction theories for the quasimap spaces, and we can complete this to obtain an exact triangle
\begin{equation*} k^* E_{\om{Q}{\PP^N}/\MM} \to E_{\om{Q}(X)/\MM} \to E_k  \xrightarrow{[1]}\end{equation*}
on $\om{Q}(X)$. The complex $E_k$ is perfect (locally isomorphic to a bounded complex of vector bundles) because the other two are. On the other hand if we look at the long exact sequence in cohomology we find
\begin{equation*} 0 \to \h^{-2}(E_k) \to \h^{-1}(k^* E_{\om{Q}(\PP^N)/\MM}) = 0\end{equation*}
where the last term is zero because $\om{Q}(\PP^N) = \Q{0}{n}{\PP^N}{d}$ is unobstructed. Hence $\h^{-2}(E_k) = 0$ and it is easy to show using similar arguments that $E_k$ is of perfect amplitude contained in $[-1,0]$.

Finally the axioms of a triangulated category give a morphism of exact triangles
\bcd
k^* E_{\om{Q}{\PP^N}/\MM} \ar[r] \ar[d] & E_{\om{Q}(X)/\MM} \ar[r] \ar[d] & E_k  \ar[r,"{[1]}"] \ar[d] & \, \\
k^*L_{\om{Q}(\PP^N)/\MM} \ar[r] & L_{\om{Q}(X)/\MM} \ar[r] & L_k \ar[r,"{[1]}"] & \,
\ecd
and it follows from a simple diagram chase that $E_k \to L_k$ is a relative perfect obstruction theory. \end{proof}
We have thus produced a virtual pull-back morphism
\begin{equation*} k^!_{\text{v}} : A_*(\Q{0}{n}{\PP^N}{d}) \to A_*(\Q{0}{n}{X}{\beta}) \end{equation*}
and more generally for any cartesian diagram
\bcd
F \ar[r] \ar[d] \ar[rd,phantom,"\square" right] & G \ar[d] \\
\Q{0}{n}{X}{\beta} \ar[r,"k"] & \Q{0}{n}{\PP^N}{d}
\ecd
we get a virtual pull-back morphism:
\begin{equation*} k^!_{\text{v}} : A_*(G) \to A_*(F) \end{equation*}

\begin{lem} \label{Relative spaces pull back} For any $\alpha$ we have:
\begin{equation*} k^!_{\text{v}} [\Q{0}{\alpha}{\PP^N|H}{d}] = \virt{\Q{0}{\alpha}{X|Y}{\beta}} \end{equation*} \end{lem}

\begin{proof} Consider the following cartesian diagram:
\bcd
\om{Q}(X|Y) \ar[r,"f \times g"] \ar[d,"f"] \ar[rd,phantom,"\square"] & \om{Q}(X) \times \om{Q}(\PP^N|H) \ar[r,"\pi_1"] \ar[d,"k \times \Id"] \ar[rd,phantom,"\square"] & \om{Q}(X) \ar[d,"k"] \\
\om{Q}(\PP^N|H) \ar[r,"j \times \Id"] \ar[d,"j"] \ar[rd,phantom,"\square"] & \om{Q}(\PP^N) \times \om{Q}(\PP^N|H) \ar[r,"\pi_1"] \ar[d,"\Id \times j"] & \om{Q}(\PP^N) \\
\om{Q}(\PP^N) \ar[r,"\Delta"] & \om{Q}(\PP^N) \times \om{Q}(\PP^N)
\ecd

Then by commutativity of virtual pull-backs and their compatibility with flat pull-backs we have
\begin{align*} \virt{\om{Q}(X|Y)} & = \Delta^! \left( \virt{\om{Q}(X)} \times [ \om{Q}(\PP^N|H) ] \right) \\
& = \Delta^! \pi_1^* \virt{\om{Q}(X)} \\
& = \Delta^! \pi_1^* k_{\text{v}}^! [\om{Q}(\PP^N)] \\
& = \Delta^! k^!_{\text{v}} \pi_1^* [\om{Q}(\PP^N)] \\
& = \Delta^! k_{\text{v}}^! \left( [ \om{Q}(\PP^N) ]\times [\om{Q}(\PP^N | H) ] \right) \\
& = k_{\text{v}}^! \Delta^! \left( [ \om{Q}(\PP^N)] \times [\om{Q}(\PP^N|H) ] \right) \\
& = k_{\text{v}}^! [ \om{Q}(\PP^N|H) ]
\end{align*}
as required.
\end{proof}

Thus, the first two terms of the recursion formula pull back easily along $k$. It remains to consider the third term, namely the virtual class of the comb locus. We wish to prove:
\begin{lem} \label{Comb loci pull back} For any $\alpha$ we have:
\begin{equation*} k_{\text{v}}^! [D^\mathcal{Q}_{\alpha,k}(\PP^N|H,d)] = \virt{D^\mathcal{Q}_{\alpha,k}(X|Y,\beta)} \end{equation*} \end{lem}
We will require a number of preparatory results. We begin by considering a simpler moduli space than the comb locus, which we call the \ildef{centipede locus}. We fix a partition $A=(A_0,\ldots,A_r)$ of the marked points and a partition $B=(\beta_0, \ldots, \beta_r)$ of the curve class satisfying the usual stability conditions for the comb loci, and then consider the space:
\begin{equation*} \mathcal{K}(X,A,B) := \Q{0}{A_0 \cup \{ q_1, \ldots, q_r \}}{X}{\beta_0} \times_{X^r} \prod_{i=1}^l \Q{0}{A_i\cup\{q_i\}}{X}{\beta_i} \end{equation*}
Notice that the comb locus lives inside this space as a closed substack. We equip this space with a virtual class in the usual way by pulling back the product class along the diagonal. That is, we set
\begin{equation*} \mathcal{L}(X,A,B) :=  \Q{0}{A_0 \cup \{ q_1, \ldots, q_r \}}{X}{\beta_0} \times \prod_{i=1}^l \Q{0}{A_i\cup\{q_i\}}{X}{\beta_i} \end{equation*}
which we equip with the class:
\begin{equation*} \virt{\mathcal{L}(X,A,B)} := \virt{\Q{0}{A_0 \cup \{ q_1, \ldots, q_r \}}{X}{\beta_0}} \times \prod_{i=1}^l \virt{\Q{0}{A_i\cup\{q_i\}}{X}{\beta_i}} \end{equation*}
We then consider the cartesian diagram
\begin{equation} \label{Product diagram}
\begin{tikzcd}
\mathcal{K}(X,A,B) \ar[r,"g"] \ar[d,"\ev_q"] \ar[rd,phantom,"\square"] & \mathcal{L}(X,A,B) \ar[d,"\ev_q"] \\
X^r \ar[r,"\Delta_{X^r}"] & X^r \times X^r
\end{tikzcd}
\end{equation}
and define:
\begin{equation*} \virt{\mathcal{K}(X,A,B)} = \Delta_{X^r}^! (\virt{\mathcal{L}(X,A,B)}) \end{equation*}
On the other hand, we have the following cartesian diagram
\begin{equation} \label{Full quasimap diagram}
\begin{tikzcd}
\mathcal{K}(X,A,B) \ar[r,"\varphi"] \ar[d,"\rho"] \ar[rd,phantom,"\square"] & \Q{0}{n}{X}{\beta} \ar[d,"\pi"] \\
\MM_{0,A,B}^{\operatorname{wt}} \ar[r,"\psi"] & \MM_{0,n,\beta}^{\operatorname{wt}}
\end{tikzcd}
\end{equation}
where the moduli spaces on the bottom row are moduli spaces of weighted nodal curves, and we set:
\begin{equation*} \MM_{0,A,B}^{\operatorname{wt}} := \MM_{0,A_0\cup\{q_1,\ldots,q_r\},\beta_0}^{\operatorname{wt}} \times \prod_{i=1}^r \MM_{0,A_i\cup\{q_i\},\beta_i}^{\operatorname{wt}} \end{equation*}

\begin{remark} The horizontal maps are not injective: due to the existence of degree--$0$ components, there may be many possible equally valid ways of breaking up a nodal curve into pieces. For instance, consider the following example of two elements which map to the same curve under $f$.
[FIGURE]
\end{remark}

\begin{lemma} \label{Lemma product class equals pullback class} The virtual class of $\mathcal{K}(X,A,B)$ as defined previously is induced by a perfect obstruction theory relative to the morphism $\psi \circ \rho$ in diagram \eqref{Full quasimap diagram}. Furthermore, there exists a compatible triple $(E_{\psi \circ \rho}, E_{\pi}, E_{\varphi})$ and therefore a virtual pull-back morphism $\varphi_{\text{v}}^!$ such that:
\begin{equation*} \virt{\mathcal{K}(X,A,B)} = \varphi_{\text{v}}^! \virt{\Q{0}{n}{X}{\beta}} \end{equation*} \end{lemma}

\begin{proof} We start by considering the diagram \eqref{Product diagram} above. Our first goal is to obtain a perfect obstruction theory for $\mathcal{L}(X,A,B)$ over $X^r \times X^r$. Consider the following sequence of morphisms:
\bcd
\mathcal{L}(X,A,B) \ar[r,"\ev \times \pi"] \ar[rr, bend right = 20, "\pi"] & X^r \times X^r \times \MM_{0,A,B}^{\operatorname{wt}} \ar[r,"\pi_2"] & \MM_{0,A,B}^{\operatorname{wt}}
\ecd
This induces an exact triangle of contangent complexes:
\begin{equation*} (\ev \times \pi)^* L_{\pi_2} = \ev^* L_{X^r \times X^r} \to L_\pi \to L_{\ev \times \pi} \xrightarrow{[1]} \end{equation*}
By construction the product virtual fundamental class on $\mathcal{L}(X,A,B)$ is induced by a perfect obstruction theory:
\begin{equation*} E_\pi \to L_\pi \end{equation*}
On the other hand, there exists a natural map
\begin{equation*} \ev^* L_{X^r \times X^r} \to E_\pi \end{equation*}
which we now describe. The (dual of the) complex $E_\pi$ is concentrated in degrees $0$ and $1$ and is given fibrewise on $\mathcal{L}(X,A,B)$ by:
\begin{equation*} H^0(C,\mathcal{F}) \xrightarrow{0} H^1(C,\mathcal{F}) \end{equation*}
Here $\mathcal{F}$ is the sheaf on $C$ defined by the following short exact sequence:
\begin{equation} \label{Sequence defining F} 0 \to \OO_C^{\oplus r} \to \bigoplus_{\rho} L_\rho \to \mathcal{F} \to 0 \end{equation}
On the other hand (the dual of) $\ev^* L_{X^r \times X^r}$ is concentrated in degree $0$ and is given fibrewise as the direct sum
\begin{equation*} \bigoplus_q T_{u(q)} X \end{equation*}
where the sum runs over all the ``nodal'' marked points $q$ of the disconnected curve $C$. Since none of these are basepoints, we can restrict to a neighbourhood of each $q$ where we obtain a map $u: C \to X$. Then the sequence \eqref{Sequence defining F} is nothing more than the pullback along $u$ of the Euler sequence on $X$:
\begin{equation*} 0 \to \OO_X^{\oplus r} \to \bigoplus_\rho \OO(D_\rho) \to T_X \to 0 \end{equation*}
From this it follows that $\mathcal{F} \cong u^* T_X$ in a neighbourhood of each $q$, and hence we can evaluate at $q$ to obtain a map
\begin{equation*} H^0(C,\mathcal{F}) \to \bigoplus_q T_q X \end{equation*}
as required. Dualising we obtain a morphism $\ev^* L_{X^r \times X^r} \to E_\pi$ which we can complete to an exact triangle, obtaining a diagram:
\begin{equation} \label{Diagram ev times pi and pi}
\begin{tikzcd}
\ev^* L_{X^r \times X^r} \ar[r] \ar[d,"\Id"] & E_\pi \ar[r] \ar[d] & E_{\ev \times \pi} \ar[r,"{[1]}"] \ar[d] & \, \\
\ev^* L_{X^r \times X^r} \ar[r] & L_\pi \ar[r] & L_{\ev \times \pi} \ar[r,"{[1]}"] & \,
\end{tikzcd}
\end{equation}
We have thus obtained a perfect obstruction theory for the morphism $\ev \times \pi$. Now let us consider the other commuting diagram
\bcd
\mathcal{L}(X,A,B) \ar[r,"\ev \times \pi"] \ar[rr, bend right = 20, "\ev"] & X^r \times X^r \times \MM_{0,A,B}^{\operatorname{wt}} \ar[r,"\pi_1"] & X^r \times X^r
\ecd
which produces an exact triangle:
\begin{equation*} \pi^* L_{\MM_{0,A,B}^{\operatorname{wt}}} \to L_{\ev} \to L_{\ev \times \pi} \xrightarrow{[1]} \end{equation*}
By composing $E_{\ev \times \pi} \to L_{\ev \times \pi}$ with the connecting homomorphism we obtain a morphism of exact triangles:
\bcd
\pi^* L_{\MM_{0,A,B}^{\operatorname{wt}}} \ar[r] \ar[d,"\Id"] & E_{\ev} \ar[r] \ar[d] & E_{\ev \times \pi} \ar[r,"{[1]}"] \ar[d] & \, \\
\pi^* L_{\MM_{0,A,B}^{\operatorname{wt}}} \ar[r] & L_{\ev} \ar[r] & L_{\ev \times \pi} \ar[r,"{[1]}"] & \,
\ecd
Thus we obtain a perfect obstruction theory $E_{\ev} \to L_{\ev}$ for $\mathcal{L}(X,A,B)$ over $X^r \times X^r$ which, by the nature of its construction, induces the standard virtual class. 

Then by compatibility of the virtual class with pullback (\cite[Proposition 7.2]{BF}) we have that
\begin{equation*} g^* E_{\ev} \to g^* L_{\ev} = L_{\ev} \end{equation*}
is a perfect obstruction theory on $\mathcal{K}(X,A,B)$ over $X^r$ which induces the virtual fundamental class $\virt{\mathcal{K}(X,A,B)}$ as defined earlier.

We will now use this fact to express the virtual class as a class induced from a perfect obstruction theory over $\MM_{0,A,B}^{\operatorname{wt}}$. Consider diagram \eqref{Full quasimap diagram} above.

As in the previous argument we can consider the composition
\begin{equation*} \mathcal{K}(X,A,B) \xrightarrow{\ev \times \rho} X^r \times \MM_{0,A,B}^{\operatorname{wt}} \xrightarrow{\pi_2} \MM_{0,A,B}^{\operatorname{wt}} \end{equation*}
and thus obtain an exact triangle:
\begin{equation*} \ev^* L_{X^r} \to L_\rho \to L_{\ev \times \rho} \xrightarrow{[1]} \end{equation*}
Now, earlier we obtained a relative obstruction theory $E_{\ev \times \pi} \to L_{\ev \times \pi}$ on $\mathcal{L}(X,A,B)$. We can use the cartesian diagram
\bcd
\mathcal{K}(X,A,B) \ar[r,"g"] \ar[d,"\ev \times \rho"] \ar[rd,phantom,"\square"] & \mathcal{L}(X,A,B) \ar[d,"\ev \times \pi"] \\
X^r \times \MM_{0,A,B}^{\operatorname{wt}} \ar[r,"\Delta \times \Id"] & X^r \times X^r \times \MM_{0,A,B}^{\operatorname{wt}}
\ecd
to pull back this back to an obstruction theory $E_{\ev \times \rho} \to L_{\ev \times \rho}$ on $\mathcal{K}(X,A,B)$. We can then compose it with the above exact triangle to obtain:
\bcd
\ev^* L_{X^r} \ar[r] \ar[d,"\Id"] & E_\rho \ar[r] \ar[d] & E_{\ev \times \rho} \ar[r,"{[1]}"] \ar[d] & \, \\
\ev^* L_{X^r} \ar[r] & L_\rho \ar[r] & L_{\ev \times \rho} \ar[r,"{[1]}"] & \,
\ecd
So we have a perfect obstructon theory for $\mathcal{K}(X,A,B)$ over $\MM_{0,A,B}^{\operatorname{wt}}$ which by construction induces the class pulled back from $\mathcal{L}(X,A,B)$.

Finally, we can construct a map
\begin{equation*} \varphi^* E_\pi \to E_\rho \end{equation*}
as follows. By the diagram above, to construct a morphism $E_\rho^\vee \to \varphi^* E_\pi^\vee$ is the same thing as to construct a morphism $E_{\ev \times \rho}^\vee \to \varphi^* E_\pi^\vee$ such that the composition
\begin{equation*} \ev^*L_{X^r}^\vee [-1] \to E_{\ev \times \rho}^\vee \to \varphi^* E_\pi^\vee \end{equation*}
is zero. Recall that $E_{\ev \times \rho} = g^* E_{\ev \times \pi}$ and hence by diagram \eqref{Diagram ev times pi and pi} above we get a map $g^* E_{\ev \times \pi}^\vee \to \varphi^* E_\pi^\vee$ if we can produce a map $g^* E_\pi^\vee \to \varphi^* E_\pi^\vee$.

We are now dealing with sheaves that we understand. Consider the following diagram:
\bcd
g^* \tilde{\mathcal{C}} \ar[r,"\nu"] \ar[rd,"\eta" below] & \varphi^* \mathcal{C} \ar[r,"\alpha"] \ar[d,"\rho"] \ar[rd,phantom,"\square"] & \mathcal{C} \ar[d,"\pi"] \\
& \mathcal{K}(X,A,B) \ar[r,"\varphi"] & \Q{0}{n}{X}{\beta}
\ecd
Here $\tilde{C}$ is the universal (disconnected) curve over $\mathcal{L}(X,A,B)$, which we have pulled back to $\mathcal{K}(X,A,B)$. On the other hand $\varphi^* \mathcal{C}$ is isomorphic to the universal curve over $\mathcal{K}(X,A,B)$. Therefore the map $\nu : g^* \tilde{\mathcal{C}} \to \varphi^* \mathcal{C}$ is a partial normalisation map given by normalising the nodes which connect the ``trunk'' of the centipede to the ``legs.''

There are natural sheaves $\mathcal{F}$ and $\tilde{\mathcal{F}}$ on $\mathcal{C}$ and $g^* \tilde{\mathcal{C}}$ respectively, such that
\begin{align*} E_\pi^\vee & = \R \pi_* \mathcal{F} \\
g^* E_\pi^\vee & = \R \eta_* \tilde{\mathcal{F}} \end{align*}
from which we obtain:
\begin{equation*} \varphi^* E_\pi^\vee = \R \rho_* \alpha^* \mathcal{F} \end{equation*}
Now, since $\nu$ is a partial normalisation there is a well-defined (injective) morphism:
\begin{equation*} \OO_{\varphi^*{\mathcal{C}}} \to \nu_* \OO_{g^* \tilde{\mathcal{C}}} \end{equation*}
On the other hand it is clear that
\begin{equation*} \nu_* \tilde{\mathcal{F}} = \nu_* \OO_{g^* \tilde{\mathcal{C}}} \otimes \alpha^* \mathcal{F} \end{equation*}
so tensoring the above morphism with $\alpha^* \mathcal{F}$ we obtain
\begin{equation*} \alpha^* \mathcal{F} \to \nu_* \tilde{\mathcal{F}} \end{equation*}
to which we can apply $\R \rho_*$ to obtain a map
\begin{equation*} \R \rho_* \alpha^* \mathcal{F} \to \R \rho_* \nu_* \tilde{\mathcal{F}} = \R \eta_* \tilde{\mathcal{F}} \end{equation*}
which is the morphism $g^* E_\pi^\vee \to \varphi^* E_\pi^\vee$ that we promised.

It remains to show that the composition
\begin{equation*} \ev^*L_{X^r}^\vee [-1] \to E_{\ev \times \rho}^\vee \to \varphi^* E_\pi^\vee \end{equation*}
is zero. It is equivalent to consider the composition
\begin{equation*}  E_{\ev \times \rho}^\vee \to \varphi^* E_\pi^\vee \to \ev^*L_{X^r}^\vee \end{equation*}
which is obtained by a simple extension of the above construction: we consider the full short exact sequence
\begin{equation*} 0 \to \OO_{\varphi^* \mathcal{C}} \to \nu_* \OO_{g^* \tilde{\mathcal{C}}} \to \OO_q 0 \end{equation*}
where $q$ is the locus of nodes connecting the trunk to the spine; tensoring with $\alpha^*$ and pushing down, we obtain precisely $\ev^* L_{X^r}^\vee$. The composition is then zero by functoriality. Thus we finally obtain a map:
\begin{equation*} \varphi^* E_\pi \to E_\rho \end{equation*}
Hence we can construct an obstruction theory for the morphism $\varphi$, and we have a virtual pull-back morphism $\varphi^!_{\text{v}}$ such that
\begin{equation*} \varphi^!_{\text{v}} \virt{\Q{0}{n}{X}{\beta}} = \virt{\mathcal{K}(X,A,B)} \end{equation*}
as claimed. \end{proof}

With this lemma proven, consider the following cartesian diagram:
\bcd
\mathcal{K}(X,A,B) \ar[r] \ar[d,"\varphi"] \ar[rd,phantom,"\square"] & \mathcal{K}(\PP^N, A, i_* B) \times \Q{0}{n}{X}{\beta} \ar[d,"p \times \Id"] \\
\Q{0}{n}{X}{\beta} \ar[d,"k"] \ar[r,"k \times \Id"] \ar[rd,phantom,"\square"] & \Q{0}{n}{\PP^N}{d} \times \Q{0}{n}{X}{\beta} \ar[d,"\Id \times k"] \\
\Q{0}{n}{\PP^N}{d} \ar[r,"\Delta"] & \Q{0}{n}{\PP^N}{d} \times \Q{0}{n}{\PP^N}{d} 
\ecd
Note that $\Delta$ is a regular embedding because $\Q{0}{n}{\PP^N}{d}$ is smooth. All the other morphisms in the diagram admit relative obstruction theories as discussed above, and hence (see REFERENCE) admit virtual pullback morphisms. The following result is the quasimap analogue of \cite[Lemma 4.3]{Ga}.
\begin{lem} \label{Centipede locus pulls back} $\Delta^! \left( [ \mathcal{K}(\PP^N, A, i_* B)] \times \virt{\Q{0}{n}{X}{\beta}} \right) = \virt{\mathcal{K}(X,A,B)}$ \end{lem}
\begin{proof}
We make use of commutativity of Gysin maps (and virtual pull-backs) and their compatibility with push-forwards. We have:
\begin{align*} \Delta^! \left( [ \mathcal{K}(\PP^N,A,i_* B) ] \times \virt{\Q{0}{n}{X}{\beta}} \right) & = \Delta^! (p \times \Id)_{\text{v}}^! \left( [\Q{0}{n}{\PP^N}{d}] \times \virt{\Q{0}{n}{X}{\beta}} \right) \\
& = \varphi_{\text{v}}^! \Delta^! \left( [\Q{0}{n}{\PP^N}{d}] \times \virt{\Q{0}{n}{X}{\beta}} \right) \\
& = \varphi_{\text{v}}^! \Delta^! \left( [\Q{0}{n}{\PP^N}{d}] \times k_{E_k}^! [ \Q{0}{n}{\PP^N}{d} ] \right) \\
& = \varphi_{\text{v}}^! \Delta^! (\Id \times k)_{\text{v}}^! \left( [\Q{0}{n}{\PP^N}{d}] \times [ \Q{0}{n}{\PP^N}{d}]  \right) \\
& = \varphi_{\text{v}}^! k_{\text{v}}^! \Delta^!\left( [\Q{0}{n}{\PP^N}{d}] \times [ \Q{0}{n}{\PP^N}{d}]  \right)  \\
& = \varphi_{\text{v}}^! k_{\text{v}}^! [ \Q{0}{n}{\PP^N}{d} ] \\
& = \varphi_{\text{v}}^! \virt{\Q{0}{n}{X}{\beta}} \\
& = \Delta_{X^r}^! \virt{\mathcal{L}(X,A,B)} \\
& = \virt{\mathcal{K}(X,A,B)}
\end{align*}
where the penultimate equality follows by Lemma \ref{Lemma product class equals pullback class} above.
\end{proof}

\begin{proof}[Proof of Lemma \ref{Comb loci pull back}] We have a cartesian diagram:
\bcd
D^\mathcal{Q}_{\alpha,k}(X|Y,\beta) \ar[r,"k"] \ar[d] \ar[rd,phantom,"\square"] & D^\mathcal{Q}_{\alpha,k}(\PP^N|H,d) \ar[d] \\
\Q{0}{n}{X}{\beta} \ar[r,"k"] & \Q{0}{n}{\PP^N}{d}
\ecd
We can write $D^\mathcal{Q}_{\alpha,k}(X|Y,\beta)$ as the disjoint union of spaces
\begin{equation*} D^{\mathcal{Q}}(X|Y,A,B,M) = \Q{0}{A_0 \cup \{q_1, \ldots, q_r\}}{Y}{\beta_0} \times_{Y^r} \prod_{i=1}^r \Q{0}{\alpha^{(i)}\cup (m_i)}{X|Y}{\beta_i} \end{equation*}
where $A$ and $B$ are partitions of the marked points and curve class as before, and $M=(m_1,\ldots,m_r)$ records the intersection multiplicities at the nodes which connect the internal component to the external components (the spine of the comb to the teeth).

In order to make the following discussion readable, we will need to introduce some shorthand notation. We suppose that the data of $A$,$B$ and $M$ has been fixed, and set:
\begin{align*} D(X|Y) & := D^{\mathcal{Q}}(X|Y,A,B,M) \\
\overline{\mathcal{Q}}(Y) & := \Q{0}{A_0 \cup \{q_1, \ldots , q_r\}}{Y}{\beta_0} \\
\overline{\mathcal{Q}}_i(X|Y) & := \Q{0}{\alpha^{(i)} \cup (m_i)}{X|Y}{\beta_i} \\
\end{align*}
We then have a cartesian diagram:
\bcd
D(X|Y) \ar[r] \ar[d,"f"] \ar[rd,phantom,"\square"] & \overline{\mathcal{Q}}(Y) \times \prod_{i=1}^r \overline{\mathcal{Q}}_i(X|Y) \ar[d,"k \times \prod_{i=1}^r l_i"] \\
\mathcal{K}(X,A,B) \ar[r,"g"] \ar[d,"\ev"] \ar[rd,phantom,"\square"] & \mathcal{L}(X,A,B) \ar[d,"\ev"] \\
X^r \ar[r,"\Delta_{X^r}"] & X^r \times X^r
\ecd
where $k$ and $l_i$ are respectively the inclusions
\begin{align*} k:& \overline{\mathcal{Q}}(Y) \hookrightarrow \Q{0}{A_0 \cup \{q_1 \ldots, q_r\}}{X}{\beta_0} \\
l_i: & \overline{\mathcal{Q}}_i(X|Y) \hookrightarrow \Q{0}{A_i \cup \{ q_i \}}{X}{\beta_i} \end{align*}
Now by definition we have:
\begin{equation*} \virt{D(X|Y)} = \Delta_{X^r}^! \left( \virt{\overline{\mathcal{Q}}(Y)} \times \prod_{i=1}^r \virt{\overline{\mathcal{Q}}_i(X|Y)} \right) \end{equation*}
But on the other hand we have seen (REFERENCE LEMMA NOT YET DONE) that:
\begin{equation*} \virt{\overline{\mathcal{Q}}(Y)} \times \prod_{i=1}^r \virt{\overline{\mathcal{Q}}_i(X|Y)} = (k \times \Pi_{i=1}^r l_i)^!_{\text{v}} \virt{\mathcal{L}(X,A,B)} \end{equation*}
Thus we obtain:
\begin{equation*} \virt{D(X|Y)} = f^! g^! \virt{\mathcal{L}(X,A,B)} = f^! \virt{\mathcal{K}(X,A,B)} \end{equation*}
Finally, consider the cartesian diagram:
\bcd
D(X|Y) \ar[r] \ar[d,"f"] \ar[rd,phantom,"\square"] & D(\PP^N|H) \ar[d,"i"] \\
\mathcal{K}(X,A,B) \ar[r] \ar[d] \ar[rd,phantom,"\square"] & \mathcal{K}(\PP^N,A,B) \ar[d] \\
\Q{0}{n}{X}{\beta} \ar[r,"k"] & \Q{0}{n}{\PP^N}{d}
\ecd
We have from Lemma \ref{Centipede locus pulls back} that
\begin{equation*} \virt{\mathcal{K}(X,A,B)} = k^! [ \mathcal{K}(\PP^N,A,B)] \end{equation*}
from which it follows that:
\begin{align*} \virt{D(X|Y)} & = f^! k^! [ \mathcal{K}(\PP^N,A,B) ] \\
& = k^! i^! [ \mathcal{K}(\PP^N,A,B) ] \\
& = k^! [ D(\PP^N|H) ] \end{align*}
Taking the disjoint union, we obtain the required result. \end{proof}

\begin{thm} Let $X$ be a smooth and proper toric variety and let $Y \subseteq X$ be a very ample hypersurface (not necessarily toric). Then, with the set-up as in the preceding discussion, we have an equality
\begin{equation*} (\alpha_k \psi_k + ev_k^* [Y]) \virt{\Q{0}{\alpha}{X|Y}{\beta}} = \virt{\Q{0}{\alpha+e_k}{X|Y}{\beta}} + \virt{D^\mathcal{Q}_{\alpha,k}(X|Y,\beta)} \end{equation*}
in the Chow group of $\Q{0}{n}{X}{\beta}$. \end{thm}
\begin{proof} Pull-back the equation for $\PP^N$ relative $H$ using the map $k^!$, applying Lemmas \ref{Relative spaces pull back} and \ref{Comb loci pull back}. \end{proof}








\newpage

\section{The Quasimap Mirror Theorem}
Assuming that quasimap invariants for $\PP^r$ coincide with Gromov-Witten invariants on the nose, we get the following result.
\begin{dfn}
 For a complete intersection $X$ in $\PP^r$ and $d>0$, let
 \[
  I^X_d=(\ev_1)_*\left(\frac{1}{z-\psi_1}[\Q{0}{2}{X}{d}]^\text{vir}\right)
 \]
 where $\ev_1$ is always thought of as landing in $\PP^r$.
 
 Set also $I^X_0=\mathbbm 1_{\PP^r}$ and $I^X=\sum_{d\geq 0}I^X_d q^d$.
\end{dfn}
\begin{thm}
Let $X\subseteq\PP^4$ be a smooth quintic 3-fold. Then
 \[
  \sum_{d\geq 0} q^d\prod_{i=0}^{5d}(X+iz)I^{\PP^4}_d= XP(q)I^X
 \]
where
\[
 P(q)=1+\sum_{d>0}dq^d\langle H^4,\mathbbm 1_{\PP^4}\rangle_{\Q{0}{\{5d,0\}}{\PP^4|X}{d}}=1+\sum_{d>0}q^d\frac{(5d)!}{(d!)^5}\sum_{i=d+1}^{5d-1}\frac{1}{i}.
\]
\end{thm}
\begin{proof}
 We'll write it for a general CY hypersurface in $i\colon X_a\hookrightarrow\PP^r$, so the degree of $X$ is $a=r+1$. Notice that dual bases for $H^*(\PP^r)$ are given by $T^i=H^i$ and $T_i=H^{r-i}$, while (induced) dual bases for $i^*H^*(\PP^r)$ are $S^i=H^i$ and $S_i=\frac{1}{a}H^{r-i-1}$; the restriction of $H^r$ is 0.
 
 Define
 \[
  I^{\PP^r|X}_{d,(m)}=(\ev_1)_*\left(\frac{1}{z-\psi_1}[\Q{0}{\{m,0\}}{\PP^r|X}{d}]^\text{vir}\right),
 \]
which coincides with the absolute $I$-function defined above for $m=0$, and
\[
 J^{\PP^r|X}_{d,(m)}=(\ev_1)_*\left(m[\Q{0}{\{m,0\}}{\PP^r|X}{d}]^\text{vir}+\frac{1}{z-\psi_1}[D_m^{\mathcal Q}(\PP^r|X,d)]^\text{vir}\right).
\]
Then, by Gathmann's formula, we can prove that
\begin{equation}\label{eqn:G}
 (X+mz) I^{\PP^r|X}_{d,(m)}= I^{\PP^r|X}_{d,(m+1)}+ J^{\PP^r|X}_{d,(m)},
\end{equation}
from which it follows that
\[
 \prod_{i=0}^{ad}(X+iz) I^{\PP^r}_d=\sum_{m=0}^{ad}\prod_{i=m+1}^{ad}(X+iz)J^{\PP^r|X}_{d,(m)}.
\]
It is now a matter of evaluating the RHS. Notice that $J^{\PP^r|X}_{d,(m)}$ is made of two parts:
\begin{itemize}
 \item the boundary terms: the strong stability condition for quasimaps and the choice of working with only two markings makes these boundary contributions particularly simple to compute. The shape of the source curve is that of a snake which the hypersurface cuts into two pieces, the internal one of degree $d^{(0)}$, and the external one of degree $d^{(1)}$ and multiplicity $m^{(1)}$ of contact with $X$, with the first marking point belonging to the internal component and the second to the external one.
 
 The invariants which we need to consider will hence be of the form
 \[
  \langle T^i\psi_1^j,S_i\rangle_{\Q{0}{2}{X}{d^{(0)}}}\langle S^i,\mathbbm 1_{\PP^r}\rangle_{\Q{0}{\{m^{(1)},0\}}{\PP^r|X}.{d^{(1)}}}
 \]
 A dimensional computation
\begin{align*}
 0\leq \codim S_i &= \dim X-\codim S^i \\
 &= \dim X-\vdim \Q{0}{\{m^{(1)},0\}}{\PP^r|X}{d^{(1)}} \\
 &= \dim X-(\dim\PP^r-3+2-m^{(1)}-K_{\PP^r}\cdot d^{(1)}\ell)\\
 &= m^{(1)}-X\cdot d^{(1)}\ell+K_{X}\cdot d^{(1)}\ell \\
 &= m^{(1)}-X\cdot d^{(1)}\ell\leq 0
\end{align*}
forces $S_1=\mathbbm 1_X$ and $S^1=\frac{1}{a}H^{r-1}$, $m^{(1)}=ad^{(1)}$ hence
\[
 m=\alpha_1=X\cdot d^{(0)}\ell+m^{(1)}=ad,
\]
so this doesn't show up but at the very end of the ``increasing the multiplicity'' process.

\item The other term in $J^{\PP^r|X}_{d,(m)}$ is $m(\ev_1)_*[\Q{0}{\{m,0\}}{\PP^r|X}{d}]^\text{vir}$; notice that it only gets insertions from the cohomology of $\PP^r$ (restricted to $X$). On the other hand
\[
 \vdim \Q{0}{\{m,0\}}{\PP^r|X}{d}=r-3+2-m+d(r+1)\geq r-1
\]
because $m\leq ad$; since the restriction of $H^r$ to $X$ vanishes, the only insertion that contributes is $H^{r-1}$, forcing the equality $m=ad$.
\end{itemize}
So, in the end, we see that equation \ref{eqn:G} reduces to
\begin{align*}
 \prod_{i=0}^{ad}(X+iz) I^{\PP^r}_d &=J^{\PP^r|X}_{d,(ad)} \\
 &= \sum_{i=0,\ldots,r-1;j\geq 0}(da)\langle H^{r-1},\mathbbm 1_{\PP^r}\rangle_{\Q{0}{\{ad,0\}}{\PP^r|X}{d}} H \\
 &+\sum_{\substack{0<d^{(0)}<d \\ d^{(0)}+d^{(1)}=d}}z^{j+1}H^{r-i}\langle H^i\psi_1^j,\mathbbm 1_{X}\rangle_{\Q{0}{2}{X}{d^{(0)}}}(ad^{(1)})\langle \frac{1}{a}H^{r-1},\mathbbm 1_{\PP^r}\rangle_{\Q{0}{\{ad^{(1)},0\}}{\PP^r|X}{d^{(1)}}}\\
 &+z^{j+1}H^{r-i}\langle H^i\psi_1^j,\mathbbm 1_{X}\rangle_{\Q{0}{2}{X}{d}}
\end{align*}
from which the first claim of the theorem is now evident (with a bit of rearranging, using $X=aH$ and $i^*(H^r)=0$, so in the last line everything is divisible by $H$).

\smallskip

In order to evaluate $P(q)$, we use again Gathmann's algorithm, this time in the other direction, to go all the way back to $\PP^r$; then we make use of our assumption that quasimap invariants and ordinary GW coincide for the projective space. So it starts:
\[
 [\Q{0}{\{ad,0\}}{\PP^r|X}{d}]^\text{vir}=(X+(ad-1)\psi_1)[\Q{0}{\{ad-1,0\}}{\PP^r|X}{d}]^\text{vir}-[D_{ad}^{\mathcal Q}(\PP^r|X,d)]^\text{vir}
\]
When looking at the boundary, the invariants that come into play are of the form
\[
 \langle H^{r-1},S_i\rangle_{\Q{0}{2}{X}{d^{(0)}}}\langle S^i,\mathbbm 1_{\PP^r}\rangle_{\Q{0}{\{a(d-d^{(0)})-1,0\}}{\PP^r|X}{d-d^{(0)}}}
\]
but notice that they must vanish by dimensional reasons, since
\[
 \codim(S_i)=\dim X-3+2-K_X\cdot d^{(0)}\ell-(r-1)=-1.
\]
So
\begin{align*}
 d\langle H^{r-1},\mathbbm 1_{\PP^r}\rangle_{\Q{0}{\{ad,0\}}{\PP^r|X}{d}} \\
 = d[\Q{0}{2}{\PP^r}{d}]\cap\ev_1^*(H^{r-1})\prod_{i=0}^{ad-1}(\ev_1^*X+i\psi_1) \\
 = d[\Q{0}{2}{\PP^r}{d}]\cap\left((da-1)!\psi_1^{ad}\ev_1^*(H^{r-1})+a\left(\sum_{j=1}^{ad-1}\frac{(ad-1)!}{j}\right)\psi_1^{ad-1}\ev_1^*(H^{r})\right) \\
 = d\left((da-1)!\langle \psi_1^{ad-1}H^{r-1}\rangle_{0,1,d}+a\left(\sum_{j=1}^{ad-1}\frac{(ad-1)!}{j}\right)\langle\psi_1^{ad-2}H^{r}\rangle_{0,1,d}\right)
\end{align*}
using the equality of quasimap and GW invariants and the string equation for the latter. These numbers can be extracted from the $J$-function for $\PP^r$
\[
 I^{\PP^r}_d=\prod_{i=1}^d\frac{1}{H+i}
\]
from which
\begin{align*}
 \langle \psi_1^{ad-2}\ev_1^*(H^{r})\rangle_{0,1,d} &=\frac{1}{(d!)^{r+1}} \\
 \langle \psi_1^{ad-1}\ev_1^*(H^{r-1}\rangle_{0,1,d} &=-(r+1)\frac{1}{(d!)^{r+1}}\sum_{j=1}^d\frac{1}{j}
\end{align*}
and the second claim of the theorem follows.
\end{proof}




\appendix
\section{Functoriality of Quasimap Spaces} \label{Functoriality of Quasimap Spaces Section}

In the case of stable maps, a morphism $f : X \to Y$ induces a morphism between the corresponding moduli spaces
\begin{equation*}\M{g}{n}{X}{\beta} \rightarrow \M{g}{n}{Y}{f_* \beta} \end{equation*}
given by composition with $f$ (in general this induced morphism may involve stabilisation of the source curve). Because of this, the construction of the moduli space of stable maps is said to be \ildef{functorial}.

It is natural to ask whether the same holds for the moduli space of quasimaps. Since here the objects of the moduli space are not maps, we cannot simply compose with $f$, and indeed it is not immediately clear how we should proceed. In \cite[Section 3.1]{CF-K-wallcrossing} a definition is given when $f$ is an embedding into a projective space; however, this uses the more general language of GIT quotients which we seek to avoid here. As such, we will provide an alternative (but entirely equivalent) construction in the setting of toric varieties, which also relaxes the conditions on the map $f$ and the target $Y$.

\footnote{We should probably look a bit harder to see if the definition exists elsewhere.}

Our approach uses the language of $\Sigma$--collections introduced by D. Cox. This approach is natural insofar as a quasimap is a generalisation of a $\Sigma$--collection. We will refer extensively to \cite{CoxRing} and \cite{CoxFunctor}, which we recommend as an  introduction for any readers unfamiliar with the theory.

Let $X$ and $Y$ be smooth and proper toric varieties with fans $\Sigma_X \subseteq N_X$ and $\Sigma_Y \subseteq N_Y$. Suppose we are given $f : Y \to X$ (which we do not assume to be a toric morphism). By \cite[Theorem 1.1]{CoxFunctor} the data of such a map is equivalent to a $\Sigma_X$--collection on $Y$:
\begin{equation*} ( (L_\rho, u_\rho)_{\rho \in \Sigma_X(1)}, (\varphi_{m_x})_{m_x \in M_X} ) \end{equation*}
In addition, \cite{CoxRing} allows us to describe line bundles on $Y$ and their global sections in terms of the homogeneous coordinates $(z_\tau)_{\tau \in \Sigma_Y(1)}$. All of these observations are combined into the following theorem, which is so useful that we will state it here in its entirety:

\begin{thm} \cite[Theorem 2.2]{CoxFunctor} \label{CoxTheorem} The data of a morphism $f:Y \to X$ is the same as the data of homogeneous polynomials
\begin{equation*} P_\rho \in S^Y_{\beta_\rho} \end{equation*}
for $\rho \in \Sigma_X(1)$, where $\beta_\rho \in \Pic Y$ and $S^Y_{\beta_\rho}$ is the corresponding graded piece of the Cox ring
\begin{equation*}S^Y = k[z_\tau : \tau \in \Sigma_Y(1)]\end{equation*}
This data is required to satisfy the following two conditions:
\begin{enumerate}
\item $\sum_{\rho \in \Sigma_X(1)} \beta_\rho \otimes n_\rho = 0$ in $\Pic Y \otimes N_X$.
\item $(P_\rho(z_\tau)) \notin Z(\Sigma_X) \subseteq \Aaff_k^{\Sigma_X(1)}$ whenever $(z_\tau) \notin Z(\Sigma_Y) \subseteq \Aaff_k^{\Sigma_Y(1)}$.
\end{enumerate}
Furthermore, two such sets of data $(P_\rho)$ and $(P^\prime_\rho)$ correspond to the same morphism if and only if there exists a $\lambda \in \Hom_\Z(\Pic X, \Gm)$ such that
\begin{equation*} \lambda(D_\rho) \cdot P_\rho = P^\prime_\rho \end{equation*}
for all $\rho \in \Sigma_X(1)$. Finally, if we define $\tilde{f}(z_\tau) = (P_\rho(z_\tau))$ then this defines a lift of $f$ to the prequotients:
\bcd
\Aaff_k^{\Sigma_Y(1)} \setminus Z(\Sigma_Y) \ar[r, "\tilde{f}"] \ar[d, "\pi"] & \Aaff_k^{\Sigma_X(1)} \setminus Z(\Sigma_X) \ar[d,"\pi"] \\
Y \ar[r, "f"] & X
\ecd
\end{thm}
\begin{aside} Throughout this section we will stick to the notation established above; in particular we will use $\rho$ to denote a ray in $\Sigma_X(1)$ and $\tau$ to denote a ray in $\Sigma_Y(1)$. \end{aside}

Recall our goal: given a map $f : Y \to X$ we wish to define a ``push-forward'' map:
\begin{equation*} f_* : \Q{g}{n}{Y}{\beta} \to \Q{g}{n}{X}{f_*\beta} \end{equation*}
Consider therefore a quasimap $(C, (L_\tau, u_\tau)_{\tau \in \Sigma_Y(1)}, (\varphi_{m_Y})_{m_Y \in M_Y})$ with target $Y$. Pick data $(P_\rho)_{\rho \in \Sigma_X(1)}$ corresponding to the map $f$, as in the theorem above; we will later see that our construction does not depend on this choice.

The idea of the construction is as follows. Let us pretend for a moment that $C$ is toric and that the quasimap is without basepoints, so that we have an actual morphism $C \to Y$. Then we can lift this morphism to the prequotient as in the following diagram

\bcd
\Aaff_k^{\Sigma_C(1)} \setminus Z(\Sigma_C) \ar[r, "(u_\tau)"] \ar[d] & \Aaff_k^{\Sigma_Y(1)} \setminus Z(\Sigma_Y) \ar[r, "(P_\rho)"] \ar[d] & \Aaff_k^{\Sigma_X(1)} \setminus Z(\Sigma_X) \ar[d] \\
C \ar[r] & Y \ar[r] & X
\ecd
from which it follows that the composition $C \to Y \to X$ is given in homogeneous coordinates by:
\begin{equation*} (P_\rho((u_\tau)_{\tau \in \Sigma_Y(1)}))_{\rho \in \Sigma_X(1)} \end{equation*}
In general of course $C$ is not a toric variety and the quasimap is not basepoint-free. Nevertheless, as we will see, we can still make sense of the expression $P_\rho(u_\tau)$ as a section of a line bundle on $C$. This will allow us to define the pushforward of our quasimap.

Let us begin. For each $\rho$, $P_\rho$ is a polynomial in the $z_\tau$; we can write it as
\begin{equation} \label{Prho} P_\rho(z_\tau) = \sum_{\underline{a}} P_\rho^{\underline{a}}(z_\tau) = \sum_{\underline{a}} \mu_{\underline{a}} \prod_{\tau} z_{\tau}^{a_{\tau}} \end{equation}
where the sum is over a finite number of multindices $\underline{a} = (a_\tau) \in \N^{\Sigma_Y(1)}$ and the $\mu_{\underline{a}}$ are nonzero scalars. For each $\underline{a}$ consider the following line bundle on $C$:
\begin{equation*} \tilde{L}_\rho^{\underline{a}} = \bigotimes_\tau L_\tau^{\otimes a_\tau} \end{equation*}
Then we may take the following section of $\tilde{L}_\rho^{\underline{a}}$:
\begin{equation*} \tilde{u}_\rho^{\underline{a}} = P_\rho^{\underline{a}}(u_\tau) = \mu_{\underline{a}} \prod_\tau u_\tau^{a_\tau} \end{equation*}
Thus each of the terms $P_\rho^{\underline{a}}$ of $P_\rho$ defines a section $\tilde{u}_\rho^{\underline{a}}$ of a line bundle $\tilde{L}_\rho^{\underline{a}}$. But what we want is a single section $\tilde{u}_\rho$ of a single line bundle $\tilde{L}_\rho$. This is where the isomorphisms $\varphi_{m_Y}$ come in.

Recall that we have a short exact sequence:
\begin{equation} \label{Pic short exact sequence for Y} 0 \longrightarrow M_Y \overset{\theta}{\longrightarrow} \Z^{\Sigma_Y(1)} \longrightarrow \Pic Y \longrightarrow 0 \end{equation}
Let $\underline{a}$ and $\underline{b}$ be multindices appearing in the sum \eqref{Prho} above. By the homogeneity of $P_\rho$ we have
\begin{equation*} \sum_\tau a_\tau D_\tau = \beta_\rho = \sum_\tau b_\tau D_\tau \end{equation*}
which is precisely the statement that in the above sequence $\underline{a}$ and $\underline{b}$ map to the same element of $\Pic Y$ (namely $\beta_\rho$). Hence there exists an $m_Y \in M_Y$ such that:
\begin{equation*} \theta(m_Y) = \underline{a} - \underline{b} \end{equation*}
Now, the isomorphism $\varphi_{m_Y}$ (contained in the data of our original quasimap) is a map:
\begin{equation*} \varphi_{m_Y} : \bigotimes_\tau L_\tau^{\otimes \langle m_Y, n_\tau \rangle} \cong \OO_C \end{equation*}
By definition, $\theta(m_Y) = (\langle m_Y,n_\tau \rangle)_{\tau \in \Sigma_Y(1)}$. But also $\theta(m_Y) = (a_\tau - b_\tau)_{\tau \in \Sigma_Y(1)}$. Hence we have:
\begin{equation*} \varphi_{m_Y} : \bigotimes_\tau L_\tau^{\otimes a_\tau} \cong \bigotimes_\tau L_\tau^{\otimes b_\tau} \end{equation*}
In other words, we have well-defined canonical isomorphisms
\begin{equation*} \tilde{L}_\rho^{\underline{a}} \cong \tilde{L}_\rho^{\underline{b}} \end{equation*}
for all $\underline{a}$ and $\underline{b}$. Let us choose one such $\underline{a}$ (it doesn't matter which); call it $\underline{a}^\rho$. We define:
\begin{equation*} \tilde{L}_\rho = \tilde{L}_\rho^{\underline{a}^\rho} \end{equation*}
Then for all $\underline{b}$ we can use the above isomorphism to view $\tilde{u}_\rho^{\underline{b}}$ as a section of $\tilde{L}_\rho$. Summing all of these together we obtain a section $\tilde{u}_\rho$ of $\tilde{L}_\rho$, which we can write (with abuse of notation) as:
\begin{equation*} \tilde{u}_\rho = \sum_{\underline{a}} \mu_{\underline{a}} \prod_\tau u_\tau^{a_\tau} \end{equation*}
Note that if we had made a different choice of $\underline{a}^\rho$ above the result would have been isomorphic.

Thus far we have constructed line bundles and sections $(\tilde{L}_\rho, \tilde{u}_\rho)_{\rho \in \Sigma_X(1)}$ on $C$. It remains to define the isomorphisms
\begin{equation*} \tilde{\varphi}_{m_X} : \otimes_\rho \tilde{L}_\rho^{\otimes \langle m_X, n_\rho \rangle} \cong \OO_C \end{equation*}
for all $m_X \in M_X$. The left hand side is:
\begin{align*} \otimes_\rho \tilde{L}_\rho^{\otimes \langle m_X, n_\rho \rangle} & = \otimes_\rho \left( \otimes_\tau L_\tau^{\otimes a_\tau^\rho} \right)^{\otimes \langle m_X, n_\rho \rangle} = \otimes_\tau L_\tau^{\otimes \left( \sum_{\rho} a_\tau^\rho  \langle m_X, n_\rho \rangle \right)} \end{align*}
Now, for $m_Y \in M_Y$ we have isomorphisms $\varphi_{m_Y} : \otimes_\tau L_\tau^{\otimes \langle m_Y, n_\tau \rangle} \cong \OO_C$. Hence, in order to construct $\tilde{\varphi}_{m_X}$ we need to find an $m_Y$ such that
\begin{equation*} \langle m_Y, n_\tau \rangle = \sum_\rho a_\tau^\rho \langle m_X, n_\rho \rangle \end{equation*}
for all $\tau \in \Sigma_Y(1)$ (we will then set $\tilde{\varphi}_{m_X} = \varphi_{m_Y}$). Consider therefore the short exact sequence \eqref{Pic short exact sequence for Y}. Recall that $\theta(m_Y) = (\langle m_Y, n_\tau \rangle)_{\tau \in \Sigma_Y(1)}$. Hence we need to show that
\begin{equation*} \left( \sum_\rho a_\tau^\rho \langle m_X, n_\rho \rangle \right)_{\tau \in \Sigma_Y(1)} \end{equation*}
belongs to the image of $\theta$, i.e. that it belongs to the kernel of the second map (notice that $m_Y$ is then unique because $\theta$ is injective). This is equivalent to saying that
\begin{equation*} \sum_\tau \sum_\rho a_\tau^\rho \langle m_X, n_\rho \rangle D_\tau = 0 \in \Pic Y \end{equation*}
Now, we have
\begin{equation*} \sum_\tau a_\tau^\rho D_\tau = \beta_\rho \end{equation*}
so that the above sum becomes
\begin{equation*} \sum_\rho \langle m_X, n_\rho \rangle \beta_\rho = \left\langle m_X, \sum_\rho \beta_\rho \otimes n_\rho \right \rangle = \langle m_X, 0 \rangle = 0 \end{equation*}
where $\sum_\rho \beta_\rho \otimes n_\rho = 0$ by Condition (1) in Theorem \ref{CoxTheorem}. So there does indeed exist a (unique) $m_Y \in M_Y$ such that $\langle m_Y, n_\tau \rangle = \sum_\rho a_\tau^\rho \langle m_X, n_\rho \rangle$, so that we can set:
\begin{equation*} \tilde{\varphi}_{m_X} = \varphi_{m_Y} : \bigotimes_\rho \tilde{L}_\rho^{\otimes \langle m_X, n_\rho \rangle} \cong \OO_C \end{equation*}
Thus, we have produced a quasimap with target $X$:
\begin{equation*} (C, (\tilde{L}_\rho, \tilde{u}_\rho)_{\rho \in \Sigma_X(1)}, (\tilde{\varphi}_{m_X})_{m_X \in M_X}) \end{equation*}
The proof that this construction does not depend on the choice of $(P_\rho)$ is straightforward and is left to the reader.

It remains to demonstrate that the quasimap thus constructed is nondegenerate and stable. Nondegeneracy follows immediately from Condition (2) in Theorem \ref{CoxTheorem}. Put differently: the original quasimap defined a rational map $C \dashrightarrow Y$, whereas the new quasimap defines a rational map which is simply the composition $C \dashrightarrow Y \to X$. Therefore the set of basepoints is exactly the same.

Stability is a bit more tricky: it is here that we will end up having to put some extra conditions on the map $f$. First, notice that there are no rational tails because the source curve is unchanged.

Next let $C^\prime \subseteq C$ be a component with exactly $2$ special points. Then we need to show (see \cite[Definition 3.1.1]{CF-K}) that the following line bundle has positive degree on $C^\prime$:
\begin{equation*} \tilde{\mathcal{L}} = \bigotimes_\rho \tilde{L}_\rho^{\otimes \tilde{\alpha}_\rho} \end{equation*}
Here the $\tilde{\alpha}_\rho$ are defined by fixing a polarisation on $X$:
\begin{equation*} \OO_X(1) = \bigotimes_\rho \OO_X(\tilde{\alpha}_\rho D_\rho) \end{equation*}
The choice of polarisation makes no difference: a quasimap is stable with respect to one polarisation if and only if it is stable with respect to all others. In order to make use of the fact that the original quasimap to $Y$ was stable, we will make the following assumption on $f$:
\begin{enumerate}
\item there exists an ample line bundle $\OO_X(1)$ on $X$ such that $f^*\OO_X(1)$ is ample on $Y$
\end{enumerate}
This is satisfied if, for example, $f$ is an embedding (which is the only case we will need in this paper). Given this assumption, we can set $\OO_Y(1) = f^*\OO_X(1)$. We then have:
\begin{align*} \OO_Y(1) & = \bigotimes_\rho f^*\OO_X(D_\rho)^{\otimes \tilde{\alpha}_\rho} = \bigotimes_\rho \OO_Y (\sum_\tau a_\tau^\rho D_\tau)^{\otimes \tilde{\alpha}_\rho} \\
& = \bigotimes_\rho \bigotimes_\tau \OO_Y(a_\tau^\rho \tilde{\alpha}_\rho D_\tau) = \bigotimes_\tau \OO_Y(D_\tau)^{\otimes \sum_\rho a_\tau^\rho \tilde{\alpha}_\rho}\end{align*}
Thus for $\tau \in \Sigma_Y(1)$ we have $\alpha_\tau = \sum_\rho a_\tau^\rho \tilde{\alpha}_\rho$ and by stability of the original quasimap the line bundle $\mathcal{L} = \otimes_\tau L_\tau^{\otimes \alpha_\tau}$ has positive degree on $C^\prime$. But:
\begin{equation*} \mathcal{L} = \bigotimes_\tau L_\tau^{\otimes \alpha_\tau} = \bigotimes_\rho \bigotimes_\tau \left( L_\tau^{\otimes a_\tau^\rho} \right)^{\otimes \tilde{\alpha}_\rho} = \bigotimes_\rho \tilde{L}_\rho^{\otimes \tilde{\alpha}_\rho} = \tilde{\mathcal{L}} \end{equation*}
We have thus proven that $\tilde{\mathcal{L}}$ has positive degree on $C^\prime$, so the pushed-forward quasimap is stable. This completes the proof of the following.

\begin{thm} Let $X$ and $Y$ be smooth proper toric varieties and $f : Y \to X$ a morphism. Assume that $f$ satisfies Condition (1) above. Then there exists a natural push-forward map
\begin{equation*} f_* : \Q{g}{n}{Y}{\beta} \to \Q{g}{n}{X}{f_* \beta} \end{equation*}
which does not modify the underlying prestable curves.\end{thm}

\begin{aside} We expect that such a map exists even if $f$ does not satisfy Condition (1). However, in this case we will need to modify the underlying prestable curves by contracting unstable components. The same is true in the stable maps case. \end{aside}

Finally, let us describe how this push-forward morphism behaves when $f$ is a nonconstant map $\PP^r \to \PP^N$, since we will make use of this later. Write $f$ in homogeneous coordinates as:
\begin{equation*} f[z_0, \ldots, z_r] = [f_0(z_0, \ldots, z_r), \ldots, f_N(z_0, \ldots, z_r)] \end{equation*}
where the $f_i$ are all homogeneous of degree $a$. Then given a quasimap with target $\PP^r$
\begin{equation*} (C, L, u_o, \ldots, u_r) \end{equation*}
the pushed-forward quasimap with target $\PP^N$ is:
\begin{equation*} (C, L^{\otimes a}, f_0(u_0, \ldots, u_r) , \ldots, f_N(u_0, \ldots, u_r)) \end{equation*}
(This is stable as long as $a > 0$, which is precisely when $f$ satsfies Condition (1) above.)

\section{The Quasimap String Equation for $\PP^r$}

The string equation for the Gromov--Witten invariants of a smooth projective variety $X$ is given by
\begin{align*} \langle \mathbbm{1} , \gamma_1 \psi^{a_1} , \ldots, & \gamma_n \psi^{a_n} \rangle_{g,n+1,\beta}^X = \\
&  \sum_{i=1}^n \langle \gamma_1 \psi^{a_1}, \ldots, \gamma_{i-1} \psi^{a_{i-1}} , \gamma_i \psi^{a_i - 1} , \gamma_{i+1} \psi^{a_{i+1}} \ldots, \gamma_n \psi^{a_n} \rangle_{g,n,\beta}^X \end{align*}
where $\mathbbm{1} \in H^*(X)$ is the unit class (by convention any term involving a negative power of $\psi$ is set to zero). Since Gromov--Witten invariants and quasimap invariants coincide for $X=\PP^r$ (\cite[Section 5.4]{Manolache-Push}) we know that the same equation holds for quasimap invariants to $\PP^r$.

Nevertheless, it would be illuminating to have a direct proof of this statement, without relying on the equivalence with Gromov--Witten theory. Amongst other things, such a proof would necessarily involve some nontrivial intersection computations in the cohomology ring of the quasimap space, which would be of independent interest.

The proof of the classical string equation (for Gromov--Witten invariants) relies on three key lemmas involving certain codimension--1 classes on the moduli space of stable maps. Let
\begin{equation*} \pi : \M{g}{n+1}{X}{\beta} \to \M{g}{n}{X}{\beta}\end{equation*}
denote the contraction map given by forgetting the last marked point and stabilising. Then we have:

\begin{enumerate}
\item $\psi_i = \pi^* \psi_i + D_{i,n+1}$
\item $\psi_i \cdot D_{i,n+1} = 0$
\item $D_{i,n+1} \cdot D_{j,n+1} = 0$ for $i \neq j$
\end{enumerate}

Here $D_{i,n+1}$ is  the locus of stable maps $(C,x_1, \ldots, x_{n+1}, f)$ such that we can split up $C$ into two pieces, $C = C^\prime \cup C^{\prime\prime}$ (intersecting in a single node) such that $C^{\prime\prime}$ has degree $0$ and contains only the markings $x_i$ and $x_{n+1}$.

[FIGURE]

We would like to have some analogue of these results in the quasimap setting. In fact, equations (2) and (3) carry over without difficulty. Equation (1), on the other hand, is rather more delicate.

In the stable map setting, equation (1) is proved by considering the following diagram
\bcd
\mathcal{C}_{g,n+1} \ar[r,"\rho"] \ar[dr, "\psi" below left] & \pi^* \mathcal{C}_{g,n} \ar[r,"\alpha"] \ar[d, "\eta"] & \mathcal{C}_{g,n} \ar[d,"\varphi"] \\
& \M{g}{n+1}{X}{\beta} \ar[r,"\pi"] & \M{g}{n}{X}{\beta}
\ecd
where the square on the right is cartesian. On fibres, the map $\rho$ contracts rational components of $\mathcal{C}_{g,n+1}$ on which $f$ is constant and which contain exactly three special points, one of which is $x_{n+1}$. Thus, we see that
\begin{equation*} \rho^*(x_i) = x_i + R_{i,n+1} \end{equation*}
where $R_{i,n+1} \subseteq C_{g,n+1}$ consists fibrewise of the rational tails containing only $x_i$ and $x_{n+1}$; it is a closed substack of $\psi^{-1}(D_{i,n+1})$ of codimension $0$.

On the other hand, we have (REFERENCE):
\begin{equation*} \rho^* \omega_\eta(\Sigma_{i=1}^n x_i) = \omega_\psi(\Sigma_{i=1}^n x_i) \end{equation*}
Taking Chern classes and combining this with the above result we obtain:
\begin{equation*} \operatorname{c}_1(\rho^* \omega_\eta) = \operatorname{c}_1(\omega_\psi) - \Sigma_{i=1}^n R_{i,n+1} \end{equation*}
We can now pull back along the section $x_i$ and  use the fact that $x_i^* R_{j,n+1} = \delta_{i,j} D_{i,n+1}$ to obtain:
\begin{equation*} \operatorname{c}_1(x_i^*\rho^* \omega_\eta) = \operatorname{c}_1(x_i^* \omega_\psi) - D_{i,n+1} \end{equation*}
Now, $\rho^* \omega_\eta = \rho^* \alpha^* \omega_\varphi$, and so:
\begin{equation*} x_i^* \rho^* \omega_\eta = \pi^* x_i^* \omega_\varphi \end{equation*}
Thus we end up with
\begin{equation*} \pi^*\operatorname{c}_1(x_i^* \omega_\varphi) = \operatorname{c}_1(x_i^* \omega_\psi) - D_{i,n+1} \end{equation*}
which is equation (1) above.

What is different in the case of quasimaps? We have a similar-looking diagram
\bcd
\mathcal{C}_{g,n+1} \ar[r,"\rho"] \ar[dr, "\psi" below left] & \pi^* \mathcal{C}_{g,n} \ar[r,"\alpha"] \ar[d, "\eta"] & \mathcal{C}_{g,n} \ar[d,"\varphi"] \\
& \Q{g}{n+1}{X}{\beta} \ar[r,"\pi"] & \Q{g}{n}{X}{\beta}
\ecd
but now, because of the stronger stability condition, $\rho$ also contracts the locus $T_{n+1}$ consisting of rational tails (of any degree) with a single marking $x_{n+1}$. We claim that:

\begin{conj} $\rho^* \omega_\eta ( \Sigma_{i=1}^n x_i ) = \omega_\psi (\Sigma_{i=1}^n x_i - T_{n+1})$ \end{conj}

Once we have this, the string equation follows as in the stable maps case by pulling back along the section $x_i$ (and using the obvious fact that $x_i^* T_{n+1} = 0$).

\bibliographystyle{alpha}
\bibliography{relqm}

\end{document}
