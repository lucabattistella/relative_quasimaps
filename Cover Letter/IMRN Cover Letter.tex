 \documentclass[11pt]{amsart}

\usepackage{setspace}
\onehalfspacing

\usepackage{accents}
\usepackage[all]{xy}
\usepackage{amsfonts}
\usepackage{amsmath}
\usepackage{amssymb}
\usepackage{amsthm}
\usepackage{cite}
\usepackage{color}
\usepackage{epstopdf}
\usepackage{fancyhdr}
\usepackage{float}
\usepackage{graphicx}
\usepackage{hyperref}
\usepackage{latexsym}
\usepackage{mathrsfs}
\usepackage{natbib}
\usepackage{sseq}
\usepackage{tikz-cd}
\usepackage{url}
\usepackage{verbatim}

%\usepackage{makeidx}
%\makeindex

\usepackage[full]{textcomp}
\usepackage[sups]{Baskervaldx}
\usepackage{cabin}
\usepackage[varqu,varl]{inconsolata}
\usepackage[baskervaldx,bigdelims,vvarbb]{newtxmath}
\usepackage[cal=cm]{mathalfa}

\usepackage[margin=1in]{geometry}

%\usepackage{moreverb}
%\usepackage{mathtools}
%\usepackage{marginnote}
%\usepackage{pifont}
%\usepackage{pictexwd,dcpic}
%\usepackage{setspace}
%\usepackage{lastpage}

\theoremstyle{plain}
\newtheorem{theorem}{Theorem}[section]

\newtheorem{claim}[theorem]{Claim}
\newtheorem{conjecture}[theorem]{Conjecture}
\newtheorem{corollary}[theorem]{Corollary}
\newtheorem{lemma}[theorem]{Lemma}
\newtheorem{proposition}[theorem]{Proposition}
%\newtheorem{question}[theorem]{Question}

\theoremstyle{remark}
\newtheorem{remark}[theorem]{Remark}

\theoremstyle{definition}
\newtheorem{aside}[theorem]{Aside}
\newtheorem*{aside*}{Aside}
\newtheorem*{question*}{Question}
\newtheorem{condition}[theorem]{Condition}
\newtheorem{construction}[theorem]{Construction}
\newtheorem{convention}[theorem]{Convention}
\newtheorem{definition}[theorem]{Definition}
\newtheorem{example}[theorem]{Example}
\newtheorem{exercise}{Exercise}
\newtheorem{notation}[theorem]{Notation}
\newtheorem{proposition-definition}[theorem]{Proposition-Definition}
\newtheorem{question}[theorem]{Question}
\newtheorem{setting}[theorem]{Setting}

%\numberwithin{equation}{section}

\DeclareMathOperator{\ev}{ev}
\DeclareMathOperator{\vdim}{vdim}
\DeclareMathOperator{\Hom}{Hom}
\DeclareMathOperator{\mult}{mult}
\DeclareMathOperator{\id}{id}
\DeclareMathOperator{\Bl}{Bl}
\DeclareMathOperator{\Pic}{Pic}
\DeclareMathOperator{\frPic}{\mathfrak{Pic}}
\DeclareMathOperator{\GW}{GW}
\DeclareMathOperator{\Achow}{A}
\DeclareMathOperator{\pt}{pt}

\newcommand{\virt}[1]{[#1]^{\operatorname{virt}}}
\newcommand{\M}[4]{\overline{\mathcal{M}}_{#1,#2}(#3,#4)}
\newcommand{\Q}[4]{\overline{\mathcal{Q}}_{#1,#2}(#3,#4)}
\newcommand{\C}{\mathbb{C}}
\newcommand{\T}{\mathbb{T}}
\newcommand{\G}{\mathbb{G}}
\newcommand{\PP}{\mathbb{P}}
\newcommand{\OO}{\mathcal{O}}
\newcommand{\N}{\mathbb{N}}
\newcommand{\Z}{\mathbb{Z}}
\newcommand{\QQ}{\mathbb{Q}}
\newcommand{\A}{\mathbb{A}}
\newcommand{\R}{\mathbb{R}}
\newcommand{\CP}{\mathbb{CP}}
\newcommand{\PD}{\mathrm{PD}}
\newcommand{\HH}{\operatorname{H}}

\newcommand{\mg}{\mathcal{M}_g}
\newcommand{\mgbar}{\bar{\mathcal{M}}_g}
\newcommand{\mgnbar}{\bar{\mathcal{M}}_{g,n}}
\newcommand{\monbar}{\bar{\mathcal{M}}_{0,n}}

\newcommand{\FF}{\mathcal{F}}
\newcommand{\calChat}{\hat{\cC}}
\newcommand{\calC}{\mathcal{C}}
\newcommand{\calO}{\mathcal{O}}
\newcommand{\calQ}{\mathcal{Q}}
\newcommand{\calL}{\mathcal{L}}
\newcommand{\calLbar}{\underline{\mathcal{L}}}
\newcommand{\frC}{\mathfrak{C}}
\newcommand{\frCbar}{\underline{\mathfrak{C}}}
\newcommand{\frD}{\mathfrak{D}}
\newcommand{\calM}{\mathcal{M}}
\newcommand{\calI}{\mathcal{I}}
\newcommand{\calE}{\mathcal{E}}
\newcommand{\frM}{\mathfrak{M}}
\newcommand{\frN}{\mathfrak{N}}
\newcommand{\frNbar}{\underline{\mathfrak{N}}}
\newcommand{\frX}{\mathfrak{X}}
\newcommand{\frXbar}{\underline{\mathfrak{X}}}
\newcommand{\Cstar}{\C^\times}
\newcommand{\pibar}{\underline{\pi}}
\newcommand{\ubar}{\underline{u}}
\newcommand{\frA}{\mathfrak{A}}
\newcommand{\frAbar}{\underline{\mathfrak{A}}}
\newcommand{\frd}{\mathfrak{d}}
\newcommand{\frF}{\mathfrak{F}}
\newcommand{\frFbar}{\underline{\mathfrak{F}}}
\newcommand{\sbar}{\underline{s}}
\newcommand{\frG}{\mathfrak{G}}
\newcommand{\frL}{\mathfrak{L}}
\newcommand{\frP}{\mathfrak{P}}
\newcommand{\frLbar}{\underline{\mathfrak{L}}}
\newcommand{\vhat}{\overline{v}}
\newcommand{\frE}{\mathfrak{E}}
\newcommand{\Cbar}{\underline{C}}
\newcommand{\Lbar}{\underline{L}}
\newcommand{\vbar}{\overline{v}}

\newcommand{\red}{\text{red}}

\newcommand{\im}{\text{im }}
\newcommand{\tick}{\ding{52}}
\newcommand{\quo}[1]{#1/\mathord{\sim}}
\newcommand{\sltwo}{\operatorname{SL_2}(\mathbb{Z})}
\newcommand{\sltwor}{\operatorname{SL_2}(\mathbb{R})}
\newcommand{\dd}{\text{d}}
\newcommand{\grad}{\text{grad}}
\newcommand{\ind}[1]{\text{Ind}(#1)}

\newcommand{\Newt}[1]{\text{Newt}(#1)}
\newcommand{\conv}{\text{conv}}
\newcommand{\val}{\text{val}}
\newcommand{\rank}[1]{\text{rank} \ #1}

\newcommand{\MnX}{\overline{\mathcal{M}}_{g,n}(X,\beta)}
\newcommand{\MnXD}{\overline{\mathcal{M}}_{g,n}(X/D,\beta,\underline{\alpha})}
\newcommand{\Spec}{\text{Spec}}
\newcommand{\acts}{\curvearrowright}

\def\bibfont{\footnotesize}

\begin{document}
 
\title{Cover Letter for IMRN submission\\ ``Relative quasimaps and mirror formulae''}
\author{Luca Battistella and Navid Nabijou}
\date{31st July 2019}
\maketitle

\noindent To whom it may concern,\bigskip

\noindent This is a resubmission of our paper, originally entitled ``Quasimap quantum Lefschetz via relative quasimaps'', which we first submitted to IMRN on 18th June 2018 (submission ID: IMRN-2018-502).\medskip

With respect to the first version, the first reviewer was broadly positive, while the second reviewer suggested an additional interesting question which could be explored in order to improve the paper. For that reason, the paper was not accepted for publication, but we were told that an improved version which addressed the reviewer's concerns could be considered.\medskip

We are now able to provide an answer to the question posed by the reviewer, and in a way which opens the door for future research. We have therefore prepared a strengthened version of the paper for resubmission. The main points of difference with the first version are:\medskip
\begin{enumerate}
\item We address the reviewer's main question, reproduced here for convenience:\bigskip

\begin{quote} The paper would be strengthened by a clear detailing of where the techniques and proofs diverge from Gathmann's earlier work. If this can be done I recommend the paper for publication in International Mathematics Research Notices.

\qquad There is a natural question which is left unanswered which would significantly strengthen the paper. That is, how do relative invariants relate to relative quasimap invariants. In the case where $X = \mathbb{P}^n$ and $Y$ is a hypersurface, can a formula be obtained relating the relative invariants obtained by integrating over $\overline{\mathcal{M}}(\mathbb{P}^n|Y)$ with those of $\mathcal{Q}(\mathbb{P}^n|Y)$? Such a formula would be interesting and would provide a clear point of divergence from Gathmann's work.\end{quote}\bigskip

\medskip In a new section of the paper (\S 5), we show that the generating function for our relative quasimap invariants coincides with the relative $I$-function, as defined via different means by Fan--Tseng--You in their study of relative mirror symmetry (\url{https://arxiv.org/abs/1811.10807}). This constitutes a ``wall-crossing'' result, in that it relates the relative quasimap invariants to the relative Gromov--Witten invariants.\medskip

As well as answering the main question posed by the reviewer (and in significantly greater generality than hypersurfaces in projective space), this result demonstrates that relative quasimaps provide a natural framework for studying mirror symmetry in the relative setting. This opens up the possibility for future work in this direction. One avenue, which we have already begun to explore, is the development and application of a theory of logarithmic quasimaps; this would constitute a natural sequel to the present paper.\medskip

\item The introduction has been completely rewritten, in order to properly contextualise and motivate the work. The connection with relative mirror symmetry is given particular emphasis (in keeping with this, we have also suggested a modification of the title).\medskip

\item The presentation has been streamlined significantly; the material has been reorganised, and the paper is 5 pages shorter than the previous version.\medskip

\item We have addressed the other, more minor comments, raised by the second reviewer.\medskip
\end{enumerate}

Thank you for taking the time to read this letter. We look forward to hearing back from you.\bigskip

\noindent Yours faithfully,\\
Luca Battistella and Navid Nabijou\\

\end{document}