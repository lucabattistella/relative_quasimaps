\section{Splitting principle}

Consider boundary strata of the space of quasimaps, i.e. where the underlying curve is reducible and has a prescribed profile, by which we mean the dual graph decorated by the degree of the universal line bundle on each component: there are two natural virtual classes on such a stratum, namely the one induced by the splitting type of the curve and the product one. We are going to show that these virtual classes coincide; this works pretty much as in Gromov-Witten theory.

Fix a smooth projective toric variety $X$, and numerical invariants $g,n,\beta$ such that $\Q{g}{n}{X}{\beta}$ is defined. Now fix a partition $A=(A_0,\ldots,A_r)$ of the genus and marked points and a partition $B=(\beta_0, \ldots, \beta_r)$ of the curve class such that every factor in the following product makes sense, and then consider the space (which we call the \ildef{centipede locus}):
\begin{equation*} \D{X}{A}{B} := \Q{g_0}{A_0 \cup \{ q_1, \ldots, q_r \}}{X}{\beta_0} \times_{X^r} \prod_{i=1}^r \Q{g_i}{A_i\cup\{q_i\}}{X}{\beta_i} \end{equation*}
We can equip the centipede locus with the product virtual class in the following way. Set
\begin{equation*} \E{X}{A}{B} :=  \Q{g_0}{A_0 \cup \{ q_1, \ldots, q_r \}}{X}{\beta_0} \times \prod_{i=1}^r \Q{g_i}{A_i\cup\{q_i\}}{X}{\beta_i} \end{equation*}
which we endow with the product class:
\begin{equation*} \virt{\E{X}{A}{B}} := \virt{\Q{g_0}{A_0 \cup \{ q_1, \ldots, q_r \}}{X}{\beta_0}} \times \prod_{i=1}^r \virt{\Q{g_i}{A_i\cup\{q_i\}}{X}{\beta_i}} \end{equation*}
We then consider the cartesian diagram
\begin{equation} \label{Product diagram}
\begin{tikzcd}
\D{X}{A}{B} \ar[r,"h"] \ar[d,"\ev_q"] \ar[rd,phantom,"\square"] & \E{X}{A}{B} \ar[d,"\ev_q"] \\
X^r \ar[r,"\Delta_{X^r}"] & X^r \times X^r
\end{tikzcd}
\end{equation}
and, since $X$ is smooth so $\Delta_{X^r}$ is a regular embedding, define:
\begin{equation*} \virt{\D{X}{A}{B}} := \Delta_{X^r}^! (\virt{\E{X}{A}{B}}) \end{equation*}
Notice that, by defining
\begin{equation*} \MM_{A,B}^{\operatorname{wt}} := \MM_{g_0,A_0\cup\{q_1,\ldots,q_r\},\beta_0}^{\operatorname{wt}} \times \prod_{i=1}^r \MM_{g_i,A_i\cup\{q_i\},\beta_i}^{\operatorname{wt}} \end{equation*}
there is a triangle
\begin{equation}\label{D-E-triangle}
 \begin{tikzcd}
  \D{X}{A}{B} \ar[dr,"\rho_D"]\ar[rr,"h"] & & \E{X}{A}{B}\ar[dl,"\rho_E"] \\
& \MM_{A,B}^{\operatorname{wt}} &
 \end{tikzcd}
\end{equation}
and the product virtual class on $\E{X}{A}{B}$ corresponds to the product of the standard obstruction theories for each factor $\Q{g_i}{A_i\cup\{q_i\}}{X}{\beta_i}\to \MM_{A_i,B_i}^{\operatorname{wt}}$ (the latter is \'etale over the usual moduli space of prestable curves by forgetting the weight, hence they have isomorphic cotangent complexes).

On the other hand, we have the following cartesian diagram
\begin{equation} \label{Full quasimap diagram}
\begin{tikzcd}
\D{X}{A}{B} \ar[r,"\varphi"] \ar[d,"\rho_D"] \ar[rd,phantom,"\square"] & \Q{0}{n}{X}{\beta} \ar[d,"\rho_Q"] \\
\MM_{A,B}^{\operatorname{wt}} \ar[r,"\psi"] & \MM_{g,n,\beta}^{\operatorname{wt}}
\end{tikzcd}
\end{equation}

\begin{remark}
The bottom horizontal map is not a closed immersion: due to the existence of degree--$0$ rational components, there may be many possible equally valid ways of breaking up a nodal curve. For instance, consider the following example of two elements which map to the same curve under $\psi$.
[FIGURE]

Yet $\psi$ has a natural perfect obstruction theory, given by $L_{\psi}$: we only need to show that it is supported in $[-1,0]$. Consider the exact triangle:
\begin{equation*} \psi^* L_{\MM_{g,n,\beta}^{\operatorname{wt}}} \to L_{\MM_{A,B}^{\operatorname{wt}}} \to L_\psi \xrightarrow{[1]} \end{equation*}
The first two terms are concentrated in degrees $[0,1]$, because they are the cotangent complexes of smooth Artin stacks. Therefore $L_\psi$ is concentrated in degrees $[-1,1]$. Furthermore, if we examine the long exact cohomology sequence near $\h^1(L_\psi)$ we find
\begin{equation*} \h^1(\psi^* L_{\MM_{g,n,\beta}^{\operatorname{wt}}}) \to \h^1(L_{\MM_{A,B}^{\operatorname{wt}}}) \to \h^1(L_\psi) \to 0 \end{equation*}
and hence we must show that the first map is surjective. But this is dual to the map which takes an infinitesimal automorphism of the disconnected curve to an infinitesimal automorphism of the corresponding connected curve (obtained by glueing together the ``nodal'' marked points). The requirement of preserving the markings translates into that of fixing the node after the gluing operation, so the (infinitesimal) automorphism groups coincide. Hence $\h^1(L_\psi) = 0$ as claimed. (This also descends from the fact that the fibres of $\psi$ are Deligne--Mumford.)
\end{remark}

\begin{lemma} \label{Lemma product class equals pullback class} $(h^*E_{\E{A}{B}{X}}, \phi^*E_{\rho_Q}, \ev_q^*E_{\Delta_{X^r}})$ is a compatible triple for the triangle \ref{D-E-triangle}, hence
\[
 \psi^![\Q{g}{n}{X}{\beta}]=\Delta_{X^r}^![\E{A}{B}{X}]\in A_*(\D{A}{B}{X}).
\]
\end{lemma}
\begin{proof}
 We need to construct a morphism of triangles
 \bcd
 h^*E_{\E{A}{B}{X}}\ar[r]\ar[d] & \phi^*E_{\rho_Q}\ar[r]\ar[d] & \ev_q^*E_{\Delta_{X^r}}\ar[r,"{[1]}"]\ar[d] & {} \\
 h^* L_{\rho_E}\ar[r] & L_{\rho_D}\ar[r] & L_h\ar[r,"{[1]}"] & {}
 \ecd
 
 Consider the following diagram:
\bcd
h^* \tilde{\mathcal{C}} \ar[r,"\nu"] \ar[rd,"\eta" below] & \varphi^* \mathcal{C} \ar[r] \ar[d] \ar[rd,phantom,"\square"] & \mathcal{C} \ar[d,"\pi"] \\
& \D{X}{A}{B} \ar[r,"\varphi"] & \Q{0}{n}{X}{\beta}
\ecd
Here $\tilde{C}$ is the universal (disconnected) curve over $\E{X}{A}{B}$, which we have pulled back to $\D{X}{A}{B}$, while $\varphi^* \mathcal{C}$ is the universal curve over $\D{X}{A}{B}$. Therefore the map $\nu : h^* \tilde{\mathcal{C}} \to \varphi^* \mathcal{C}$ is (fiberwise) a partial normalisation map given by normalising the nodes which connect the ``trunk'' of the centipede to the ``legs.''

There are natural sheaves $\mathcal{F}$ and $\tilde{\mathcal{F}}$ on $\mathcal{C}$ and $h^* \tilde{\mathcal{C}}$ respectively, such that
\begin{align*} \varphi^*E_{\rho_Q}^\vee & = \R \pi_* \mathcal{F} \\
h^* E_{\rho_E}^\vee & = \R \eta_* \tilde{\mathcal{F}} \end{align*}
Furthermore, $\nu^*\mathcal{F}=\tilde{\mathcal{F}}$, hence by tensoring the partial normalisation short exact sequence
\begin{equation*} 0 \to \OO_{\varphi^*{\mathcal{C}}} \to \nu_* \OO_{h^* \tilde{\mathcal{C}}} \to \OO_q \to 0 \end{equation*}
with $\mathcal{F}$ and applying the projection formula, we obtain
\begin{equation*} 0 \to \mathcal{F} \to \nu_* \tilde{\mathcal{F}} \to \mathcal{F}_q \to 0 \end{equation*}
on $\varphi^*\mathcal{C}$, where $q$ is the locus of nodes connecting the trunk to the spine. (The fact that the morphism on the left is injective follows by applying the Snake Lemma to the short exact sequence defining $\mathcal{F}$.) To this we can apply $\R \pi_*$ to obtain an exact triangle
\begin{equation} \label{Normalisation exact triangle} \R \pi_* \mathcal{F} \to \R \eta_* \tilde{\mathcal{F}} \to \R \pi_* \mathcal{F}_q \xrightarrow{[1]} \end{equation}
Finally, notice that, since quasimaps are required not to have base-points at the nodes, the fibre of the sheaf $\mathcal F$ at each of the nodes $q$ can actually be identified with the tangent to the toric variety $X$ at the image of the node itself, i.e. $\R \pi_* \mathcal{F}_q\simeq \ev_q^* T_{X^r}=T_{\Delta_{X^r}}[-1]$.

The statement now follows from functoriality of virtual pull-backs.
\end{proof}
