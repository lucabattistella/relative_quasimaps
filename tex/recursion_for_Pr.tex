\section{Recursion formula for $\PP^N$ relative $H$} \label{Section recursion for PN}

In this section we will show that the moduli space
\begin{equation*} \Q{0}{\alpha}{\PP^N}{d} \end{equation*}
is irreducible of the expected dimension, and thus admits a fundamental class. We then prove a recursion formula for these fundamental classes by pushing forward Gathmann's recursion formula along the comparison morphism $\chi$.

We set $X=\PP^N$ and $Y=H=\{ z_0 = 0 \}$. Then given a quasimap
\begin{equation*} (C,x_1,\ldots,x_n,L,u_0,\ldots,u_N) \in \Q{0}{n}{\PP^N}{d} \end{equation*}
the line bundle $L_Y$ of the previous section is equal to $L$ and the section $u_Y$ is equal to $u_0$. Let us denote by
\begin{equation*} \mathcal Q_{0,\alpha}(\PP^N|H,d) \subseteq \Q{0}{n}{\PP^N}{d} \end{equation*}
the \ilemph{nice locus}, consisting of those quasimaps with irreducible source curve (i.e. a $\PP^1$), no basepoints (so we get an actual map), with no component of the curve mapping inside $H$ and with the map having tangency at least $\alpha_i$ to $H$ at the marking $x_i$.

Notice that this is an irreducible, locally closed substack of $\Q{0}{n}{\PP^N}{d}$ of codimension $\Sigma_i \alpha_i$, by essentially the same argument as in \cite[Lemma 1.8]{Ga}. In fact it is isomorphic to the nice locus inside the stable map space, denoted $\mathcal M_{0,\alpha}(\PP^N|H,d)$ by Gathmann (see \cite[Def. 1.6]{Ga}; the stricter stability condition has no effect when the source curve is irreducible, provided of course that $n\geq2$). We thus obtain:

\begin{lem}\label{lem:comparison}
The comparison morphism restricts to a morphism 
\begin{equation*} \chi: \M{0}{\alpha}{\PP^N|H}{d}\to \Q{0}{\alpha}{\PP^N|H}{d} \end{equation*}
which is birational.
\end{lem}
\begin{proof}
We need to verify that a relative stable map is sent to a relative stable quasimap by $\chi$. Since the contraction of a rational tail $R$ always occurs away from the markings, we only need to examine the internal components $Z$ of the quasimap.

Consider then $Z$; for each basepoint $x$ on $Z$ there is a rational tail $R$ of the stable map attached to $Z$ at $x$. This is either internal (mapped into $H$) or external (not mapped into $H$).

If $R$ is internal then both $R$ and $Z$ live inside the same connected component $Z^\prime$ of $f^{-1}(H)$. Applying $\chi$ has the effect of contracting $R$ and adding a line bundle to $Z$ of degree equal to $H \cdot f_* [R]$. Thus the left hand side of the inequality \eqref{Relative quasimap internal component inequality} is left unchanged, and since the right hand side is also unaltered the inequality is satisfied.

On the other hand if $R$ is external then the multiplicity $m^{(R)}$ of $R \cap Z$ satisfies:
\begin{equation*} m^{(R)} \leq H \cdot f_* [R] \end{equation*}
Since applying $\chi$ has the effect of replacing $m^{(R)}$ by $H \cdot f_* [R]$ in the left hand side of \eqref{Relative quasimap internal component inequality}, the inequality still holds for the quasimap. Thus we obtain a morphism from the relative stable map space to the relative quasimap space, as claimed.

The birationality statement follows from the fact that the comparison morphism restricts to give an isomorphism between the nice loci.
\end{proof}

\begin{lem}
With notations as above (with $\sum\alpha\leq d$), $\Qt{0}{\alpha}{\PP^r|H}{d}$ is the closure of the nice locus $\mathcal Q_{0,\alpha}(\PP^r|H,d)$ inside $\Q{0}{n}{\PP^r}{d}$. 
\end{lem}
\begin{proof}
$\Qt{0}{\alpha}{\PP^r|H}{d}\subseteq\overline{\mathcal Q_{0,\alpha}(\PP^r|H,d)}$: we show that, given any quasimap satisfying the $\alpha$-tangency conditions spelled above, it can be (infinitesimally) deformed to a stable \emph{map} satisfying Gathmann's conditions \cite[Def. 1.1 and Rmk. 1.4]{Ga}, and then appeal to \cite[Prop. 1.14]{Ga}.

We induct on the number of components containing at least one base-point. If this number is zero, we're done (because quasimap stability is stronger than map stability); otherwise, pick such a component $C_0$, with base-points $p_1,\ldots,p_h$ and adjacent rational trees $R_1,\ldots,R_k$, joined to $C_0$ at the nodes $q_1,\ldots,q_k$. Since there are base-points but the quasimap respects the nondegeneracy condition, $\deg(L_{|C_0})>0$, and since $C_0\simeq\PP^1$ we can find a section $w$ of $L_{|C_0}\simeq\mathcal O_{\PP^1}(d_0)$ not vanishing at any of the base-points $p_i$'s; then it is enough to deform the section $u_{r|C_0}$ to $u_{r|C_0}+\epsilon w$ (and keep the other sections the same) in order to delete the base-points belonging to $C_0$. Notice that $u_{0|C_0}$ is unchanged, so the deformation still respects $\alpha$-tangency at the markings lying on $C_0$ (whether the latter is an internal or an external component). We need to check that such a deformation can be extended to the whole curve $C$ without changing the vanishing conditions on $u_0$. Notice that the action of $PGL_{r+1}$ on $\PP^r$ extends to an action of the group on the space of quasimaps; we can apply the matrix
\[
\begin{bmatrix}
1 & & \\
 & \ddots & \\
 & \epsilon \frac{w(q_i)}{u_j(q_i)} & 1
\end{bmatrix}
\]
to the restriction of the original quasimap to $R_i$, where $j$ is any index s.t. $u_j(q_i)\neq 0$ (one such must exist because the node is not allowed to be a base-point), and by doing this separately to every rational tree springing from $C_0$ we get a deformation of the original quasimap that still has $\alpha$-tangency with the hyperplane $H$ ($u_0$ hasn't been touched at all), but the base-points on $C_0$ have been eliminated.

$\overline{\mathcal Q_{0,\alpha}(\PP^r|H,d)}\subseteq\Qt{0}{\alpha}{\PP^r|H}{d}$: consider a family of relative quasimaps over a smooth curve $S$, such that the generic fiber lies in the nice locus. Then we may blow-up the source curve (which is a fibered surface) in the base-points of the quasimap (that are finitely many smooth points of the central fiber) in order to get an actual morphism to $\mathbb P^r$; we may as well suppose that the central fiber of the new family is stable. Notice that the central fiber actually belongs to Gathmann's space $\M{0}{\alpha}{\PP^r|H}{d}$: we have just introduced some rational tails away from the markings, hence the only thing we have to check is, when we blow-up a base-point on an internal component, the rational tail will again be internal ($u_0\equiv 0$ in a neighborhood of the base-point), so it will contribute to the LHS of the $\alpha$-tangency condition nr. 2 in the very same way. We may now invoke \cite[Lemma 1.9]{Ga} and the quasimap case follows from Lemma \ref{lem:comparison}.
\end{proof}
From now on we shall denote this closed substack by $\Q{0}{\alpha}{\PP^r|H}{d}$.

\medskip

Increasing the multiplicity can be naively performed in the very same way as Gathmann did:
\[
\sigma^m_k:=x_k^*d^m_{\mathcal C/\overline{\mathcal Q}}(u_0)\in H^0(\overline{\mathcal Q},x_k^*\mathcal P^m_{\mathcal C/\overline{\mathcal Q}}(\mathcal L))
\]
with $m=\alpha_k+1$ cuts $\Q{0}{\alpha+e_k}{\PP^r|H}{d}$ inside $\Q{0}{\alpha}{\PP^r|H}{d}$, together with a bunch of degenerate contributions from quasimaps where the component on which $x_k$ lies is internal (call it $Z$) and (notice the equality sign!)
\[
\deg(L_{|Z})+\sum m^{(i)}=\sum_{x_j\in Z}\alpha_j.
\]
Of course, quasimap stability forces these degenerate contributions not to have any rational tail; this is really the only difference with the case of stable maps, and indeed we can pushforward Gathmann's formula along the comparison morphism $\comp\colon \M{0}{n}{\PP^r}{d}\to\Q{0}{n}{\PP^r}{d}$ and the only terms that are going to change are the degenerate ones with rational tails (in fact they disappear, since the restriction of the comparison map has positive dimensional fibers there). With an eye to the future, we remark that these contributions do matter when computing GW invariants of a CY hypersurface in projective space, and may well account for the divergence between GW and quasimap invariants in the CY case \cite[Rmk. 1.6]{Ga-MF}.

\begin{lem}\label{lem:compare_psi}
$\comp^*(\psi_k)=\psi_k$ and $\comp^*(x_k^*\mathcal L)=\ev_k^*(\mathcal O_{\mathbb P^r}(H))$.
\end{lem}
\begin{proof}
Recall that $\psi_k=c_1(x_k^*\omega_{\mathcal C/\mathcal M})$ and contemplate the following diagram
\bcd
& & & \PP^r & \\
\mathcal C_{\overline {\mathcal M}}\ar[rr,"\sst"]\ar[rd] \ar[urrr,"f"] & & \comp^*\mathcal C_{\overline {\mathcal Q}} \ar[ld]\ar[rr]\ar[ur,dashed] & & \mathcal C_{\overline {\mathcal Q}} \ar[d]\ar[ul,dashed] \\
& \M{0}{n}{\PP^r}{d} \ar[rrr,"\comp"] \ar[ul,bend left,"x_k"]\ar[ur,bend right,"x_k"right=.2cm]& & & \Q{0}{n}{\PP^r}{d}\ar[u,bend right, "x_k"right]
\ecd
where $\sst$ is the strong stabilisation map, i.e. contracting the rational tails, which is an isomorphism near the markings.
\end{proof}

\begin{lem}\label{lem:posdimfiber}
$\dim(\M{0}{(m^{(i)})}{\PP^r|H}{d}\cap \ev_1^*(p))>0$ everytime $rd>1$, where $p$ is a point of $H$, so the pushforward along $\comp$ of a degenerate locus with rational tails is 0.
\end{lem}
\begin{proof}
$\dim(\M{0}{(m^{(i)})}{\PP^r|H}{d}\cap \ev_1^*(p))=(r-3)+(1-m^{(i)})+d(r+1)-(r-1)=(rd-1)+(d-m^{(i)})$.
\end{proof}

\begin{prop}
Denote by $[D^\mathcal{Q}_{\alpha,k}(\PP^r|H,d)]$ the sum of the (product) fundamental classes of
\[
\Q{0}{\alpha^{(0)}\cup {(0,\ldots,0)}}{H}{d_0}\times_{(\PP^r)^k}\prod_{i=1}^k \Q{0}{(m^{(i)})\cup\alpha^{(i)}}{\PP^r|H}{d_i}
\]
with coefficient $\frac{m^{(1)}\ldots m^{(k)}}{k!}$, where the sum runs over all splittings $d=\sum d_i$ and $\alpha=\bigcup \alpha^{(i)}$ such that the above spaces are well-defined, in particular $|\alpha^{(0)}|+k$ and $|\alpha^{(i)}|+1$ are all $\geq 2$, and such that
\[
d_0+\sum_{i=1}^k m^{(i)}=\sum \alpha^{(0)}
\]

The following formula holds
\[
(\alpha_k\psi_k+x_k^*\mathcal L)\cdot[\Q{0}{\alpha}{\PP^r|H}{d}]=[\Q{0}{\alpha+e_k}{\PP^r|H}{d}]+[D^\mathcal{Q}_{\alpha,k}(\PP^r|H,d)].
\]
\end{prop}
\begin{proof}
Follows from \cite[Thm. 2.6]{Ga} by pushforward along $\comp\colon \M{0}{n}{\PP^r}{d}\to\Q{0}{n}{\PP^r}{d}$, using the projection fomula and Lemmas \ref{lem:comparison}, \ref{lem:compare_psi} and \ref{lem:posdimfiber}.
\end{proof}