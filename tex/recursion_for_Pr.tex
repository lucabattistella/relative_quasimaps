\section{Recursion formula for $\PP^N$ relative $H$} \label{Section recursion for PN}

\subsection{Basic properties of the moduli space} \label{Subsection basic properties of the moduli space}
In this section we will show that the moduli space
\begin{equation*} \Q{0}{\alpha}{\PP^N|H}{d} \end{equation*}
is irreducible of the expected dimension, and thus admits a fundamental class. We then prove a recursion formula for these fundamental classes by pushing forward Gathmann's recursion formula along the comparison morphism $\chi$.

We set $X=\PP^N$ and $Y=H=\{ z_0 = 0 \}$. Given a quasimap
\begin{equation*} (C,x_1,\ldots,x_n,L,u_0,\ldots,u_N) \in \Q{0}{n}{\PP^N}{d} \end{equation*}
the line bundle $L_Y$ of the previous section is equal to $L$ and the section $u_Y$ is equal to $u_0$. Let us denote by
\begin{equation*} \mathcal Q_{0,\alpha}(\PP^N|H,d) \subseteq \Q{0}{n}{\PP^N}{d} \end{equation*}
(without the bar) the \ilemph{nice locus}, consisting of those quasimaps with irreducible source curve (i.e. a $\PP^1$), no basepoints (so we get an actual map), with no component of the curve mapping inside $H$ and with the map having tangency at least $\alpha_i$ to $H$ at the marking $x_i$.

This is an irreducible, locally closed substack of $\Q{0}{n}{\PP^N}{d}$ of codimension $\Sigma_i \alpha_i$, by essentially the same argument as in \cite[Lemma 1.8]{Ga}. In fact it is isomorphic to the nice locus inside the stable map space, denoted $\mathcal M_{0,\alpha}(\PP^N|H,d)$ by Gathmann (see \cite[Def. 1.6]{Ga}; the stricter stability condition has no effect when the source curve is irreducible, provided of course that $n\geq2$). We thus obtain:

\begin{lem}\label{lem:comparison}
The comparison morphism restricts to a morphism 
\begin{equation*} \chi: \M{0}{\alpha}{\PP^N|H}{d}\to \Q{0}{\alpha}{\PP^N|H}{d} \end{equation*}
\end{lem}
\begin{proof}
We need to verify that a relative stable map is sent to a relative stable quasimap by $\chi$. Since the contraction of a rational tail $R$ always occurs away from the markings, we only need to examine the internal components $Z$ of the quasimap.

Consider then $Z$; for each basepoint $x$ on $Z$ there is a rational tail $R$ of the stable map attached to $Z$ at $x$. This is either internal (mapped into $H$) or external (not mapped into $H$).

If $R$ is internal then both $R$ and $Z$ live inside the same connected component $Z^\prime$ of $f^{-1}(H)$. Applying $\chi$ has the effect of contracting $R$ and adding a line bundle to $Z$ of degree equal to $H \cdot f_* [R]$. Thus the left hand side of the inequality \eqref{Relative quasimap internal component inequality} is left unchanged, and since the right hand side is also unaltered the inequality is satisfied.

On the other hand if $R$ is external then the multiplicity $m^{(R)}$ of $R \cap Z$ satisfies:
\begin{equation*} m^{(R)} \leq H \cdot f_* [R] \end{equation*}
Since applying $\chi$ has the effect of replacing $m^{(R)}$ by $H \cdot f_* [R]$ in the left hand side of \eqref{Relative quasimap internal component inequality}, the inequality still holds for the quasimap. Thus we obtain a morphism from the relative stable map space to the relative quasimap space, as claimed.
\end{proof}

\begin{lem}
$\Q{0}{\alpha}{\PP^N|H}{d}$ is equal to the closure of the nice locus $\mathcal{Q}_{0,\alpha}(\PP^N|H,d)$ inside $\Q{0}{n}{\PP^N}{d}$. 
\end{lem}
\begin{proof}
$\Q{0}{\alpha}{\PP^N|H}{d}\subseteq\overline{\mathcal Q_{0,\alpha}(\PP^N|H,d)}$: we show that any relative stable quasimap can be infinitesimally deformed to a relative stable quasimap with no basepoints. This is in particular a relative stable map; we then appeal to \cite[Prop. 1.14]{Ga} to deform this stable map and obtain a point in the nice locus. Since this deformation does not introduce any rational tails, this is also a deformation of quasimaps, and the statement follows.

We induct on the number of components containing at least one base-point. Suppose this number is non-zero (otherwise there is nothing to prove) and pick such a component $C_0$, with base-points $y_1 ,\ldots, y_k$. Recall that this means that $u_i(y_j)=0$ for all $i$ and $j$. We will deform the section $u_N|_{C_0}$ to a new section $u_N^\prime|_{C_0}$ in such a way that $u_N^\prime|_{C_0}(y_j) \neq 0$ and in such a way that we do not introduce any new basepoints. Notice that since the relative condition only depends on $u_0$, the resulting deformed quasimap will still be a relative quasimap.

Now, by the nondegeneracy condition we must have $\deg(L|_{C_0})>0$, and since $C_0 \cong \PP^1$ we can find a section $w_0$ of $L|_{C_0} \cong \OO_{\PP^1}(d_0)$ not vanishing at any of the base-points $p_i$.

We then set
\begin{equation*} u_N^\prime|_{C_0} := u_N|_{C_0} + \epsilon w_0 \end{equation*}
and $u_i^\prime|_{C_0} = u_i|_{C_0}$ for all other $i$. Notice that $u_N^\prime|_{C_0}(y_j) \neq 0$ for all $j$ as claimed. It is also clear that we do not introduce any new basepoints, since $u_N^\prime|_{C_0}(y) = 0$ implies $u_N|_{C_0}(y)=0$ (put differently: being a basepoint is a close condition).

It remains to extend the section $u_N^\prime|_{C_0}$ to a section $u_N^\prime$ on the whole curve. Let $C_1, \ldots, C_r$ be the components of $C$ adjacent to $C_0$ and let $q_i = C_0 \cap C_i$. We need to modify the sections $u_N|_{C_i}$ in such a way that $u_N^\prime|_{C_i}(q_i) = u_N^\prime|_{C_0}(q_i)$.

By nondegeracy, we can choose a section $w_i$ of $L|_{C_i}$ such that $w(q_i) \neq 0$. Then set:
\begin{equation*} {u_N^\prime}_{C_i} := {u_N}|_{C_i} + \epsilon \left( \frac{w_0(q_i)}{w_i(q_i)} \right) \cdot w_i \end{equation*}
Then indeed we have:
\begin{equation*} {u_N^\prime}|_{C_i}(q_i) = u_N(q_i) + \epsilon \left( \frac{w_0(q_i)}{w_i(q_i)} \right) \cdot w_i(q_i) = u_N(q_i) + \epsilon w_0(q_i) = {u_N^\prime}|_{C_0}(q_i) \end{equation*}
We can continue this process, replacing $C_0$ by $C_i$; since the genus of the curve is zero, we will never come to the same curve twice. In this way we obtain a new quasimap
\begin{equation*} (C,x_1,\ldots,x_n,L,u_0, \ldots, u_{N-1}, u_N^\prime) \end{equation*}
over $\Spec \kfield[\epsilon]/(\epsilon^2)$ which has no basepoints on $C_0$. We can repeat this process for all the components of $C$ (using higher powers of $\epsilon$ each time in order to ensure that we never introduce additional basepoints) and thus we obtain an infinitesimal deformation of our original quasimap which has no basepoints, as required.

$\overline{\mathcal Q_{0,\alpha}(\PP^N|H,d)}\subseteq\Q{0}{\alpha}{\PP^N|H}{d}$: consider a family of stable quasimaps over a smooth curve $S$, such that the generic fibre lies in the nice locus. We may blow-up the source curve (a fibered surface over $S$) in the locus of basepoints (which consists of finitely many smooth points of the central fiber) in order to obtain an actual morphism to $\PP^N$. This has the effect of adding rational tails at the basepoints in the central fibre. If the morphism is constant on any of these rational tails we may contract them, and thus we obtain a family of stable maps which pushes down along $\chi$ to our original family of quasimaps.

The general fibre is not modified at all, and so is still in the nice locus. By \cite[Lemma 1.9]{Ga} it follows that the central fibre is a relative stable map, and then by applying $\chi$ and appealing to Lemma \ref{lem:comparison} it follows that the same is true for the central fibre of the family of quasimaps.
\end{proof}

\begin{cor} The moduli space $\Q{0}{\alpha}{\PP^N|H}{d}$ is irreducible of the expected dimension. Hence it admits a fundamental class. \end{cor}
\begin{proof} This holds because the moduli space is equal to the closure of the nice locus, which is irreducible of the expected dimension. \end{proof}

\begin{cor} \label{Comparison morphism birational} The comparison morphism is a birational map from the moduli space of relative stable maps to the moduli space of stable maps. In particular it sends the fundamental class to the fundamental class. \end{cor}
\begin{proof} This follows because the comparison morphism restricts to an isomorphism on the nice locus, which by the lemma above is a dense open subset of both spaces. \end{proof}


\subsection{Proof of the recursion formula} \label{Subsection recursion formula for PN}
We wish to obtain a recursion formula relating the quasimap invariants of multiplicity $\alpha$ with the quasimap invariants of multiplicity $\alpha + e_k$, as in \cite[Theorem 2.6]{Ga}. This process of ``increasing the multiplicities'' can be naively performed in the same way as Gathmann: for $m = \alpha_k + 1$ the following section (of the pull-back of the jet bundle of the universal line bundle)
\[
\sigma^m_k:=x_k^*d^m_{\mathcal C/\overline{\mathcal Q}}(u_0)\in \HH^0(\overline{\mathcal Q},x_k^*\mathcal P^m_{\mathcal C/\overline{\mathcal Q}}(\mathcal L))
\]
cuts out $\Q{0}{\alpha+e_k}{\PP^N|H}{d}$ inside $\Q{0}{\alpha}{\PP^N|H}{d}$, along with a number of degenerate contributions (called the \ilemph{comb loci}) parametrising quasimaps for which $x_k$ belongs to an internal component $Z \subseteq C$ (a component on which $u_0$ vanishes), such that
\begin{equation*}\deg(L|_{Z})+\sum_{i=1}^r m^{(i)}=\sum_{x_i\in Z}\alpha_i \end{equation*}
(here by ``component'' we really mean ``connected component of the vanishing locus of $u_0$''). Quasimap stability means that these degenerate contributions cannot contain any rational tails; this is really the only difference with the case of stable maps.

Indeed, we can actually push forward Gathmann's formula along the comparison morphism
\begin{equation*} \comp \colon \M{0}{\alpha}{\PP^N|H}{d}\to\Q{0}{\alpha}{\PP^N|H}{d} \end{equation*}
and due to Corollary \ref{Comparison morphism birational} above, the only terms which change are the comb loci containing rational tails. In fact these disappear, since the restriction of the comparison map to these loci has positive-dimensional fibres:
\begin{lem}\label{lem:posdimfiber} Consider a rational tail component in the comb locus of the moduli space of stable maps, i.e. a moduli space of the form:
\begin{equation*} \M{0}{(m^{(i)})}{\PP^N|H}{d} \end{equation*}
Then (assuming that $Nd > 1$) we have
\begin{equation*} \dim \left( [\M{0}{(m^{(i)})}{\PP^N|H}{d}] \cap \ev_1^*(\pt_H) \right) > 0 \end{equation*}
where $\pt_H \in \Achow^{N-1}(H)$ is a point class. Thus the pushforward along $\comp$ of any comb locus with a rational tail is $0$.
\end{lem}
\begin{proof} This is a simple dimension count. We have
\begin{align*} \dim \left( [\M{0}{(m^{(i)})}{\PP^N|H}{d}] \cap \ev_1^*(\pt_H) \right) & =(N-3)+d(N+1)+(1-m^{(i)})-(N-1) \\
& =(Nd-1)+(d-m^{(i)})
\end{align*}
from which the lemma follows because $m^{(i)} \leq d$.
\end{proof}
\begin{remark} With an eye to the future, we remark that these rational tail components contribute nontrivially to the Gromov--Witten invariants of a Calabi--Yau hypersurface in projective space, and so their disappearance in our recursion formula may account for the divergence between Gromov--Witten and quasimap invariants in the Calabi--Yau case \cite[Rmk. 1.6]{Ga-MF}. \end{remark}

Since we wish to apply the projection formula to Gathmann's recursion relation, we should express the cohomological terms which appears as pull-backs:
\begin{lem}\label{lem:compare_psi} We have:
\begin{align*} \comp^*(\psi_k) & =\psi_k \\
\comp^*(\ev_k^* H)& =\ev_k^* H
\end{align*}
\end{lem}
\begin{proof}
We will actually show that:
\begin{align*} \chi^* x_k^* \omega_{\mathcal{C}/\om{Q}} & = x_k^* \omega_{\mathcal{C}/\om{M}} \\
\chi^* x_k^* \mathcal{L} & = \ev_k^* \OO_{\PP^N}(H) \end{align*}
This follows by considering the following diagram:
\bcd
& & \PP^N & \\
\mathcal C_{\overline {\mathcal M}} \ar[r,"\sst" below] \ar[rd] \ar[urr, bend left=10, "f"] & \comp^*\mathcal C_{\overline {\mathcal Q}} \ar[d] \ar[rr]\ar[ur,bend right=10,dashed] \ar[rrd,phantom,"\square" right] & & \mathcal C_{\overline {\mathcal Q}} \ar[d]\ar[ul,bend right=10,dashed] \\
& \M{0}{\alpha}{\PP^N|H}{d} \ar[rr,"\comp"] \ar[ul,bend left,"x_k"] \ar[u,bend right,"x_k"right=.2cm] & & \Q{0}{\alpha}{\PP^N|H}{d}\ar[u,bend right, "x_k"right]
\ecd
where $\sst$ is the strong stabilisation map which contracts the rational tails, and so is an isomorphism near the markings.
\end{proof}
\begin{prop}
Define the \ilemph{quasimap comb locus} $\mathcal{D}^\mathcal{Q}_{\alpha,k}(\PP^N|H,d)$ as the union of the moduli spaces
\begin{equation*}
\mathcal{D}^{\mathcal{Q}}(\PP^N|H,A,B,M) = \Q{0}{|\alpha^{(0)}| + r}{H}{d_0}\times_{H^r}\prod_{i=1}^r \Q{0}{(m^{(i)})\cup\alpha^{(i)}}{\PP^N|H}{d_i} \end{equation*}
where the union runs over all splittings $A = (\alpha^{(0)}, \ldots, \alpha^{(r)})$ of the markings, $B = (d_0, \ldots, d_r)$ of the degree and all valid multiplicites $M = (m^{(1)}, \ldots, m^{(r)})$ such that the above spaces are all well-defined (in particular $|\alpha^{(0)}|+k$ and $|\alpha^{(i)}|+1$ are all $\geq 2$) and such that
\[
d_0+\sum_{i=1}^r m^{(i)}=\sum \alpha^{(0)}
\]
Equip this with the sum of the (product) fundamental classes. Then the following formula holds
\[
(\alpha_k\psi_k+\ev_k^*H)\cdot[\Q{0}{\alpha}{\PP^N|H}{d}]=[\Q{0}{\alpha+e_k}{\PP^N|H}{d}]+[\mathcal{D}^\mathcal{Q}_{\alpha,k}(\PP^N|H,d)].
\]
\end{prop}
\begin{proof}
This follows from \cite[Thm. 2.6]{Ga} by pushforward along $\chi$, using the projection fomula and Lemmas \label{Comparison morphism birational}, \ref{lem:posdimfiber} and \ref{lem:compare_psi} .
\end{proof}