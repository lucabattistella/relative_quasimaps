\section{Recursion formula for $\PP^N$ relative a hyperplane} \label{Section recursion for PN}
At this stage we do not know much about the moduli space of relative quasimaps. In this section we will examine the case $X=\PP^N$ and $Y=H$ a hyperplane in detail.

\subsection{Basic properties of the moduli space} \label{Subsection basic properties of the moduli space}
We now show that the moduli space
\begin{equation*} \Q{0}{\alpha}{\PP^N|H}{d} \end{equation*}
is irreducible of the expected dimension, and thus admits a fundamental class. We then prove a recursion formula for these fundamental classes by pushing forward Gathmann's recursion formula along the comparison morphism:
\begin{equation*}\chi : \M{0}{n}{\PP^N}{d} \to \Q{0}{n}{\PP^N}{d} \end{equation*}
Let us briefly recall what this morphism does. Every stable map defines a quasimap which is stable except for the fact that it may have rational tails. $\chi$ has the effect of contracting these rational tails and introducing a basepoint at the corresponding node, with multiplicity equal to the degree of the rational tail; see Appendix \ref{Section comparison morphism} for more details.

For the remainder of this section we set $X=\PP^N$, denote the projective co-ordinates on $X$ by $[z_0: \cdots: z_N]$ and set $Y=H=\{ z_0 = 0 \}$. Given a quasimap
\begin{equation*} \big((C,x_1,\ldots,x_n), L, u_0,\ldots,u_N\big) \in \Q{0}{n}{\PP^N}{d} \end{equation*}
the line bundle $L_Y$ of the previous section is equal to $L$ and the section $u_Y$ is equal to $u_0$.

\begin{lem}\label{lem:comparison}
The comparison morphism restricts to a morphism 
\begin{equation*} \chi: \M{0}{\alpha}{\PP^N|H}{d}\to \Q{0}{\alpha}{\PP^N|H}{d} \end{equation*}
\end{lem}
\begin{proof}
We need to verify that a relative stable map is sent to a relative stable quasimap by $\chi$. Since the contraction of a rational tail $R$ always occurs away from the markings, we only need to examine the internal components $Z$ of the quasimap. To be more precise, we have to show that the inequality \eqref{Relative quasimap internal component inequality} is satisfied, using the fact that the inequality \eqref{Relative stable map internal component inequality} is satisfied by the stable map that we started with.  Let us describe this stable map around $Z$. For each basepoint $x$ on $Z$ there is a rational tail $R$ of the stable map attached to $Z$ at $x$. This is either internal (mapped into $H$) or external (not mapped entirely into $H$).  

If $R$ is internal then both $R$ and $Z$ live inside the same connected component $Z^\prime$ of $f^{-1}(H)$. Applying $\chi$ has the effect of contracting $R$ and increasing the degree of the line bundle on $Z$ by $H \cdot f_* [R]$. Thus the left hand side of inequality \eqref{Relative stable map internal component inequality} is left unchanged, and since the right hand side is also unaltered we obtain inequality \eqref{Relative quasimap internal component inequality}.

On the other hand if $R$ is external then the multiplicity $m^{(R)}$ of $R \cap Z$ satisfies:
\begin{equation*} m^{(R)} \leq H \cdot f_* [R] \end{equation*}
Since applying $\chi$ has the effect of replacing $m^{(R)}$ by $H \cdot f_* [R]$ in the left hand side of \eqref{Relative stable map internal component inequality}, inequality \eqref{Relative quasimap internal component inequality} holds a fortiori for the quasimap. Thus we obtain a morphism from the relative stable map space to the relative quasimap space, as claimed.
\end{proof}
Let us denote by
\begin{equation*} \mathcal Q^{\circ}_{0,\alpha}(\PP^N|H,d) \subseteq \Q{0}{\alpha}{\PP^N|H}{d} \end{equation*}
the \ilemph{nice locus}, consisting of those quasimaps with irreducible source curve $C \cong \PP^1$ and no basepoints (so that we have an actual map $u : C \to \PP^N$) such that the curve is not mapped inside $H$ and $u$ has tangency at least $\alpha_i$ to $H$ at the marking $x_i$.

This is an irreducible, locally closed substack of $\Q{0}{n}{\PP^N}{d}$ of codimension $\Sigma_i \alpha_i$, by essentially the same argument as in \cite[Lemma 1.8]{Ga}. In fact it is isomorphic to the nice locus inside the stable map space, denoted $\mathcal M_{0,\alpha}(\PP^N|H,d)$ by Gathmann (see \cite[Def. 1.6]{Ga}); the stricter stability condition has no effect when the source curve is irreducible.

\begin{lem} \label{lem:closure}
$\Q{0}{\alpha}{\PP^N|H}{d}$ is equal to the closure of the nice locus $\mathcal{Q}^{\circ}_{0,\alpha}(\PP^N|H,d)$ inside $\Q{0}{n}{\PP^N}{d}$. 
\end{lem}
\begin{proof}
$\Q{0}{\alpha}{\PP^N|H}{d}\subseteq\overline{\mathcal{Q}^{\circ}_{0,\alpha}(\PP^N|H,d)}$: we show that any relative stable quasimap can be infinitesimally deformed to a relative stable quasimap with no basepoints. This is in particular a relative stable map; we then appeal to \cite[Prop. 1.14]{Ga} to deform this stable map and obtain a point in the nice locus. Since this deformation does not introduce any rational tails, this is also a deformation of quasimaps, and the statement follows.

We induct on the number of components containing at least one base-point. Suppose this number is non-zero (otherwise there is nothing to prove) and pick such a component $C_0$, with base-points $y_1 ,\ldots, y_k$. Recall that this means that $u_i(y_j)=0$ for all $i$ and $j$. We will deform the section $u_N|_{C_0}$ to a new section $u_N^\prime|_{C_0}$ in such a way that $u_N^\prime|_{C_0}(y_j) \neq 0$ and in such a way that we do not introduce any new basepoints. Notice that since the relative condition only depends on $u_0$, the resulting deformed quasimap will still be a relative quasimap.

Now, by nondegeneracy and the fact that there exists a basepoint, we must have $\deg(L|_{C_0})>0$, and since $C_0 \cong \PP^1$ we can find a section $w_0$ of $L|_{C_0} \cong \OO_{\PP^1}(d_0)$ not vanishing at any of the base-points $y_i$.
We then set
\begin{equation*} u_N^\prime|_{C_0} := u_N|_{C_0} + \epsilon w_0 \end{equation*}
and $u_i^\prime|_{C_0} = u_i|_{C_0}$ for all other $i$. Notice that $u_N^\prime|_{C_0}(y_j) \neq 0$ for all $j$ as claimed. It is also clear that we do not introduce any new basepoints, since $u_N^\prime|_{C_0}(y) = 0$ implies $u_N|_{C_0}(y)=0$ (put differently: being a basepoint is a closed condition).

It remains to extend the section $u_N^\prime|_{C_0}$ to a section $u_N^\prime$ on the whole curve. Let $C_1, \ldots, C_r$ be the components of $C$ adjacent to $C_0$ and let $q_i = C_0 \cap C_i$. We need to modify the sections $u_N|_{C_i}$ in such a way that $u_N^\prime|_{C_i}(q_i) = u_N^\prime|_{C_0}(q_i)$.

By nondegeneracy, we can choose a section $w_i$ of $L|_{C_i}$ such that $w_i(q_i) \neq 0$. Then set:
\begin{equation*} {u_N^\prime}|_{C_i} := {u_N}|_{C_i} + \epsilon \left( \frac{w_0(q_i)}{w_i(q_i)} \right) \cdot w_i \end{equation*}
Then indeed we have:
\begin{equation*} {u_N^\prime}|_{C_i}(q_i) = u_N(q_i) + \epsilon \left( \frac{w_0(q_i)}{w_i(q_i)} \right) \cdot w_i(q_i) = u_N(q_i) + \epsilon w_0(q_i) = {u_N^\prime}|_{C_0}(q_i) \end{equation*}
We can continue this process, replacing $C_0$ by $C_i$; since the genus of the curve is zero there are no cycles in the dual intersection graph, and so we will never come to the same component twice. In this way we obtain a new quasimap
\begin{equation*} \big((C,x_1,\ldots,x_n),L,u_0, \ldots, u_{N-1}, u_N^\prime\big) \end{equation*}
over $\Spec \CC[\epsilon]/(\epsilon^2)$ which has no basepoints on $C_0$. We can repeat this process for all the components of $C$ (using higher powers of $\epsilon$ each time in order to ensure that we never introduce additional basepoints) and thus we obtain an infinitesimal deformation of our original quasimap which has no basepoints, as required.

$\overline{\mathcal{Q}^{\circ}_{0,\alpha}(\PP^N|H,d)}\subseteq\Q{0}{\alpha}{\PP^N|H}{d}$: consider a family of stable quasimaps over a smooth curve $S$, such that the generic fibre lies in the nice locus. We may  blow-up the source curve (a fibered surface over $S$) in the locus of basepoints (which consists of finitely many smooth points of the central fiber) and repeat this process a finite number of times in order to obtain an actual morphism to $\PP^N$. This has the effect of adding rational tails at the basepoints in the central fibre. If the morphism is constant on any of these rational tails we may contract them, and thus we obtain a family of stable maps which pushes down along $\chi$ to our original family of quasimaps.

The general fibre is not modified at all, and so is still in the nice locus. By \cite[Lemma 1.9]{Ga} it follows that the central fibre is a relative stable map, and then by applying $\chi$ and appealing to Lemma \ref{lem:comparison} it follows that the same is true for the central fibre of the family of quasimaps.
\end{proof}

\begin{cor} The moduli space $\Q{0}{\alpha}{\PP^N|H}{d}$ is irreducible of the expected dimension. Hence it has a fundamental class. \end{cor}
\begin{proof} This holds because the moduli space is equal to the closure of the nice locus, which is irreducible of the expected dimension. \end{proof}

Since the moduli space of relative quasimaps has a fundamental class, we can define \emph{relative quasimap invariants} for the pair $(\PP^N,H)$:
\begin{equation*} \left\langle \gamma_1 \psi_1^{k_1} , \ldots, \gamma_n \psi_n^{k_n} \right\rangle_{0,\alpha,d}^{\PP^N|H} := \int_{[\Q{0}{\alpha}{\PP^N|H}{d}]} \prod_{i=1}^n \ev_i^* \gamma_i \cdot \psi_i^{k_i} \end{equation*}
We will now establish a number of properties of the fundamental class.
\begin{cor} \label{Comparison morphism birational} The comparison morphism from relative stable maps to relative quasimaps is birational. In particular it sends the fundamental class to the fundamental class, and thus the invariants coincide. \end{cor}
\begin{proof} This follows because the comparison morphism restricts to an isomorphism on the nice locus, which by Lemma~\ref{lem:closure} is a dense open subset of both spaces. \end{proof}


\subsection{Proof of the recursion formula} \label{Subsection recursion formula for PN}
We wish to obtain a recursion formula relating the quasimap invariants of multiplicity $\alpha$ with the quasimap invariants of multiplicity $\alpha + e_k$, as in \cite[Theorem 2.6]{Ga}. For $m = \alpha_k + 1$ the following section (of the pull-back of the jet bundle of the universal line bundle)
\[
\sigma^m_k := x_k^*d^m_{\mathcal C/\om{Q}}(u_0)\in \HH^0(\om{Q},x_k^*\mathcal P^m_{\mathcal C/\overline{\mathcal Q}}(\mathcal L))
\]
vanishes along $\Q{0}{\alpha+e_k}{\PP^N|H}{d}$ inside $\om{Q} = \Q{0}{\alpha}{\PP^N|H}{d}$, and also along a number of \ilemph{comb loci}.  The latter parametrise quasimaps for which $x_k$ belongs to an internal component $Z \subseteq C$ (a connected component of the vanishing locus of $u_0$), such that:
\begin{equation*}\deg(L|_{Z})+\sum_{i=1}^r m^{(i)}=\sum_{x_i\in Z}\alpha_i \end{equation*}
The strong stability condition means that quasimaps in the comb loci cannot contain any rational tails; this is really the only difference with the case of stable maps.

Indeed, we can push forward Gathmann's recursion formula for stable maps along the comparison morphism
\begin{equation*} \comp \colon \M{0}{\alpha}{\PP^N|H}{d}\to\Q{0}{\alpha}{\PP^N|H}{d} \end{equation*}
and, due to Corollary \ref{Comparison morphism birational} above, the only terms which change are the comb loci containing rational tails. In fact these disappear, since the restriction of the comparison map to these loci has positive-dimensional fibres:
\begin{lem}\label{lem:posdimfiber} Consider a rational tail component in the comb locus of the moduli space of stable maps, i.e. a moduli space of the form:
\begin{equation*} \M{0}{(m^{(i)})}{\PP^N|H}{d} \end{equation*}
and assume that $Nd>1$.  Then
\begin{equation*} \dim \left( [\M{0}{(m^{(i)})}{\PP^N|H}{d}] \cap \ev_1^*(\pt_H) \right) > 0 \end{equation*}
where $\pt_H \in \Achow^{N-1}(H)$ is a point class. Thus the pushforward along $\comp$ of any comb locus with a rational tail is zero.
\end{lem}
\begin{proof} This is a simple dimension count. We have
\begin{align*} \dim \left( [\M{0}{(m^{(i)})}{\PP^N|H}{d}] \cap \ev_1^*(\pt_H) \right) & =(N-3)+d(N+1)+(1-m^{(i)})-(N-1) \\
& =(Nd-1)+(d-m^{(i)})
\end{align*}
from which the lemma follows because $m^{(i)} \leq d$.
\end{proof}
\begin{remark} With an eye to the future, we remark that these rational tail components contribute nontrivially to the Gromov--Witten invariants of a Calabi--Yau hypersurface in projective space, and so their absence from the quasimap recursion formula accounts for the divergence between Gromov--Witten and quasimap invariants in the Calabi--Yau case \cite[Rmk. 1.6]{Ga-MF}. \end{remark}

Since we wish to apply the projection formula to Gathmann's recursion relation, we should express the cohomological terms which appear as pull-backs:
\begin{lem}\label{lem:compare_psi} We have:
\begin{align*} \comp^*(\psi_k) & =\psi_k \\
\comp^*(\ev_k^* H)& =\ev_k^* H
\end{align*}
\end{lem}
\begin{proof}
It suffices to show that:
\begin{align*} \chi^* x_k^* \omega_{\mathcal{C}/\om{Q}} & = x_k^* \omega_{\mathcal{C}/\om{M}} \\
\chi^* x_k^* \mathcal{L} & = \ev_k^* \OO_{\PP^N}(H) \end{align*}
This follows by considering the following diagram:
\bcd
& & \PP^N & \\
\mathcal C_{\overline {\mathcal M}} \ar[r,"\sst" below] \ar[rd] \ar[urr, bend left=10, "f"] & \comp^*\mathcal C_{\overline {\mathcal Q}} \ar[d] \ar[rr]\ar[ur,bend right=10,dashed] \ar[rrd,phantom,"\square" right] & & \mathcal C_{\overline {\mathcal Q}} \ar[d]\ar[ul,bend right=10,dashed] \\
& \M{0}{\alpha}{\PP^N|H}{d} \ar[rr,"\comp"] \ar[ul,bend left,"x_k"] \ar[u,bend right,"x_k"right=.2cm] & & \Q{0}{\alpha}{\PP^N|H}{d}\ar[u,bend right, "x_k"right]
\ecd
where $\sst$ is the strong stabilisation map which contracts the rational tails.  Note that $\sst$ is an isomorphism near the markings.
\end{proof}
\begin{prop} \label{Recursion formula for PN}
Define the \ilemph{quasimap comb locus} $\mathcal{D}^\mathcal{Q}_{\alpha,k}(\PP^N|H,d)$ as the union of the moduli spaces
\begin{equation*}
\mathcal{D}^{\mathcal{Q}}(\PP^N|H,A,B,M) := \Q{0}{A^{(0)} \cup \{q_1,\ldots,q_r\}}{H}{d_0}\times_{H^r}\prod_{i=1}^r \Q{0}{\alpha^{(i)}\cup (m^{(i)})}{\PP^N|H}{d_i}
\end{equation*}
where the union runs over all splittings $A =(A^{(0)},\ldots,A^{(r)})$ of the markings (inducing a splitting $(\alpha^{(0)}, \ldots, \alpha^{(r)})$ of the corresponding tangency conditions), $B = (d_0, \ldots, d_r)$ of the degree and all valid multiplicites $M = (m^{(1)}, \ldots, m^{(r)})$ such that the above spaces are all well-defined (in particular we require that $|A^{(0)}|+r$ and $|A^{(i)}|+1$ are all $\geq 2$) and such that

\[
d_0+\sum_{i=1}^r m^{(i)}=\sum \alpha^{(0)}
\]
Write $[\mathcal{D}^\mathcal{Q}_{\alpha,k}(\PP^N|H,d)]$ for the sum of the (product) fundamental classes, where each term is weighted by:
\begin{equation*}\dfrac{m^{(1)} \cdots m^{(r)}}{r!} \end{equation*}
Then
\[
(\alpha_k\psi_k+\ev_k^*H)\cdot[\Q{0}{\alpha}{\PP^N|H}{d}]=[\Q{0}{\alpha+e_k}{\PP^N|H}{d}]+[\mathcal{D}^\mathcal{Q}_{\alpha,k}(\PP^N|H,d)].
\]
\end{prop}
\begin{proof}
This follows from \cite[Thm. 2.6]{Ga} by pushing forward along $\chi$, using the projection fomula and Lemmas \ref{Comparison morphism birational}, \ref{lem:posdimfiber} and \ref{lem:compare_psi} .
\end{proof}

\begin{remark} \label{Remark on definition of comb locus} In the discussion above we have implicitly used the fact that there exists a commuting diagram of comb loci:
\bcd
\mathcal{D}^{\mathcal{M}}(\PP^N|H,A,B,M) \ar[r] \ar[d] & \M{0}{\alpha}{\PP^N|H}{d} \ar[d] \\
\mathcal{D}^{\mathcal{Q}}(\PP^N|H,A,B,M) \ar[r] & \Q{0}{\alpha}{\PP^N|H}{d}
\ecd
The vertical arrow on the left is a product of comparison morphisms (notice that $H\simeq\PP^{r-1}$). The horizontal arrow at the top is the glueing morphism which glues together the various pieces of the comb to produce a single relative stable map. Here we explain how to define the corresponding glueing morphism for quasimaps, that is, the bottom horizontal arrow.

Suppose for simplicity that $r=1$ and consider an element of the quasimap comb locus, consisting of two quasimaps:
\begin{align*} & \big((C^0,x^0_1,\ldots,x^0_{n_0},q^0),L^0,u^0_0, \ldots, u^0_N\big) \\
&\big((C^1,x^1_1,\ldots,x^1_{n_1},q^1),L^1,u^1_0, \ldots, u^1_N\big) \end{align*}
such that $u^0(q^0) = u^1(q^1)$ in $\PP^N$. We want to glue these quasimaps together at $q^0$,~$q^1$. The definition of the curve is obvious; we simply take:
\begin{equation*} C = C^0 \,_{q^0}\!\sqcup_{q^1} C^1 \end{equation*}
On the other hand, glueing together the line bundles $L^0$ and $L^1$ to obtain a line bundle $L$ over $C$ requires a choice of scalar $\lambda \in \Gm$, in order to match up the fibres over $q^i$. Furthermore if the sections are to extend as well then this scalar must be chosen in such a way that it takes $(u^0_0(q^0), \ldots, u^0_N(q^0)) \in (L^0_{q^0})^{\oplus (N+1)}$ to $(u^1_0(q^1), \ldots, u^1_N(q^1)) \in (L^1_{q^1})^{\oplus (N+1)}$. Since neither $q^0$ nor $q^1$ are basepoints (because they are markings), these tuples are nonzero, and so $\lambda$ is unique if it exists. Furthermore it exists if and only if these tuples belong to the same $\Gm$-orbit in $\Aaff^{N+1}$. This is precisely the statement that $u^0(q^0) = u^1(q^1) \in \PP^N$. 

Similar arguments apply for $r>1$, and for more general toric varieties. \end{remark}