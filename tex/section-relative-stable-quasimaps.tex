\section{Relative stable quasimaps} \label{Section relative stable quasimaps}

\subsection{Review of absolute stable quasimaps} \label{Subsection stable quasimaps}
We briefly recall the definition and basic properties of the moduli space of toric quasimaps; see \cite{CF-K} for more details.
\begin{definition}[{\cite[Definition 3.1.1]{CF-K}}] Let $N$ be a lattice, let $\Sigma \subseteq N_{\QQ}$ be a fan, and let $X= X_{\Sigma}$ be the corresponding toric variety.  Suppose that $X$ is smooth and projective.   Let $M = N^\vee = \Hom(N,\Z)$ and let $\OO_{X_\Sigma}(1)$ be a fixed polarisation, which we can write (non-uniquely) in terms of the torus-invariant divisors as:
\begin{equation*} \OO_{X_\Sigma}(1) = \otimes_{\rho \in \Sigma(1)} \OO_{X_\Sigma}(D_\rho)^{\otimes \alpha_\rho} \end{equation*}
for some $\alpha_\rho \in \Z$. We fix the following numerical invariants: a genus $g \geq 0$, a number of marked points $n \geq 0$, and an effective curve class $\beta \in \HH_2^+(X)$. A \ilemph{stable (toric) quasimap} is given by the data
\begin{equation*} \Big((C,x_1,\ldots,x_n), (L_\rho,u_\rho)_{\rho \in \Sigma(1)}, (\varphi_m)_{m \in M}\Big) \end{equation*}
where:
\begin{enumerate}
\item $(C,x_1,\ldots,x_n)$ is a prestable curve of genus $g$ with $n$ marked points;
\item the $L_\rho$ are line bundles on $C$ of degree $d_\rho = D_\rho \cdot \beta$;
\item the $u_\rho$ are global sections of $L_\rho$;
\item $\varphi_m \colon \bigotimes_{\rho \in \Sigma(1)} L_\rho^{\otimes \langle \rho, m \rangle} \to \OO_C$ are isomorphisms, such that $\varphi_{m} \otimes \varphi_{m^\prime} = \varphi_{m + m^\prime}$ for all $m, m^\prime \in M$.
\end{enumerate}
These are required to satisfy the following two conditions:
\begin{enumerate}
\item \ilemph{nondegeneracy}: there is a finite (possibly empty) set of smooth and non-marked points $B \subseteq C$, called the \ilemph{basepoints} of the quasimap, such that for all $x \in C \setminus B$ there exists a maximal cone $\sigma \in \Sigma_{\operatorname{max}}$ with $u_\rho(x) \neq 0$ for all $\rho \not\subset \sigma$;
\item \ilemph{stability}: if we let $L = \otimes_\rho L_\rho^{\otimes \alpha_\rho}$ then the following $\QQ$-divisor is ample
\begin{equation*} \omega_C(x_1 + \ldots + x_n)\otimes L^{\otimes \epsilon} \end{equation*}
for every rational $\epsilon > 0$.  This does not depend on the choice of polarisation. Note that necessarily $2g-2+n \geq 0$.
\end{enumerate}
\end{definition}

\begin{remark} This definition is motivated by D. A. Cox's description of the functor of points of a toric variety in terms of $\Sigma$-collections \cite{CoxFunctor}; see also Appendix \ref{Functoriality of Quasimap Spaces Section}. A quasimap defines\footnote{This can be expressed in a more generalisable manner as follows: a quasimap is a map to the stack quotient $\big[\mathbb{A}^{\Sigma(1)} / \Gm^r\big]$ such that $B$ is the preimage of the unstable locus.} a rational map $C \dashrightarrow X$ with base locus equal to $B$.
In particular a quasimap without any basepoints defines a morphism $C \to X$. Thus maps with basepoints appear in the (virtual) boundary of the moduli space of quasimaps, in much the same way as maps with rational tails appear in the boundary of the moduli space of stable maps. This is something more than just a vague analogy; these loci correspond to each other under the comparison morphism when $X\cong\PP^N$; see Appendix \ref{Section comparison morphism}. \end{remark}

More generally, one can define the notion of a family of quasimaps over a base scheme $S$, and what it means for two such families to be isomorphic; one thus obtains a moduli stack
\begin{equation*} \Q{g}{n}{X}{\beta} \end{equation*}
of stable (toric) quasimaps to $X$, which is a proper Deligne--Mumford stack of finite type \cite[\S 3]{CF-K}.

\medskip

As with the case of stable maps, there is a combinatorial characterisation of stability which is easy to check in practice; a prestable quasimap is stable if and only if the following conditions hold:
\begin{enumerate}
\item the line bundle $L = \otimes_\rho L_\rho^{\otimes \alpha_\rho}$ must have strictly positive degree on any rational component with fewer than three special points, and on any elliptic component with no special points;
\item $C$ cannot have any rational components with fewer than two special points (that is, no \emph{rational tails}).
\end{enumerate}
Condition (1) is analogous to the ordinary stability condition for stable maps. Condition (2) is new, however, and gives quasimaps a distinctly different flavour to stable maps; we shall sometimes refer to it as the \ilemph{strong stability condition}. 

%\begin{remark}
% Notice that quasimap spaces are ``smaller'', so if a comparison morphism exists it should be from stable maps to quasimaps; in fact a morphism between the moduli spaces comes with a morphism of the universal curves, and this ought to be a contraction of the rational tails (the other way round, one could sprout a rational tail from any base points, and maybe even the degree of the line bundles would be determined, but a fundamental indeterminacy remains as to what the sections should be, i.e. where does the rational tail map).
%\end{remark}

\begin{remark} Unlike in Gromov--Witten theory, $\Q{g}{n+1}{X}{\beta}$ is \emph{not} the universal curve over $\Q{g}{n}{X}{\beta}$ since markings cannot be basepoints. In fact there is not even a morphism between these spaces in general.\end{remark}
%Instead, the universal curve $\mathcal C_{g,n}^\mathcal{Q}$ is a substack of $\MM_{g,n+1}(X,\beta)$ and comparing the geometry of $\om{Q}_{g,n+1}$ and $\mathcal{C}^{\mathcal{Q}}_{g,n}$ results in \emph{string transformations} that determine the difference between generating functions for $\epsilon_1$- and $\epsilon_2$-stable quasimaps \cite[\S 6 and 7]{CF-K-wallcrossing}.

The moduli space $\Q{g}{n}{X}{\beta}$ admits a perfect obstruction theory relative to the moduli space $\MM_{g,n}$ of source curves \cite[\S 5]{CF-K}, and hence one can construct a virtual class
\begin{equation*} \virt{\Q{g}{n}{X}{\beta}} \in \Achow_{\operatorname{vdim}\Q{g}{n}{X}{\beta}} \left( \Q{g}{n}{X}{\beta} \right) \end{equation*}
where the virtual dimension is the same as for stable maps:
\begin{equation*} \operatorname{vdim}\Q{g}{n}{X}{\beta} = (\dim{X} - 3)(1-g) - (K_X \cdot \beta) + n \end{equation*}
Since the markings are not basepoints there exist evaluation maps
\begin{equation*} \ev_i : \Q{g}{n}{X}{\beta} \to X \end{equation*}
and there are $\psi$-classes defined in the usual way by pulling back the relative dualising sheaf of the universal curve
\begin{equation*} \psi_i = \operatorname{c}_1(x_i^* \omega_{\mathcal{C}/\om{Q}}) \end{equation*}
where $\mathcal{C} \to \om{Q} = \Q{g}{n}{X}{\beta}$ is the universal curve and $x_i : \om{Q} \to \mathcal{C}$ is the section defining the $i$th marked point. Putting all these pieces together, we can define \ilemph{quasimap invariants}:
\begin{equation*} \langle \gamma_1 \psi_1^{k_1} , \ldots, \gamma_n \psi_n^{k_n} \rangle_{g,n,\beta}^X = \int_{\virt{\Q{g}{n}{X}{\beta}}} \prod_{i=1}^n \ev_i^* (\gamma_i) \cdot \psi_i^{k_i} \end{equation*}
We use the same correlator notation as in Gromov--Witten theory; this should not cause any confusion.

\begin{example} Consider $\Q{0}{2}{\PP^2}{1}$. What are its objects? By the strong stability condition (2) above, we see that the source curve must be irreducible. On the other hand since $\PP^2$ has Picard rank $1$ we may exploit the isomorphisms $\varphi_m$ to reduce ourselves to considering one line bundle equipped with three sections. Thus the data of the quasimap is $((C,x_1,x_2),L,u_0,u_1,u_2)$ where $(C,L)\cong(\PP^1,\OO_{\PP^1}(1))$.

Pick coordinates $[s:t]$ on $\PP^1$ such that the marked points are $[1:0]$ and $[0:1]$. We can express the sections as $u_i=a_is+b_it$; the requirement that the markings are not basepoints then translates into the following stability condition:
 \[
  (a_0,a_1,a_2)\neq(0,0,0)\quad \text{and}\quad (b_0,b_1,b_2)\neq(0,0,0).
 \]
The group $\Aut(C;x_1,x_2)\cong\Gm$ acts by rotation $\lambda\colon[s:t]\mapsto[s:\lambda t]$, while $\Aut(L)\cong\Gm$ acts by scalar multiplication on $\underline{a}$ and $\underline{b}$. Thus the $\Gm^2$ action on $\mathbb A^6_{\underline{a} ,\underline{b}}$ is encoded by the following weight matrix:
\[ \left[ \begin{array}{cccccc}
1 & 1 & 1 & 1 & 1 & 1 \\
0 & 0 & 0 & 1 & 1 & 1 \end{array} \right]\] 
It is now clear that the quotient is $\PP^2\times\PP^2$; in fact, we see that the evaluation map
\[
 (\ev_1,\ev_2)\colon\Q{0}{2}{\PP^2}{1}\to\PP^2\times\PP^2
\]
is an isomorphism. It is given in the above notation by:
\begin{equation*} ((\PP^1;[1:0],[0:1]); \OO_{\PP^1}(1); u_0, u_1, u_2) \mapsto ([a_0 : a_1 : a_2],[b_0 : b_1 : b_2]) \end{equation*}
Notice that the locus where $(a_0,a_1,a_2)=\mu(b_0,b_1,b_2)$, i.e. the diagonal in $\PP^2 \times \PP^2$ is precisely the locus of quasimaps which have a basepoint. The point $[a_0:a_1:a_2]=[b_0:b_1:b_2]\in\PP^2$ is the image of the underlying ``residual map'' of degree 0, obtained by dividing all the sections by a local equation of the basepoint (equivalently, by extending the rational map $C\dashrightarrow \PP^2$ to a morphism $C \to \PP^2$).

On the other hand, $(\ev_1,\ev_2)\colon\M{0}{2}{\PP^2}{1}\to\PP^2\times\PP^2$ is \emph{not} an isomorphism. Off the diagonal, the images of the two marked points determine uniquely the image of the stable map, i.e. the line through them. On the diagonal however, the following maps with a rational tail appear:
\begin{center}
\begin{tikzpicture}[scale=.6]
 \draw[thick] (-1,0) -- (3,0) node[right]{d=1};
 \draw (0,-1) -- (0,3) node[above]{d=0};
 \draw[fill=black] (0,1) circle[radius=3pt];
 \draw[fill=black] (0,2) circle[radius=3pt];
\end{tikzpicture} 
\end{center}
The image of the degree $1$ component under $f$ can be any line passing through the point of $\PP^2$ to which the other component is contracted. Hence $\M{0}{2}{\PP^2}{1}\cong\operatorname{Bl}_{\Delta}(\PP^2\times\PP^2)$. The comparison morphism $\comp$ (see Appendix \ref{Section comparison morphism}) can be interpreted as the blow-down map, and it induces an isomorphism of the rational tail-free locus with the basepoint-free locus.
\end{example}

%[EXAMPLE: $\Q{0}{2}{\PP^2}{1}$ and $\M{0}{2}{\PP^2}{1}$]

\begin{remark}
 There is a more general notion of \ilemph{$\epsilon$-stable quasimap} \cite[\S 7.1]{CFKM}. Here the stability condition, namely that the line bundle
\begin{equation*} \omega_C(x_1 + \ldots + x_n)\otimes L^{\otimes \epsilon} \end{equation*} 
is ample, is only required to hold for a fixed $\epsilon \in \QQ_{>0}$ (instead of for arbitrary $\epsilon$, as was the case with ordinary quasimaps).

This has the effect of allowing some rational tails to appear, as long as their degree is high enough with respect to $\epsilon$. In order to keep the moduli space Deligne--Mumford and seperated, one also has to bound the multiplicity of the basepoints that can occur.

By boundedness and the fact that the degree is an integer-valued function, there exist finitely many critical values of $\epsilon$ which divide $\QQ_{>0}$ into chambers inside which the moduli spaces $\Qe{g}{n}{X}{\beta}$ do not change.
For $\epsilon$ sufficiently small we recover the space of (ordinary) quasimaps, and for $\epsilon$ sufficiently large we obtain the moduli space of stable maps. Thus one can view the spaces of $\epsilon$-stable quasimaps as interpolating between these two extremes, and they have proven  successful as a tool for comparing quasimap invariants to stable map invariants \cite{TodaStableQuotient} \cite{CF-K-wallcrossing}.
\end{remark}

Another variant of the theory, which will play a role in later sections, is that of \emph{parametrised quasimaps} \cite[\S 7]{CF-K}. A parametrised quasimap comes with a preferred rational component, which is equipped with the extra data of an isomorphism with $\PP^1$, and the stability condition is imposed \emph{on all but the preferred component}. This mimics the construction of graph spaces in Gromov-Witten theory -- for example, there is a $\Gm$-action on $\QG{g}{n}{X}{\beta}$ by rotating the preferred component, which plays the role of the $\Gm$-action that rotates the graph. The fixed loci and their equivariant normal bundles are well-understood, at least in the toric setting \cite[\S 7]{CF-K}.  In the parametrised case we no longer require $2g-2+n\geq 0$, due to the modified stability condition. In particular it makes sense, and turns out to be very useful, to consider unmarked parametrised quasimaps $\QG{0}{0}{X}{\beta}$. In this case the source curve is necessarily irreducible. 

\begin{example} $\QG{0}{0}{\PP^N}{d} = \PP^k$ with $k=(N+1)(d+1)-1$. Indeed, the curve and line bundle must be $(\PP^1,\mathcal O_{\PP^1}(d))$ and we are left with choosing $N+1$ sections of $\mathcal O_{\PP^1}(d)$ (not all zero) up to automorphisms of $\mathcal O_{\PP^1}(d)$, i.e. up to scaling. For early appearances of such spaces, see for instance \cite{Givental-mirror} \cite{MorrisonPlesser} \cite{Bertram}.\end{example}

\subsection{Review of relative stable maps} \label{Subsection relative stable maps} Given a smooth projective variety $X$ and a smooth very ample divisor $Y$, Gathmann's moduli space of relative stable maps parametrises stable maps to $X$ with specified tangencies to $Y$ at the marked points.

\begin{definition} {\cite[Definition 1.1]{Ga}}  Let $X$ be a smooth projective variety and $Y \subseteq X$ a smooth very ample divisor. Fix a number $n \geq 0$ of marked points, an effective curve class $\beta \in \HH_2^+(X)$ and an $n$-tuple $\alpha = (\alpha_1, \ldots, \alpha_n)$ of non-negative integers such that $\Sigma_i \alpha_i \leq Y \cdot \beta$. The moduli space
\begin{equation*} \M{0}{\alpha}{X|Y}{\beta} \end{equation*}
of relative stable maps to $(X,Y)$ is defined to be the locus in $\M{0}{n}{X}{\beta}$ of stable maps $(C \to S , (x_i : S \to C)_{i=1}^n , f : C \to X)$ satisfying the following two conditions:
\begin{enumerate}
\item if $x_i$ is a marked point such that $\alpha_i > 0$ then $f(x_i) \in Y$;
\item if we consider the class $f^* [Y] \in \Achow^0(f^{-1}Y \to S)$ then the difference $f^* [Y] - \sum_i \alpha_i [x_i]$ is an effective class.
\end{enumerate}
These conditions define a closed substack of $\M{0}{n}{X}{\beta}$. Condition (1) is required in order for the class $\sum_i \alpha_i [x_i]$ to make sense in $\Achow^0(f^{-1}Y \to S)$.
\end{definition}

\begin{remark} The notation in (2) comes from bivariant intersection theory: see \cite[\S 17]{FUL}. Fibrewise, the condition is that the class $f^*[Y] - \sum_i \alpha_i [x_i] \in A_0(f^{-1}Y)$ is required to be effective.\end{remark}

The definition given above works in families; however there is an equivalent, more combinatorial definition for individual maps which is more useful in practice (see \cite[Remark 1.4]{Ga}): a stable map $(C,x_1, \ldots, x_n,f)$ is a relative stable map if and only if, for each connected component $Z$ of $f^{-1}(Y) \subseteq C$:
\begin{enumerate}
\item if $Z$ is a point and is equal to a marked point $x_i$, then the multiplicity of $f$ to $Y$ at $x_i$ is greater than or equal to $\alpha_i$;
\item if $Z$ is one-dimensional (hence a union of irreducible components of $C$) and if we let $C^{(i)}$ for $1 \leq i \leq r$ denote the irreducible components of $C$ adjacent to $Z$ and $m^{(i)}$ denote the multiplicity of $f|_{C^{(i)}}$ to $Y$ at the node $Z \cap C^{(i)}$, then:
\begin{equation} \label{Relative stable map internal component inequality} Y \cdot f_* [Z] + \sum_{i=1}^r m^{(i)} \geq \sum_{x_i \in Z} \alpha_i \tag{\textasteriskcentered} \end{equation}
\end{enumerate}

\begin{remark} In case (2) above we call $Z$ an \ilemph{internal} component and the $C^{(i)}$ \ilemph{external} components. Note that $Z$ is not necessarily irreducible. \end{remark}

\begin{remark} When $\alpha = (0, \ldots, 0)$, condition (2) becomes $Y \cdot \beta \geq 0$, so $\M{0}{(0,\ldots,0)}{X|Y}{\beta} = \M{0}{n}{X}{\beta}$ as long as $Y$ is nef.\end{remark}

\begin{remark} In the case of maximal multiplicity $\Sigma_{i} \alpha_i = Y \cdot \beta$, all the inequalities in the above definition must be equalities. \end{remark}

In the case $X=\PP^N$ and $Y=H$ a hyperplane, Gathmann showed \cite[Proposition 1.14]{Ga} that $\M{0}{\alpha}{\PP^N|H}{d}$ is irreducible with dimension equal to the expected dimension:
\begin{equation*} \vdim \M{0}{\alpha}{X|Y}{\beta} = \vdim{\M{0}{n}{X}{\beta}} - \sum_{i=1}^n \alpha_i \end{equation*}
Hence it has a fundamental class from which one can define relative Gromov--Witten invariants.
More generally if $Y \subseteq X$ is very ample one can use the embedding $X \hookrightarrow \PP^N$ given by $|\OO_X(Y)|$ to obtain a cartesian diagram:
\bcd
\M{0}{\alpha}{X|Y}{\beta} \ar[r] \ar[d] \ar[rd,phantom,"\square"] & \M{0}{\alpha}{\PP^N|H}{d} \ar[d] \\
\M{0}{n}{X}{\beta} \ar[r,"\varphi"] & \M{0}{n}{\PP^N}{d}
\ecd
Then the fact that $\M{0}{n}{\PP^N}{d}$ is smooth allows one to define a virtual class on $\M{0}{\alpha}{X|Y}{\beta}$ by diagonal pull-back (see Appendix \ref{appendix:intersection} of the current paper):
\begin{equation*} \virt{\M{0}{\alpha}{X|Y}{\beta}} := \varphi^! [\M{0}{\alpha}{\PP^N|H}{d}] \end{equation*}
Thus one can define relative Gromov--Witten invariants in the usual way, by capping the virtual class with products of evaluation classes and psi classes.

In \cite[\S\S 2-4]{Ga} Gathmann establishes a recursion relation inside the Chow group of $\M{0}{\alpha}{X|Y}{\beta}$. This describes what happens when we increase the multiplicity at one of the marked points by $1$. Let us therefore fix a marked point $x_k \in \{ x_1, \ldots, x_n \}$ and let $e_k = (0,\ldots,1,\ldots,0)$. Then
\begin{equation*} (\alpha_k \psi_k + \ev_k^* [Y]) \cap \virt{\M{0}{\alpha}{X|Y}{\beta}} = \virt{\M{0}{\alpha+e_k}{X|Y}{\beta}} + \virt{\mathcal{D}_{\alpha,k}(X,\beta)} \end{equation*}
where $\mathcal{D}_{\alpha,k}(X,\beta)$ is an appropriate \ilemph{comb locus}. This parametrises relative stable maps where the component containing $x_k$ is mapped entirely into $Y$, and which satisfy inequality \eqref{Relative stable map internal component inequality} for $\alpha$ but not for $\alpha+e_k$; these form a divisor in $\M{0}{\alpha}{X|Y}{\beta}$.

Repeated application of this result shows that both the relative Gromov--Witten invariants of $(X,Y)$ and the (restricted) Gromov--Witten invariants of $Y$ are completely determined by the Gromov--Witten invariants of $X$ \cite[Corollary 5.7]{Ga}. This result is then applied in \cite{Ga-MF} to obtain a new proof of the mirror theorem for projective hypersurfaces.

\begin{remark} There are many other approaches to defining relative stable maps besides Gathmann's: the moduli space of maps to expanded degenerations of J.~Li \cite{Li1} \cite{Li2}, the twisted stable maps of D.~Abramovich and B.~Fantechi \cite{AbramovichFantechi}, the logarithmic stable maps with expansions of B.~Kim \cite{KimLog} and the logarithmic stable maps (without expansions) of M.~Gross and B.~Siebert \cite{GrossSiebertLog} \cite{GrossSiebertIntrinsic}, Q.~Chen \cite{ChenLog} and D.~Abramovich and Q.~Chen \cite{AbramovichChenLog}. However, the invariants defined via these theories are all known to coincide \cite{AbramovichMarcusWiseComparison} \cite{GathmannThesis}, so the choice of which moduli space to work with mainly depends on one's intended application. \end{remark}

\subsection{Definition of relative stable quasimaps} \label{Subsection relative stable quasimaps}

We now give the main definition of the paper. From here on $X$ will denote a smooth projective toric variety and $Y \subseteq X$ a very ample hypersurface. We \emph{do not} require that $Y$ is toric.
Consider the line bundle $\OO_X(Y)$ and the section $s_Y$ cutting out $Y$. By \cite{CoxRing} we have a natural isomorphism of $\CC$-vector spaces
\begin{equation*} \HH^0(X,\OO_X(Y)) = \left\langle \prod_{\rho} z_\rho^{a_\rho} : \Sigma_\rho a_\rho [D_\rho] = [Y] \right\rangle_{\CC} \end{equation*}
where the $z_\rho$ for $\rho \in \Sigma(1)$ are the generators of the Cox ring of $X$ and the $a_\rho$ are non-negative integers. We can therefore write $s_Y$ as
\begin{equation*} s_Y = \sum_{\underline{a}=(a_\rho)} \lambda_{\underline{a}} \prod_\rho z_\rho^{a_\rho} \end{equation*}
for some scalars $\lambda_{\underline{a}} \in \CC$. The idea is that a quasimap
\begin{equation*} \big((C,x_1,\ldots,x_n), (L_\rho,u_\rho)_{\rho \in \Sigma(1)}, (\varphi_m)_{m \in M}\big) \end{equation*}
should ``map'' a point $x \in C$ into $Y$ if and only if the section
\begin{equation} \label{uY expression} u_Y := \sum_{\underline{a}} \lambda_{\underline{a}} \prod_\rho u_\rho^{a_\rho} \end{equation}
vanishes at $x$. We now explain how to make sense of expression \eqref{uY expression}. For each $\underline{a}$ we have a well-defined section
\begin{equation*} u_{\underline{a}} := \lambda_{\underline{a}} \prod_\rho u_\rho^{a_\rho} \in \HH^0(C,\otimes_\rho L_\rho^{\otimes a_\rho}) \end{equation*}
and if we have $\underline{a}$ and $\underline{b}$ such that $\sum_\rho a_\rho [D_\rho] = [Y] = \sum_\rho b_\rho [D_\rho]$ then these divisors differ by an element $m$ of $M$. Thus the isomorphism $\varphi_m$ allows us to view the sections $u_{\underline{a}}$ and $u_{\underline{b}}$ as sections of the same bundle, which we denote by $L_Y$. Then we can sum these together to obtain $u_Y$. There is a choice involved here, but up to isomorphism it does not matter; see the proof of functoriality in Appendix \ref{Functoriality of Quasimap Spaces Section} for more details.

The upshot is that we obtain a line bundle $L_Y$ on $C$, which plays the role of the ``pull-back'' of $\OO_X(Y)$ along the ``map'' $C \to X$, and a global section
\begin{equation*} u_Y \in \HH^0(C,L_Y) \end{equation*}
which plays the role of the ``pull-back'' of $s_Y$.

\begin{definition} With notation as above, let $n \geq 2$ be a number of marked points, $\beta \in \HH_2^+(X)$ be an effective curve class and $\alpha=(\alpha_1, \ldots, \alpha_n)$ be a collection of non-negative integers such that $\Sigma_i \alpha_i \leq Y \cdot \beta$. The \ilemph{moduli space of relative stable quasimaps}
\begin{equation*} \Q{0}{\alpha}{X|Y}{\beta} \subseteq \Q{0}{n}{X}{\beta} \end{equation*}
is defined to be the locus of quasimaps
\begin{equation*} \big((C \to S, (x_i : S \to C)_{i=1}^n), (L_\rho,u_\rho)_{\rho \in \Sigma(1)}, (\varphi_m)_{m \in M}\big) \end{equation*}
such that:
\begin{enumerate}
\item if $x_i$ is a marking such that $\alpha_i > 0$, then $x_i^* u_Y = 0$;
\item if we let $u_Y^*(0) \in \Achow^0(u_Y^{-1}(0)\to S)$ denote the class defined by the Gysin map for $L_Y$, then the difference $u_Y^*(0) - \Sigma_i \alpha_i [x_i]$ is an effective class.
\end{enumerate}
\end{definition}

\noindent The class $u_Y^*(0)$ is defined as follows. Consider the cartesian diagram
\bcd
u_Y^{-1}(0) \ar[r] \ar[d] \ar[rd,phantom,"\square"] & C \ar[d,"u_Y"] \\
C \ar[r,"0_Y"] & L_Y
\ecd
where $0_Y$ is the zero section. There is a Gysin map \cite[\S 2.6]{FUL}
\begin{equation*} 0_Y^! : \Achow_*(C) \to \Achow_*(u_Y^{-1}(0)) \end{equation*}
and we define $u_Y^*(0) := 0_Y^!([C])$.


\begin{remark} As in the case of relative stable maps (see \S \ref{Subsection relative stable maps}) there is an equivalent definition which is more useful in practice: a quasimap is a relative quasimap if and only if for every connected component $Z$ of $u_Y^{-1}(0)$ we have that:
\begin{enumerate}
\item if $Z$ is a point and is equal to a marked point $x_i$, then the order of vanishing of $u_Y$ at $x_i$ is greater than or equal to $\alpha_i$;
\item if $Z$ is one-dimensional (hence a union of irreducible components) and if we let $C^{(i)}$ for $1 \leq i \leq r$ denote the irreducible components of $C$ adjacent to $Z$ and $m^{(i)}$ the order of vanishing of $u_Y$ at the node $Z \cap C^{(i)}$, then:
\begin{equation} \label{Relative quasimap internal component inequality} \deg L_Y|_Z + \sum_{i=1}^r m^{(i)} \geq \sum_{x_i \in Z} \alpha_i \tag{\textasteriskcentered\textasteriskcentered} \end{equation}
\end{enumerate}
\end{remark}

\begin{remark} The definitions above also make sense for $\epsilon$-stable quasimaps for $\epsilon > 0$ an arbitrary rational number. Thus we have a notion of $\epsilon$-stable relative quasimap. \end{remark}