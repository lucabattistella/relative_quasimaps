\section{Introduction}
In this paper we construct moduli spaces of relative quasimaps as substacks of moduli spaces of (absolute) quasimaps. This provides a common generalisation of two theories: stable quasimaps on the one hand, and relative stable maps (in the sense of A. Gathmann) on the other. In this introductory section we briefly summarise these, providing the context for our work.

\subsection{Stable quasimaps}
The moduli space of \ilemph{stable toric quasimaps}
\begin{equation*}\Q{g}{n}{X}{\beta}\end{equation*}
was constructed by I. Ciocan-Fontanine and B. Kim \cite{CF-K} as a compactification of the moduli space of smooth curves in a smooth and complete toric variety $X$. It is a proper Deligne--Mumford stack of finite type, and admits a virtual fundamental class, which is used to define curve-counting invariants for $X$, called \ilemph{quasimap invariants}.

This theory agrees with that of stable quotients \cite{MOP} when both are defined, namely when $X$ is a projective space.  There is a common generalisation given by the theory of stable quasimaps to GIT quotients \cite{CFKM}. For simplicity, however, we will work mostly in the toric setting (though this restriction is probably not essential for our arguments). Thus when we say ``quasimap'' we are implicitly talking about a toric quasimap.

[GIVENTAL, MUSTATA+MUSTATA, MORISSON-PLESSER]

The quasimap invariants are expected to coincide with the Gromov--Witten invariants when $X$ is a toric Fano variety \cite{CM}; this has been proven for $X=\PP^N$ \cite[Theorem 3]{MOP} \cite[\S 5.4]{Manolache-Push}. More generally, the case of a projective complete intersection of Fano index at least $2$ can be obtained by combining the work of A. Givental on the mirror theorem \cite[Theorem 0.1]{Givental-mirror} and the work of I. Ciocan-Fontanine and B. Kim on wall-crossing for $\epsilon$-stable quasimaps \cite[Conjecture 7.2.10]{CF-K}\  \cite[\S 5.5 and Conjecture 6.3.1]{CF-K-wallcrossing}.

In general, however, the invariants differ, the difference being encoded by certain wall-crossing formulas \cite{CF-K-wallcrossing}. The motivation for this comes from mirror symmetry: the idea is that the quasimap invariants of $X$ should correspond to the $B$-side theory of $X^\vee$ (this is in contrast to the Gromov--Witten invariants, which live on the $A$-side); see \cite[\S 7]{CF-K}.

%An important fact for understanding the difference between these two theories is the observation that $\Q{g}{n+1}{X}{\beta}$ is \emph{not} the universal curve over $\Q{g}{n}{X}{\beta}$; in fact there is not even a morphism between these spaces in general. Instead, the universal curve $\mathcal C_{g,n}^\mathcal{Q}$ is a substack of $\MM_{g,n+1}(X,\beta)$ and comparing the geometry of $\om{Q}_{g,n+1}$ and $\mathcal{C}^{\mathcal{Q}}_{g,n}$ results in \emph{string transformations} that determine the difference between generating functions for $\epsilon_1$- and $\epsilon_2$-stable quasimaps \cite[\S 6 and 7]{CF-K-wallcrossing}.

\subsection{Relative stable maps}
In \cite{Ga} A. Gathmann constructs a moduli space of relative stable maps to the pair $(X,Y)$ as a closed substack of the moduli space of (absolute) stable maps to $X$
\begin{equation*} \M{0}{\alpha}{X|Y}{\beta} \hookrightarrow \M{0}{n}{X}{\beta} \end{equation*}
parametrising stable maps with prescribed tangencies to $Y$ at the marked points.

Unfortunately this space does not admit a natural perfect obstruction theory. Nevertheless in the case where $Y$ is very ample it is still possible to construct a virtual fundamental class by intersection-theoretic methods, and hence one can define relative Gromov--Witten invariants.

There is a recursion formula for these virtual classes which allows one to express any relative invariant of $(X,Y)$ in terms of absolute invariants of $Y$ and relative invariants with lower contact multiplicities. 

%In particular, when the required multiplicity is higher than is possible for degree reasons, one recovers absolute invariants of $Y$.

By succesively increasing the contact multiplicites from zero to the maximum value possible, Gathmann obtains an algorithm expressing the absolute invariants of $Y$ in terms of those of $X$. This is explained in \cite[Corollary 5.7]{Ga}. In \cite{Ga-MF} this result is applied to obtain an alternative proof of the mirror theorem for projective hypersurfaces \cite{Giv} \cite{LLY}.

\subsection{Relative stable quasimaps}
In this paper we combine the two stories above, constructing moduli spaces of relative stable quasimaps in genus $0$. We prove a recursion relation similar to Gathmann's formula, and use this to derive a quantum Lefschetz formula for quasimap invariants.

The plan of the paper is as follows. In \S\S \ref{Subsection stable quasimaps}-\ref{Subsection relative stable maps} we provide a brief review of the theories of stable quasimaps and relative stable maps mentioned above. Then in \S \ref{Subsection relative stable quasimaps} we define the moduli spaces of relative stable quasimaps
\begin{equation*} \Q{g}{\alpha}{X|Y}{\beta} \end{equation*}
where $X$ is a smooth toric variety and $Y$ is a smooth hypersurface. We \emph{do not} require that $Y$ is toric.

In \S \ref{Section recursion for PN} we examine the special case of $H \subseteq \PP^N$. We find that, although the moduli space is not in general smooth, it is irreducible of the expected dimension (in fact, more than this: it is the closure of the so-called ``nice locus'' consisting of maps from a smooth curve which do not land inside the hypersurface). Thus it has an actual fundamental class which we can use to define relative quasimap invariants.

Also for $\PP^N$ there exists a comparison morphism from the moduli space of stable maps to the moduli space of quasimaps, which is birational. We use this morphism to push down Gathmann's recursion formula for relative stable maps to obtain a recursion formula for relative stable quasimaps. The stronger stability condition for quasimaps significantly simplifies the correction terms which appear.

In \S \ref{Section recursion formula in general case} we extend the recursion formula to arbitrary pairs $(X,Y)$ where $Y$ is very ample, by taking the embedding $X \hookrightarrow \PP^N$ defined by $\OO_X(Y)$ and pulling back the formula for $(\PP^N,H)$. This of course requires some comparison theorems for virtual classes, for which we have to examine the perfect obstruction theories.

In \S \ref{Section quasimap mirror theorem} we apply the recursion formula obtained in \S \ref{Section recursion formula in general case} to obtain a quantum Lefschetz theorem for quasimap invariants. This recovers Corollary 5.5.1 in \cite{CF-K-wallcrossing}.

We also include several appendices, collecting together results which are presumably well-known to experts, but for which we could not find references in the literature.

Appendix \S \ref{Section comparison morphism} discusses the comparison morphism from maps to quasimaps (used in the proof of the recursion relation in \S \ref{Section recursion for PN}).

Appendix \S \ref{Section quasimap theory} contains foundational lemmas of quasimap theory, including:
\begin{enumerate}
\item \ilemph{Functoriality} of quasimap spaces: given a morphism $f\colon Y\to X$ which does not contract effective curve classes (e.g. an embedding) we describe the induced map:
\begin{equation*} \overline{\mathcal{Q}}(f)\colon\Q{g}{n}{Y}{\beta}\to\Q{g}{n}{X}{f_*\beta} \end{equation*}
For stable maps this is easy to define: simply compose a stable map with $f$. However for quasimaps one must be slightly more ingenious in order to cook up the structure of a quasimap which is compatible with the interpretation as composition in the case where the quasimap has no base-points. The construction we give is based on the description of the functor of a smooth toric variety and its homogeneous coordinate ring by D. Cox \cite{CoxFunctor}\cite{CoxRing}. We also discuss when $\overline{\mathcal{Q}}(f)$ admits a compatible perfect obstruction theory.
\item \ilemph{Splitting axiom}: this gives an equality between two natural virtual classes on boundary strata (i.e. loci where the underlying curve is reducible of a prescribed type).
\item Comparison with the \ilemph{GIT construction} \cite{CFKM} in the case where $Y$ is a (not necessarily toric) very ample hypersurface in a toric variety $X$.
\end{enumerate}

Finally Appendix \ref{appendix:intersection} discusses the (\emph{diagonal}) pull-back along a morphism whose target is unobstructed (used in \cite{Ga}) and verifies that this agrees with the virtual pull-back of \cite{Manolache-Pull} (when both are defined) and that it satisfies expected compatibility properties.


\begin{acknowledgements} The authors wish to thank Tom Coates, Cristina Manolache and Andrea Petracci for many helpful discussions. L.B. is supported by [REF] and N.N. is supported by [REF]
\end{acknowledgements}

\subsection{Table of notation} We will use the following notation, most of which is introduced in the main body of the paper.
\begin{longtabu}{r c c p{0.8\linewidth}}
$X$ & & & a smooth projective toric variety \\
$Y$ & & & a very ample hypersurface in $X$ \\
$\Sigma$ & & & the fan of $X$ \\
$\Sigma(1)$ & & & the set of $1$-dimensional cones of $\Sigma$ \\
$\rho$ & & & an element of $\Sigma(1)$ \\
$D_\rho$ & & & the toric divisor in $X$ corresponding to $\rho$ \\
$\M{g}{n}{X}{\beta}$ & & & the moduli space of stable maps \\
$\mathcal{M}_{0,\alpha}(X|Y,\beta)$ & & & the nice locus of relative stable maps \\
$\M{0}{n}{X|Y}{\beta}$ & & & the moduli space of relative stable maps (\S \ref{Subsection relative stable maps}) \\
$\Q{g}{n}{X}{\beta}$ & & & the moduli space of toric quasimaps (\S \ref{Subsection stable quasimaps}) \\
$\mathcal{Q}_{0,\alpha}(X|Y,\beta)$ & & & the nice locus of relative quasimaps (\S \ref{Subsection basic properties of the moduli space}) \\
$\Q{0}{\alpha}{X|Y}{\beta}$ & & & the moduli space of relative quasimaps (\S \ref{Subsection relative stable quasimaps}) \\
$\mathcal{D}^{\mathcal{Q}}_{\alpha,k}(X|Y,\beta)$ & & & the quasimap comb locus (\S \ref{Subsection recursion formula for PN}) \\
$\mathcal{D}^{\mathcal{Q}}(X|Y,A,B,M)$ & & & (a component of) the (quasimap) comb locus (\S \ref{Subsection recursion formula for PN}) \\
$\mathcal{E}^{\mathcal{Q}}(X|Y,A,B,M)$ & & & the total product for the comb locus (\S [REF]) \\
$\mathcal{D}^{\mathcal{Q}}(X,A,B)$ & & & the quasimap centipede locus (\S [REF])  \\
$\mathcal{E}^{\mathcal{Q}}(X,A,B)$ & & & the total product for the cenitepede locus (\S [REF]) \\
$\MM_{g,n}$ & & & the moduli stack of prestable curves \\
$\mathfrak{Pic}$ & & & the relative Picard stack of the universal curve over $\MM_{g,n}$ \\
$\mathfrak{Bun}_{G}$ & & & the moduli stack of principal $G$-bundles on the universal curve over $\MM_{g,n}$ \\
$\om{Q}(f)$ & & & the push-forward between quasimap spaces (\S \ref{Functoriality of Quasimap Spaces Section}) \\
$\chi$ & & & the comparison map from stable maps to quasimaps (\S \ref{Section comparison morphism}) \\
$f^!$ & & & Gysin morphism for $f$ an l.c.i. embedding \\
$f^!_{\text{v}}$ & & & virtual pull-back for $f$ virtually smooth (\S [REF]) \\
$f^!_{\Delta}$ & & & diagonal virtual pull-back (\S [REF])
\end{longtabu}