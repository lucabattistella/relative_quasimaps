\subsection{Comparison with the GIT construction} \label{Section comparison with GIT construction}

Let $X$ be a smooth projective toric variety and $Y \hookrightarrow X$ a smooth very ample hypersurface. The complete linear system $|\OO_X(Y)|$ gives an embedding $i : X \hookrightarrow \PP^N$ such that $Y$ is the intersection of $X$ with a certain hyperplane $H$ inside $\PP^N$. We can define the moduli space of quasimaps to $Y$ via the following cartesian diagram:
\bcd
\Q{g}{n}{Y}{\beta}\ar[d]\ar[r]\ar[rd,phantom,"\square"] & \Q{g}{n}{H}{d} \ar[d,] \\
\Q{g}{n}{X}{\beta}\ar[r,"k"] & \Q{g}{n}{\PP^N}{d}
\ecd
where $d=i_*\beta$. Thus defined, this moduli space is easy to describe. Let $s_Y$ denote the section of $\OO_X(Y)$ cutting out $Y$ inside $X$. Recall from \S \ref{Subsection relative stable quasimaps} that for any quasimap
\begin{equation*} ((C,x_1,\ldots,x_n),(L_\rho,u_\rho)_{\rho \in \Sigma_X(1)}, (\varphi_m)_{m \in M_X}) \in \Q{g}{n}{X}{\beta} \end{equation*}
we can construct a section $u_Y$ of a line bundle $L_Y$ on $C$, which plays the role of the pull-back of $s_Y$ to $C$. Then
\begin{equation*} \Q{g}{n}{Y}{\beta} \subseteq \Q{g}{n}{X}{\beta} \end{equation*}
consists of those quasimaps such that $u_Y \equiv 0$.

The cartesian diagram above can also be used to endow $\Q{g}{n}{Y}{\beta}$ with a virtual class via virtual (or diagonal) pull-back along $k$.

We wish to compare this definition with the GIT approach of \cite{CFKM}. We view $Y$ as the GIT quotient of the affine cone $C(Y)\subseteq \mathbb A^{\Sigma_X(1)}$ with respect to the ``diagonal'' action of $G:=\Hom_{\mathbb Z}(\Pic(X),\Gm)\cong\Gm^{\rho_X}\hookrightarrow \Gm^{\Sigma_X(1)}$ (here $C(Y)$ is invariant because it is cut out by a homogeneous equation). Objects of $\om{Q}^{\operatorname{GIT}}_{g,n}(Y,\beta)$ are diagrams of the form
\bcd
P \ar[d] \ar[r] & C(Y) \\ 
C & \,
\ecd
where $P$ is a principal $G$-bundle and the horizontal map is $G$-equivariant. Equivalently, this is the data of $P$ and a section of the associated $C(Y)$-bundle:
\bcd
P\times_{G} C(Y) \ar[d,"\rho" left] \\
C \ar[u,bend right,"u"right]
\ecd
The obstruction theory on this space is defined relative to the stack $\mathfrak{Bun}_{G}$ parametrising principal $G$-bundles on the universal curve:
\begin{equation*} \mathcal{C}_{\MM} \to \MM_{g,n} \end{equation*}
It is given by:
\begin{equation*} \EE_{\om{Q}/\mathfrak{Bun}_G}^\vee = \R \pi_* (u^*\LL_\rho) \end{equation*}
where
\begin{equation*} \pi \colon \mathcal C_{\mathfrak{Bun}}\to\mathfrak{Bun}_{G} \end{equation*}
is the universal curve.

Notice that $\mathfrak{Bun}_{G}\simeq\times^r_{\MM_{g,n}} \mathfrak{Pic}$ by taking the line bundles $\bigoplus_{i=1}^{\rho_X} L^{(i)}=P\times_{G}\mathbb A^{\rho_X}\to C$ associated to the $G$-torsor $P\to C$. Furthermore, the $G$-equivariant embedding in a smooth stack
\bcd
P\times_{G} C_Y \ar[r,hook]\ar[d,"\rho" left] & P\times_{G}\mathbb A^{\lvert\Sigma_X(1)\rvert}\simeq \bigoplus_{\rho\in\Sigma_X(1)} L_{\rho}\ar[dl]\\
C \ar[u,bend right,"u"right] &
\ecd
gives us $u^*T^\bullet_\rho\simeq[\bigoplus_{\rho\in\Sigma_X(1)} \mathcal L_{\rho}\to E_{g,n,\beta}^Y]$, where $E_{g,n,\beta}^Y$ is the line bundle associated to the universal ones $(\mathcal L_\rho)$ by the same rule that takes $(\mathcal O_X(D_\rho))$ to $\mathcal O_X(Y)$, and the arrow is induced by $P_Y$. This shows that both the modular interpretation and the obstruction theory coincide.
