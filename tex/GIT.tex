\subsection{Comparison with the GIT construction} \label{Section comparison with GIT construction}

Let $Y\subseteq X$ be a very ample hypersurface in a smooth projective toric variety, which is cut by a homogeneous polynomial (of degree $\mathcal O_X(Y)$) $P_Y\in k[z_\rho : \rho \in \Sigma_X(1)]$. The complete linear system associated to $Y$ gives an embedding $X\hookrightarrow \PP^N$ such that $Y$ is the intersection of $X$ and a certain hyperplane $H\subseteq\PP^N$. Consider the following cartesian diagram
\bcd
\Q{g}{n}{Y}{\beta}\ar[d]\ar[r] & \Q{g}{n}{X}{\beta}\ar[d,"k"] \\
\Q{g}{n}{H}{d}\ar[r] & \Q{g}{n}{\PP^N}{d}
\ecd
where $d\ell$ is the push-forward of the curve class $\beta$. Here $\Q{g}{n}{Y}{\beta}$ is seen as the closed substack of $\Q{g}{n}{X}{\beta}$ representing those quasimaps $(C,\mathbf x;L_\rho\colon\rho\in\Sigma_X(1),u_\rho\in H^0(C,L_\rho))$ such that $P_Y(\mathbf u)=0$. This diagram can be used to endow $\Q{g}{n}{Y}{\beta}$ with a virtual class.

We wish to compare this with the GIT approach of \cite{CFKM}. Here $Y$ is seen as the GIT quotient of the affine cone $C_Y\subseteq \mathbb A^{\lvert\Sigma_X(1)\rvert}$ with respect to the ``diagonal'' action of $G:=\Hom_{\mathbb Z}(\Pic(X),\Gm)\simeq\Gm^{\rho_X}\to \Gm^{\lvert\Sigma_X(1)\rvert}$ ($C_Y$ is invariant because it is cut by a homogeneous equation). Objects of $\Q{g}{n}{Y}{\beta}^\text{GIT}$ are diagrams of the form

\bcd
P\ar[d,"G"]\ar[r] & C_Y & \text{or, equivalently,} & P\times_{G} C_Y \ar[d,"\rho" left] \\
C & & & C \ar[u,bend right,"u"right]
\ecd
and the dual perfect obstruction theory with respect to $\mathfrak{Bun}_{G}$ is given by $R^\bullet\pi_*(u^*\mathbb T^\bullet_\rho$), where $\pi\colon \mathcal C_{\mathfrak{Bun}}\to\mathfrak{Bun}_{G}$ is the universal curve.

Notice that $\mathfrak{Bun}_{G}\simeq\times^r_{\MM_{g,n}} \mathfrak{Pic}$ by taking the line bundles $\bigoplus_{i=1}^{\rho_X} L^{(i)}=P\times_{G}\mathbb A^{\rho_X}\to C$ associated to the $G$-torsor $P\to C$. Furthermore, the $G$-equivariant embedding in a smooth stack
\bcd
P\times_{G} C_Y \ar[r,hook]\ar[d,"\rho" left] & P\times_{G}\mathbb A^{\lvert\Sigma_X(1)\rvert}\simeq \bigoplus_{\rho\in\Sigma_X(1)} L_{\rho}\ar[dl]\\
C \ar[u,bend right,"u"right] &
\ecd
gives us $u^*T^\bullet_\rho\simeq[\bigoplus_{\rho\in\Sigma_X(1)} \mathcal L_{\rho}\to E_{g,n,\beta}^Y]$, where $E_{g,n,\beta}^Y$ is the line bundle associated to the universal ones $(\mathcal L_\rho)$ by the same rule that takes $(\mathcal O_X(D_\rho))$ to $\mathcal O_X(Y)$, and the arrow is induced by $P_Y$. This shows that both the modular interpretation and the obstruction theory coincide.
