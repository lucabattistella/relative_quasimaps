\subsection{Comparison with the GIT construction} \label{Section comparison with GIT construction}

Let $X$ be a smooth projective toric variety and $Y \hookrightarrow X$ a smooth very ample hypersurface. The complete linear system $|\OO_X(Y)|$ gives an embedding $i : X \hookrightarrow \PP^N$ which expresses $Y$ as the intersection inside $\PP^N$ of $X$ and a certain hyperplane $H$: $Y=i^{-1}(H)$. We can \ilemph{define} the moduli space of quasimaps to $Y$ via the following cartesian diagram:
\bcd
\Q{g}{n}{Y}{\beta}\ar[d, hook] \ar[r, hook]\ar[rd,phantom,"\square"] & \Q{g}{n}{H}{d} \ar[d, hook] \\
\Q{g}{n}{X}{\beta}\ar[r,hook, "k"] & \Q{g}{n}{\PP^N}{d}
\ecd
where $d=i_*\beta$. This moduli space is easy to describe: let $s_Y$ denote the section of $\OO_X(Y)$ cutting out $Y$ inside $X$. Recall from \S \ref{Subsection relative stable quasimaps} that for any quasimap
\begin{equation*} ((C,x_1,\ldots,x_n),(L_\rho,u_\rho)_{\rho \in \Sigma_X(1)}, (\varphi_m)_{m \in M_X}) \in \Q{g}{n}{X}{\beta} \end{equation*}
we can construct a section $u_Y$ of a line bundle $L_Y$ on $C$, which plays the role of the pull-back of $s_Y$ to $C$. Then
\begin{equation*} \Q{g}{n}{Y}{\beta} \subseteq \Q{g}{n}{X}{\beta} \end{equation*}
consists of those quasimaps such that $u_Y \equiv 0$.

The cartesian diagram above can also be used to endow $\Q{g}{n}{Y}{\beta}$ with a virtual class via virtual (or diagonal) pull-back along $k$. Thus we can define quasimap invariants for $Y$.

On the other hand, $Y$ has the natural structure of a GIT quotient
\begin{equation*} Y = C(Y) \sslash G \end{equation*}
where $C(Y)\subseteq \mathbb A^{\Sigma_X(1)}$ is the affine cone over $Y$ and $G=\Hom_{\mathbb Z}(\Pic(X),\Gm)\cong\Gm^{r_X}$ acts on $C(Y)$ via the natural inclusion
\begin{equation*} \Gm^{r_X}\hookrightarrow \Gm^{\Sigma_X(1)} \end{equation*}
(here $C(Y)\subseteq \mathbb{A}^{\Sigma_X(1)}$ is preserved by $G$ because it is cut out by a homogeneous equation). In \cite{CFKM} moduli spaces of quasimaps are constructed for GIT quotient targets (satisfying a number of conditions, all of which hold for $Y$). There is thus a moduli space
\begin{equation*} \om{Q}^{\operatorname{GIT}}_{g,n}(Y,\beta) \end{equation*}
which admits a virtual class. Hence we have two moduli spaces of quasimaps to $Y$, each equipped with a virtual class, and we want to check that these definitions agree.

Objects of $\om{Q}^{\operatorname{GIT}}_{g,n}(Y,\beta)$ are diagrams of the form
\bcd
P \ar[d] \ar[r] & C(Y) \\ 
C & \,
\ecd
where $C$ is a prestable curve, $P$ is a principal $G$-bundle and the map $P \to C(Y)$ is $G$-equivariant. Equivalently, an object consists of a prestable curve $C$, a principal $G$-bundle $P$ and a section $u$ of the associated $C(Y)$-bundle:
\bcd
P\times_{G} C(Y) \ar[d,"p" left] \\
C \ar[u,bend right,"u"right]
\ecd
The obstruction theory on this space is defined relative to the stack $\mathfrak{Bun}_{G}$ parametrising principal $G$-bundles on the universal curve:
\begin{equation*} \mathcal{C}_{\MM_{g,n}} \to \MM_{g,n} \end{equation*}
It is given by
\begin{equation*} \EE_{\om{Q}/\mathfrak{Bun}_G}^\vee = \R \pi_* (u^*\LL_p) \end{equation*}
where $\pi$ is the universal curve over $\om{Q} = \om{Q}^{\operatorname{GIT}}_{g,n}(Y,\beta)$. There is a natural isomorphism
\begin{equation*} \mathfrak{Bun}_{G} \cong \times^{r_X}_{\MM_{g,n}} \mathfrak{Pic_{g,n}} \end{equation*}
given by sending $P$ to the $r_X$ individual factors of the affine bundle $P\times_{G}\mathbb A^{r_X}$. Furthermore there is a $G$-equivariant embedding
\bcd
P\times_{G} C(Y) \ar[r,hook, "j"]\ar[d,"p" left] & P\times_{G}\mathbb A^{\Sigma_X(1)}\cong \bigoplus_{\rho\in\Sigma_X(1)} L_{\rho}\ar[dl]\\
C \ar[u,bend right,"u"right] &
\ecd
which expresses $P \times_G C(Y)$ as the vanishing locus of $u_Y$ in $\oplus_{\rho \in \Sigma_X(1)} L_\rho$. This shows that the two definitions of the moduli space agree.

Finally we must compare the virtual classes. Using the normal sheaf sequence for the inclusion $j$ (relative to the base $C$) we obtain a short exact sequence on $C$:
\begin{equation*} 0 \to u^* \TT_p \to \bigoplus_{\rho \in \Sigma_X(1)} L_\rho \to u^* \NN_{P \times_G C(Y)/\oplus_{\rho \in \Sigma_X(1)} L_\rho} \to 0 \end{equation*}
Since $P \times_G C(Y)$ is defined by the vanishing of $u_Y$, we see that the final term is isomorphic to the line bundle $L_Y$ discussed above. Thus as elements of the derived category
\begin{equation*} u^* \TT_p = [ \bigoplus_{\rho \in \Sigma_X(1)} L_\rho \to L_Y] \end{equation*}
Applying $\R \pi_*$ we obtain on the left hand side the obstruction theory for the GIT moduli space relative $\mathfrak{Bun}_G$. On the other hand, the first term on the right hand side is the obstruction theory for $\om{Q}(X)$ relative the product of the Picard stacks (isomorphic to $\mathfrak{Bun}_G$ via the discussion above) whereas the second term is the relative obstruction theory for $\om{Q}(Y)$ inside $\om{Q}(X)$. Thus the virtual classes agree, as claimed.
