\section{Recursion formula in the general case}\label{Section recursion formula in general case}
We now move on to the general case. Let $X$ be an arbitrary toric variety (smooth and proper) and $Y \subseteq X$ a very ample hypersurface (not necessarily toric). The complete linear system associated to $\OO_X(Y)$ defines an embedding $i : X \hookrightarrow \PP^N$ such that $i^{-1}(H) = Y$ (for a certain hyperplane $H$). By the functoriality property of quasimap spaces (see Appendix \ref{Functoriality of Quasimap Spaces Section}) we have a map:
\begin{equation*} k := \om{Q}(i) : \Q{0}{n}{X}{\beta} \to \Q{0}{n}{\PP^N}{d} \end{equation*}
where $d=i_*\beta$. Since $i$ is a closed embedding it follows that $k$ is as well. Furthermore $k$ admits a compatible perfect obstruction theory (see Appendix \ref{section:rel_pot_for_qm_functoriality}), so we have a notion of virtual pull-back along $k$ (which coincides with the diagonal pull-back according to Lemma \ref{lem:diagonal_virtual_coincide}).

It is easy to show that $k$ restricts to a morphism between the relative spaces, and thus we have a diagram of embeddings
\bcd
\Q{0}{\alpha}{X|Y}{\beta} \ar[d, "f", hook] \ar[r, "g", hook] \ar[dr, phantom, "\square"] & \Q{0}{\alpha}{\PP^N|H}{d} \ar[d, "j", hook] \\
\Q{0}{n}{X}{\beta}  \ar[r, "k", hook] & \Q{0}{n}{\PP^N}{d}
\ecd
which one can show is cartesian. As such we can define a virtual class on $\Q{0}{\alpha}{X|Y}{\beta}$ by (virtual or diagonal) pullback.

The idea is to prove the recursion formula for $(X,Y)$ by pulling back the formula for $(\PP^N,H)$ along $k$. In order to do this, we need to understand how the various virtual classes involved in the formula pull back along this map. The first two terms of the recursion formula pull back by the very definition of the virtual class:
\begin{lemma} \label{Relative spaces pull back} $k^! [\Q{0}{\alpha}{\PP^N|H}{d}] = \virt{\Q{0}{\alpha}{X|Y}{\beta}}$ \end{lemma}
It remains to consider the third term, namely the virtual class of the comb locus. This is the technical heart of the proof.

\subsection{Comb loci pull back} \label{Subsection comb loci pull back}
Recall that we can write $\mathcal D^\mathcal{Q}_{\alpha,k}(X|Y,\beta)$ as a union of comb loci
\begin{equation*} \mathcal D^{\mathcal{Q}}(X|Y,A,B,M) := \Q{0}{A_0 \cup \{q_1, \ldots, q_r\}}{Y}{\beta_0} \times_{Y^r} \prod_{i=1}^r \Q{0}{\alpha^{(i)}\cup (m_i)}{X|Y}{\beta_i} \end{equation*}
where $A$ and $B$ are partitions of the marked points and curve class respectively, and $M=(m_1,\ldots,m_r)$ records the intersection multiplicity with $Y$ at the nodes connecting the internal component to the external components (the spine of the comb to the teeth). Since the virtual class on $\mathcal D^\mathcal{Q}_{\alpha,k}(X|Y,\beta)$ is equal to the sum of the virtual classes of the $\mathcal D^{\mathcal{Q}}(X|Y,A,B,M)$, we can deal with each of these comb loci separately.


\begin{remark} \label{GIT comparison remark} Note that $Y$ is not toric, and so we should clarify what we mean by:
\begin{equation*} \om{Q}(Y) = \Q{0}{A_0 \cup \{ q_1, \ldots, q_n \}}{Y}{\beta_0} \end{equation*}
There are two possibilities here: one is to \emph{define} this space as the cartesian product:
\bcd
\om{Q}(Y) \ar[r] \ar[d] \ar[rd,phantom,"\square"] & \om{Q}(H) \ar[d] \\
\om{Q}(X) \ar[r] & \om{Q}(\PP^N)
\ecd
and equip it with the pullback virtual class (using the fact that the base is smooth).

This has obvious advantages from the point of view of our computations, but is conceptually unsatisfying. On the other hand, $Y \subseteq X$ defines a $(\CC^*)^r$-invariant subvariety in the prequotient of $X$, which we refer to (by analogy with the case $X=\PP^r$) as the \ildef{cone of Y}:
\begin{equation*} C(Y) \subseteq \Aaff_k^{\Sigma_X(1)} \end{equation*}
Then $Y$ is equal to the GIT quotient
\begin{equation*} Y = C(Y) \sslash (\CC^*)^r \end{equation*}
and so we may use the more general theory of quasimaps to GIT quotients (\cite{CFKM}) to define $\om{Q}(Y)$ and its virtual class.

We should then check that these two definitions of $\om{Q}(Y)$ agree (i.e. that there exists an isomorphism between these moduli spaces which preserves the virtual class). This is carried out in an Appendix.
\end{remark}

The comb locus sits inside the full product
\begin{equation*} \mathcal E^{\mathcal{Q}}(X|Y,A,B,M) := \Q{0}{A_0 \cup \{q_1, \ldots, q_r\}}{Y}{\beta_0} \times \prod_{i=1}^r \Q{0}{\alpha^{(i)}\cup (m_i)}{X|Y}{\beta_i} \end{equation*}
which we may endow with the product virtual class
\begin{equation*} \virt{\mathcal E^{\mathcal{Q}}(X|Y,A,B,M)} := \virt{\Q{0}{A_0 \cup \{q_1, \ldots, q_r\}}{Y}{\beta_0}} \times \prod_{i=1}^r \virt{\Q{0}{\alpha^{(i)}\cup (m_i)}{X|Y}{\beta_i}} \end{equation*}
We have the following cartesian diagram
\bcd
\mathcal D^{\mathcal{Q}}(X|Y,A,B,M)\ar[r]\ar[d]\ar[dr,phantom,"\Box"] & \mathcal E^{\mathcal{Q}}(X|Y,A,B,M)\ar[d] \\
X^r\ar[r,"\Delta_{X^r}"] & X^r\times X^r
\ecd
and we can use this to define a \ilemph{product virtual class} on the comb locus:
\[
 \virt{\mathcal D^{\mathcal{Q}}(X|Y,A,B,M)} := \Delta_{X^r}^!\virt{\mathcal E^{\mathcal{Q}}(X|Y,A,B,M)}
\]
\begin{remark} This is the same definition of the virtual class of the comb locus that we gave in \S \ref{Subsection recursion formula for PN} in the case $(X,Y)=(\PP^N,H)$. \end{remark}

On the other hand, there is another cartesian diagram defining the comb locus:
\bcd
\mathcal D^\mathcal{Q}(X|Y,A,B,M) \ar[r,"k"] \ar[d] \ar[rd,phantom,"\square"] & \mathcal D^\mathcal{Q}(\PP^N|H,A,i_* B,M) \ar[d] \\
\Q{0}{n}{X}{\beta} \ar[r,"k"] & \Q{0}{n}{\PP^N}{d}
\ecd
\begin{remark} Technically this is not quite correct: really the fibre product is the union of comb loci over all partitions $B^\prime$ such that $i_* B^\prime = i_*B$ . But this subtlety makes no difference to the aguments. \end{remark}

\begin{lem} \label{Comb loci pull back} For any $\alpha$ we have:
\begin{equation*} k^! \virt{\mathcal D^\mathcal{Q}(\PP^N|H,A,i_*B,M)} = \virt{\mathcal D^\mathcal{Q}(X|Y,A,B,M)} \end{equation*} \end{lem}
Let us introduce the following shorthand notation: we fix the the data of $A$, $B$, $M$ and set:
\begin{align*}
\mathcal{D}(X|Y) & := \mathcal D^{\mathcal{Q}}(X|Y,A,B,M) \\
\mathcal{E}(X|Y) & := \mathcal E^{\mathcal{Q}}(X|Y,A,B,M) \\
\mathcal{D}(X) & := \mathcal{D}^{\mathcal{Q}}(X,A,B) \\
\mathcal{E}(X) & := \mathcal{E}^{\mathcal{Q}}(X,A,B) \\
\om{Q}(X) & :=\Q{0}{n}{X}{\beta}
\end{align*}
and similarly for $(\PP^N,H)$; see Appendix \ref{Subsection splitting} for the definition of $\mathcal{D}(X)$ and $\mathcal{D}(Y)$. We have a cartesian diagram
\bcd
\mathcal{E}(X|Y) \ar[d]\ar[r]\ar[dr,phantom,"\Box"] & \mathcal{E}(\PP^N|H)\ar[d,"\theta"] \\
\mathcal{E}(X)\ar[r] & \mathcal{E}(\PP^N)
\ecd
and since $\mathcal{E}(\PP^N)$ is smooth and there is a natural fundamental class on $\mathcal{E}(\PP^N|H)$, we have a diagonal pull-back morphism $\theta^! = \theta_{\Delta}^!$ (see Appendix \ref{appendix:intersection}).
\begin{lemma}\label{theta-pull} $\virt{\mathcal{E}(X|Y)}=\theta^!\virt{\mathcal{E}(X)}$ \end{lemma}
\begin{proof}
It suffices to check that in the following cartesian diagram
\bcd
\om{Q}(Y) \ar[r] \ar[d] \ar[rd,phantom,"\square"] & \om{Q}(H) \ar[d,"\theta"] \\
\om{Q}(X) \ar[r] & \om{Q}(\PP^N)
\ecd
we have $\theta^! \virt{\om{Q}(X)} = \virt{\om{Q}(Y)}$; this is carried out in an appendix.
\end{proof}

Now consider the following cartesian diagram
\bcd
\mathcal D(X)\ar[r]\ar[d,"\varphi_X"]\ar[dr,phantom,"\Box"] & \mathcal D(\PP^N)\ar[r]\ar[d,"\varphi_{\PP^N}"]\ar[dr,phantom,"\Box"] & \MM_{A,B}^{\operatorname{wt}}\ar[d,"\psi"] \\
\om{Q}(X)\ar[r,"k"] & \om{Q}(\PP^N)\ar[r] & \MM_{0,n}^{\operatorname{wt}}
\ecd
from which we see that
\[
 \psi^!\virt{\om{Q}(X)}= \psi^! k^! [ \om{Q}(\PP^N)] = k^!\psi^![\om{Q}(\PP^N)]
\]
by commutativity of virtual pullbacks. Note that we have:
\begin{equation*} \psi^! \virt{\om{Q}(X)} = \Delta_{X^r}^! \virt{\mathcal{E}(X)} \end{equation*}
by the splitting axiom (see Lemma \ref{Lemma product class equals pullback class}).

\begin{proof}[Proof of Lemma \ref{Comb loci pull back}] Putting all the preceding results together, we consider the cartesian digram:
\bcd
\mathcal D(X|Y)\ar[d]\ar[r]\ar[dr,phantom,"\Box"] & \mathcal E(X|Y)\ar[d]\ar[r]\ar[dr,phantom,"\Box"] & \mathcal E(\PP^N|H)\ar[d,"\theta"] \\
\mathcal D(X)\ar[d]\ar[r]\ar[dr,phantom,"\Box"] & \mathcal E(X)\ar[d]\ar[r] & \mathcal E(\PP^N) \\
X^r\ar[r,"\Delta_{X^r}"] & X^r\times X^r & {}
\ecd
We then have:
\begin{align*} \virt{\mathcal D(X|Y)} & = \Delta_{X^r}^! \virt{\mathcal E(X|Y)} & \\
& =  \Delta_{X^r}^! \theta^!\virt{\mathcal E(X)} & \text{by Lemma \ref{theta-pull}} \\
& =  \theta^!\Delta_{X^r}^! \virt{\mathcal E(X)} & \text{by commutativity} \\
& =  \theta^!\psi^!\virt{\om{Q}(X)} & \text{by the splitting axiom} \\
& =  \theta^!k^!\psi^![\om{Q}(\PP^N)] & \text{by the above} \\
& =  \theta^!k^!\Delta_{(\PP^N)^r}^!\virt{\mathcal E(\PP^N)} & \text{by the splitting axiom} \\
& =  k^!\Delta_{(\PP^N)^r}^!\theta^!\virt{\mathcal E(\PP^N)} & \text{by commutativity} \\
& =  k^!\virt{\mathcal D(\PP^N|H)}
\end{align*}
Summing over all the components of $\mathcal{D}^{\mathcal{Q}}_{\alpha,k}(\PP^N|H,d)$ we obtain the result. \end{proof}

\begin{thm} Let $X$ be a smooth and proper toric variety and let $Y \subseteq X$ be a very ample hypersurface (not necessarily toric). Then, with the set-up as in the preceding discussion, we have an equality
\begin{equation*} (\alpha_k \psi_k + ev_k^* [Y]) \virt{\Q{0}{\alpha}{X|Y}{\beta}} = \virt{\Q{0}{\alpha+e_k}{X|Y}{\beta}} + \virt{\mathcal D^\mathcal{Q}_{\alpha,k}(X|Y,\beta)} \end{equation*}
in the Chow group of $\Q{0}{\alpha}{X|Y}{\beta}$. \end{thm}
\begin{proof} Apply $k^!$ to Proposition \ref{Recursion formula for PN}, using Lemmas \ref{Relative spaces pull back} and \ref{Comb loci pull back}. \end{proof}