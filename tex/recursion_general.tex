\section{Recursion formula in the general case}\label{Section recursion formula in general case}

In this section we prove the main result of this paper: a recursion formula for relative quasimap invariants of a general pair $(X,Y)$.  

\begin{thm} \label{Theorem general recursion} Let $X$ be a smooth and proper toric variety and let $Y \subseteq X$ be a very ample hypersurface (not necessarily toric). Then
\begin{equation*} (\alpha_k \psi_k + \ev_k^* [Y]) \cap \virt{\Q{0}{\alpha}{X|Y}{\beta}} = \virt{\Q{0}{\alpha+e_k}{X|Y}{\beta}} + \virt{\mathcal D^\mathcal{Q}_{\alpha,k}(X|Y,\beta)} \end{equation*}
in the Chow group of $\Q{0}{\alpha}{X|Y}{\beta}$. 
\end{thm}

\noindent We begin by defining the terms that appear in the statement.

\subsection{The virtual class on the space of relative quasimaps} Let $X$ and $Y$ be as in the statement of Theorem~\ref{Theorem general recursion}.  The complete linear system associated to $\OO_X(Y)$ defines an embedding $i : X \hookrightarrow \PP^N$ such that $i^{-1}(H) = Y$ for a certain hyperplane $H$. By the functoriality property of quasimap spaces (see Appendix~\ref{Functoriality of Quasimap Spaces Section}) we have a map:
\begin{equation*} k := \om{Q}(i) : \Q{0}{n}{X}{\beta} \to \Q{0}{n}{\PP^N}{d} \end{equation*}
where $d=i_*\beta$. Since $i$ is a closed embedding it follows that $k$ is as well. Furthermore $k$ admits a compatible perfect obstruction theory (see Appendix~\ref{section:rel_pot_for_qm_functoriality}), so we have a notion of virtual pull-back along $k$.

It is easy to show that $k$ restricts to a morphism between moduli space of relative quasimaps, and thus we have a diagram of embeddings
\bcd
\Q{0}{\alpha}{X|Y}{\beta} \ar[d, "f", hook] \ar[r, "g", hook] \ar[dr, phantom, "\square"] & \Q{0}{\alpha}{\PP^N|H}{d} \ar[d, "j", hook] \\
\Q{0}{n}{X}{\beta}  \ar[r, "k", hook] & \Q{0}{n}{\PP^N}{d}
\ecd
which one can show is cartesian. As such we can define a virtual class on $\Q{0}{\alpha}{X|Y}{\beta}$ by pullback along $k$:
\begin{equation*} \virt{\Q{0}{\alpha}{X|Y}{\beta}} := k^! [ \Q{0}{\alpha}{\PP^N|H}{d} ] \end{equation*}
We use this class to define relative quasimap invariants in general:
\begin{equation*} \langle \gamma_1 \psi_1^{k_1}, \ldots, \gamma_n \psi_n^{k_n} \rangle_{0,\alpha,\beta}^{X|Y} := \int_{\virt{\Q{0}{\alpha}{X|Y}{\beta}}} \prod_{i=1}^n \ev_i^*(\gamma_i) \cdot \psi_i^{k_i} \end{equation*}
These invariants will play a role in our proof of the quantum Lefschetz formula in \S \ref{Section quasimap mirror theorem}.

\subsection{Relative spaces pull back}
The idea is to prove the recursion formula for general $(X,Y)$ by pulling back the formula for $(\PP^N,H)$ along $k$. In order to do this, we need to understand how the various virtual classes involved in the formula pull back along this map. The first two terms pull back by the very definition of the virtual class:
\begin{lemma} \label{Relative spaces pull back} $k^! [\Q{0}{\alpha}{\PP^N|H}{d}] = \virt{\Q{0}{\alpha}{X|Y}{\beta}}$ \end{lemma}

\noindent It thus remains to consider the third term, namely the virtual class of the comb locus. This is the technical heart of the proof.

\subsection{Comb loci pull back} \label{Subsection comb loci pull back} As in the previous section, we define $\mathcal D^\mathcal{Q}_{\alpha,k}(X|Y,\beta)$ to be the union of the moduli spaces
\begin{equation*} \mathcal D^{\mathcal{Q}}(X|Y,A,B,M) := \Q{0}{A^{(0)} \cup \{q_1, \ldots, q_r\}}{Y}{\beta^{(0)}} \times_{Y^r} \prod_{i=1}^r \Q{0}{\alpha^{(i)}\cup (m_i)}{X|Y}{\beta^{(i)}} \end{equation*}
where the union runs over all splittings $A = (A^{(0)},\ldots,A^{(r)})$ of the markings (inducing a splitting $(\alpha^{(0)}, \ldots, \alpha^{(r)})$ of the corresponding tangency requirements), $B = (\beta^{(0)}, \ldots, \beta^{(r)})$ of the curve class $\beta$ and all valid multiplicities $M = (m^{(1)}, \ldots, m^{(r)})$ such that the above spaces are non-empty and such that:
\[
Y \cdot \beta^{(0)} +\sum_{i=1}^r m^{(i)}=\sum \alpha^{(0)}
\]
We refer to the $\mathcal D^{\mathcal{Q}}(X|Y,A,B,M)$ as \emph{comb loci}.

\begin{remark} \label{GIT comparison remark} Note that $Y$ is not in general toric, and so we should clarify what we mean by:
\begin{equation*} \om{Q}(Y) = \Q{0}{A^{(0)} \cup \{ q_1, \ldots, q_n \}}{Y}{\beta^{(0)}} \end{equation*}
There are two possibilities here: one is to \emph{define} this space as the cartesian product
\bcd
\om{Q}(Y) \ar[r] \ar[d] \ar[rd,phantom,"\square"] & \om{Q}(H) \ar[d] \\
\om{Q}(X) \ar[r,"k"] & \om{Q}(\PP^N)
\ecd
and equip it with the virtual class pulled back along $k$:
\begin{equation*} \virt{\om{Q}(Y)} := k^! [ \om{Q}(H) ] \end{equation*}
Using this definition, $\om{Q}(Y)$ consists of those quasimaps in $\om{Q}(X)$ for which $u_Y \equiv 0$.
This has obvious advantages from the point of view of our computations, but is conceptually unsatisfying. 

On the other hand, $X$ is a GIT quotient  $\Aaff^{\Sigma_X(1)}  \sslash \Gm^r$, and $Y \subseteq X$ defines a $\Gm^r$-invariant subvariety $C(Y)$ of $\Aaff^{\Sigma_X(1)}$, which we call the \ilemph{cone of Y}.
Then $Y$ is equal to the GIT quotient
\begin{equation*} Y = C(Y) \sslash \Gm^r \end{equation*}
and so we may use the more general theory of quasimaps to GIT quotients \cite{CFKM} to define $\om{Q}(Y)$ and its virtual class.  

In fact these two definitions of $\om{Q}(Y)$ agree:  there exists an isomorphism between these moduli spaces which preserves the virtual classes. We show this in Appendix~\ref{Section comparison with GIT construction}.
\end{remark}

We now construct a virtual class on the comb locus $\mathcal D^{\mathcal{Q}}(X|Y,A,B,M)$.  Consider the product (\emph{not} the fibre product over $Y^r$)
\begin{equation*} \mathcal E^{\mathcal{Q}}(X|Y,A,B,M) := \Q{0}{A^{(0)} \cup \{q_1, \ldots, q_r\}}{Y}{\beta^{(0)}} \times \prod_{i=1}^r \Q{0}{\alpha^{(i)}\cup (m_i)}{X|Y}{\beta^{(i)}} \end{equation*}
which we may endow with the product virtual class (with weighting as before):
\begin{align*} \virt{\mathcal E^{\mathcal{Q}}(X|Y, & A,B,M)} := \\
& \left( \dfrac{m^{(1)} \cdots m^{(r)}}{r!}\right) \cdot \left( \virt{\Q{0}{A^{(0)} \cup \{q_1, \ldots, q_r\}}{Y}{\beta^{(0)}}} \times \prod_{i=1}^r \virt{\Q{0}{\alpha^{(i)}\cup (m_i)}{X|Y}{\beta^{(i)}}} \right) \end{align*}
We have the following cartesian diagram
\bcd
\mathcal D^{\mathcal{Q}}(X|Y,A,B,M)\ar[r]\ar[d]\ar[dr,phantom,"\Box"] & \mathcal E^{\mathcal{Q}}(X|Y,A,B,M)\ar[d] \\
X^r\ar[r,"\Delta_{X^r}"] & X^r\times X^r
\ecd
and we can use this to define a virtual class on the comb locus:
\[
 \virt{\mathcal D^{\mathcal{Q}}(X|Y,A,B,M)} := \Delta_{X^r}^!\virt{\mathcal E^{\mathcal{Q}}(X|Y,A,B,M)}
\]
The virtual class on the union $\mathcal D^\mathcal{Q}_{\alpha,k}(X|Y,\beta)$ of the comb loci is defined to be the sum of the virtual classes $\virt{\mathcal D^{\mathcal{Q}}(X|Y,A,B,M)}$.


\begin{remark} This is the same definition of the virtual class of the comb locus that we gave in \S \ref{Subsection recursion formula for PN} in the case $(X,Y)=(\PP^N,H)$. \end{remark}

On the other hand, there is another cartesian diagram:
\bcd
{\displaystyle \coprod_{B \colon i_* B = B^\prime} \mathcal D^\mathcal{Q}(X|Y,A,B,M)} \ar[r] \ar[d] & \mathcal D^\mathcal{Q}(\PP^N|H, A, B^\prime, M) \ar[d] \ar[ld,phantom,"\square"]  \\
\Q{0}{n}{X}{\beta} \ar[r,"k"] & \Q{0}{n}{\PP^N}{d}
\ecd
Recall that we are trying to show that the virtual class of the comb locus pulls back nicely along $k$. The result that we need is:
\begin{lem} \label{Comb loci pull back} $\displaystyle k^! \virt{\mathcal D^\mathcal{Q}(\PP^N|H,A,B^\prime,M)} = \sum_{B : i_* B = B^\prime} \virt{\mathcal D^\mathcal{Q}(X|Y,A,B,M)}$ \end{lem}

For the proof of Lemma~\ref{Comb loci pull back}, let us introduce the following shorthand notation.  We fix the the data of $A$, $B^\prime$, $M$ and set:
\begin{align*}
\mathcal{D}(X|Y) & := \textstyle \coprod_{B \colon i_* B = B^\prime} \mathcal D^{\mathcal{Q}}(X|Y,A,B,M) &&& \mathcal{D}(\PP^N|H) & := \mathcal D^{\mathcal{Q}}(\PP^N|H,A,B^\prime,M)\\
\mathcal{E}(X|Y) & := \textstyle \coprod_{B \colon i_* B = B^\prime} \mathcal E^{\mathcal{Q}}(X|Y,A,B,M) &&& \mathcal{E}(\PP^N|H) & := \mathcal E^{\mathcal{Q}}(\PP^N|H,A,B^\prime,M)\\
\mathcal{D}(X) & := \textstyle \coprod_{B \colon i_* B = B^\prime} \mathcal{D}^{\mathcal{Q}}(X,A,B)  &&& \mathcal{D}(\PP^N) & := \mathcal D^{\mathcal{Q}}(\PP^N,A,B^\prime)\\
\mathcal{E}(X) & := \textstyle \coprod_{B \colon i_* B = B^\prime} \mathcal{E}^{\mathcal{Q}}(X,A,B) &&& \mathcal{E}(\PP^N) & := \mathcal E^{\mathcal{Q}}(\PP^N,A,B^\prime)\\
\om{Q}(X) & :=\Q{0}{n}{X}{\beta} &&& \om{Q}(\PP^N) & :=\Q{0}{n}{\PP^N}{i_* \beta} 
\end{align*}
Here $\mathcal{D}(X)$ and $\mathcal{E}(X)$ are the centipede loci introduced in Appendix \ref{Subsection splitting}; they are defined in the same way as the comb loci, except that we replace both the quasimaps to $Y$ and the relative quasimaps to $(X,Y)$ by quasimaps to $X$. There is a cartesian diagram
\bcd
\mathcal{E}(X|Y) \ar[d]\ar[r]\ar[dr,phantom,"\Box"] & \mathcal{E}(\PP^N|H)\ar[d,"\theta"] \\
\mathcal{E}(X)\ar[r] & \mathcal{E}(\PP^N)
\ecd
and, since $\mathcal{E}(\PP^N)$ is smooth and there is a natural fundamental class on $\mathcal{E}(\PP^N|H)$, we have a diagonal pull-back morphism $\theta^! = \theta_{\Delta}^!$ (see Appendix~\ref{appendix:intersection}).
\begin{lemma}\label{theta-pull} $\virt{\mathcal{E}(X|Y)}=\theta^!\virt{\mathcal{E}(X)}$ \end{lemma}
\begin{proof}
It suffices to check that in the following cartesian diagram
\bcd
\om{Q}(Y) \ar[r] \ar[d] \ar[rd,phantom,"\square"] & \om{Q}(H) \ar[d,"\theta"] \\
\om{Q}(X) \ar[r] & \om{Q}(\PP^N)
\ecd
we have $\theta^! \virt{\om{Q}(X)} = \virt{\om{Q}(Y)}$. Depending on one's definition of $\om{Q}(Y)$ (see Remark \ref{GIT comparison remark} above) this is either true by definition or is proved in Appendix \ref{Section comparison with GIT construction}.
\end{proof}

Now consider the following cartesian diagram
\bcd
\mathcal D(X)\ar[r]\ar[d,"\varphi_X"]\ar[dr,phantom,"\Box"] & \mathcal D(\PP^N)\ar[r]\ar[d,"\varphi_{\PP^N}"]\ar[dr,phantom,"\Box"] & \MM_{A,B}^{\operatorname{wt}}\ar[d,"\psi"] \\
\om{Q}(X)\ar[r,"k"] & \om{Q}(\PP^N)\ar[r] & \MM_{0,n,\beta}^{\operatorname{wt}}
\ecd

where $\MM_{0,n,\beta}^{\operatorname{wt}}$ is the moduli space of prestable curves weighted by the class~$\beta$ \cite[\S 2]{Costello} and:
\begin{equation*} \MM_{A,B}^{\operatorname{wt}} := \MM_{0,A^{(0)} \cup \{ q_1^0, \ldots, q_r^0 \}, \beta^{(0)}}^{\operatorname{wt}} \times \prod_{i=1}^r \MM_{0,A^{(i)} \cup \{ q_i^1 \},\beta^{(i)}}^{\operatorname{wt}} \end{equation*}
The maps $\mathcal{D}(X) \to \MM_{A,B}^{\operatorname{wt}}$ and $\om{Q}(X) \to \MM_{0,n,\beta}^{\operatorname{wt}}$ admit relative perfect obstruction theories which are the same as the usual perfect obstruction theories relative to the moduli spaces of \emph{unweighted} curves. Furthermore the morphism $\psi$ admits a perfect obstruction theory; see Appendix \ref{Subsection splitting} for details. Thus there are virtual pull-back morphisms $\psi^!$, and by the splitting axiom (see Lemma \ref{Lemma product class equals pullback class}) we have

\begin{equation*} \virt{\mathcal{D}(X)} := \Delta_{X^r}^! \virt{\mathcal{E}(X)} = \psi^! \virt{\om{Q}(X)} \end{equation*}
Commutativity of virtual pull-backs then implies that:
\begin{equation} \virt{\mathcal{D}(X)} = \label{psishriek formula} \psi^!\virt{\om{Q}(X)}= \psi^! k^! [ \om{Q}(\PP^N)] = k^!\psi^![\om{Q}(\PP^N)] = k^! [ \mathcal{D}(\PP^N) ]\end{equation}

\begin{proof}[Proof of Lemma \ref{Comb loci pull back}] Putting all the preceding results together, we consider the cartesian digram:
\bcd
\mathcal D(X|Y)\ar[d]\ar[r]\ar[dr,phantom,"\Box"] & \mathcal E(X|Y)\ar[d]\ar[r]\ar[dr,phantom,"\Box"] & \mathcal E(\PP^N|H)\ar[d,"\theta"] \\
\mathcal D(X)\ar[d]\ar[r]\ar[dr,phantom,"\Box"] & \mathcal E(X)\ar[d]\ar[r] & \mathcal E(\PP^N) \\
X^r\ar[r,"\Delta_{X^r}"] & X^r\times X^r & {}
\ecd
We then have:
\begin{align*} \virt{\mathcal D(X|Y)} & = \Delta_{X^r}^! \virt{\mathcal E(X|Y)} & \text{by definition}\\
& =  \Delta_{X^r}^! \theta^!\virt{\mathcal E(X)} & \text{by Lemma \ref{theta-pull}} \\
& =  \theta^!\Delta_{X^r}^! \virt{\mathcal E(X)} & \text{by commutativity} \\
& =  \theta^! \virt{\mathcal{D}(X)} & \text{by definition} \\
& =  \theta^!k^! [\mathcal{D}(\PP^N)] & \text{by formula \eqref{psishriek formula} above} \\
& =  \theta^!k^!\Delta_{(\PP^N)^r}^! [\mathcal E(\PP^N)] & \text{by definition} \\
& =  k^!\Delta_{(\PP^N)^r}^!\theta^! [\mathcal E(\PP^N)] & \text{by commutativity} \\
& = k^! \Delta_{{\PP^N}^r}^! [\mathcal{E}(\PP^N|H)] & \text{by Lemma \ref{theta-pull}} \\
& =  k^![\mathcal{D}(\PP^N|H)] & \text{by definition}
\end{align*}
Summing over all the components of $\mathcal{D}^{\mathcal{Q}}_{\alpha,k}(\PP^N|H,d)$ we obtain the result. \end{proof}

\begin{proof}[Proof of Theorem~\ref{Theorem general recursion}]
Apply $k^!$ to Proposition~\ref{Recursion formula for PN}, using Lemmas~\ref{Relative spaces pull back} and~\ref{Comb loci pull back}. 
\end{proof}