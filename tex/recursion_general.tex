\section{Recursion formula in the general case}
We now move on to the general case. Let $X$ be an arbitrary toric variety (smooth and proper) and $Y \subseteq X$ a very ample hypersurface (not necessarily toric). The complete linear system associated to $\OO(Y)$ defines an embedding $i : X \hookrightarrow \PP^N$ such that $i^{-1}(H) = Y$ (for a certain hyperplane $H$). By the functoriality property of quasimap spaces (see Appendix \ref{Functoriality of Quasimap Spaces Section}) we have a map:
\begin{equation*} k := \mathcal Q(i) : \Q{0}{n}{X}{\beta} \to \Q{0}{n}{\PP^N}{d} \end{equation*}
where $d=i_*\beta$. Since $i$ is a closed embedding it follows that $k$ is as well. Furthermore $k$ admits a compatible perfect obstruction theory - see Section \ref{section:rel_pot_for_qm_functoriality} -, so we have a notion of virtual pull-back along $k$ (which coincides with the \emph{diagonal} pull-back according to Lemma \ref{lem:diagonal_virtual_coincide}).

It is easy to show that $k$ restricts to a morphism between the relative spaces, and thus we have a diagram of embeddings
\bcd
\Q{0}{\alpha}{X|Y}{\beta} \ar[d, "f", hook] \ar[r, "g", hook] \ar[dr, phantom, "\square"] & \Q{0}{\alpha}{\PP^N|H}{d} \ar[d, "j", hook] \\
\Q{0}{n}{X}{\beta}  \ar[r, "k", hook] & \Q{0}{n}{\PP^N}{d}
\ecd
which one can show is cartesian. As such we can define a virtual class on $\Q{0}{\alpha}{X|Y}{\beta}$ by (virtual or diagonal) pullback.

The idea is to prove the recursion formula for $(X,Y)$ by pulling back the formula for $(\PP^N,H)$ along $k = \mathcal Q(i)$. In order to do this, we need to understand how the various virtual classes involved in the formula pull back along this map. Note that the first two terms of the recursion formula pull back trivially along $k$. It remains to consider the third term, namely the virtual class of the comb locus. This is the technical heart of the proof.

\subsection{Centipede loci and obstruction theories}
 We wish to prove (see Lemma \ref{Comb loci pull back}) that:
\begin{equation*} k_{\text{v}}^! [D^\mathcal{Q}_{\alpha,k}(\PP^N|H,d)] = \virt{D^\mathcal{Q}_{\alpha,k}(X|Y,\beta)} \end{equation*}
We will require a number of preparatory results. 

There is one other way of obtaining $\D{X}{A}{B}$ as a pullback. Recall that we have an embedding $i: X \hookrightarrow \PP^N$. Let $i_*B = (i_* \beta_0, \ldots, i_* \beta_r)$. Then we have a cartesian diagram:
\bcd
\D{X}{A}{B} \ar[r,"\eta"] \ar[d,"\varphi_X"] \ar[rd,phantom,"\square"] & \mathcal{K}(\PP^N,A,i_*B) \ar[d,"\varphi_{\PP^N}"] \\
\Q{0}{n}{X}{\beta} \ar[r,"k"] & \Q{0}{n}{\PP^N}{d}
\ecd

\begin{lemma} \label{Centipede loci pull back} $k_{\text{v}}^! [ \mathcal{K}(\PP^N,A,i_*B)] = \virt{\D{X}{A}{B}}$ \end{lemma}

\begin{proof} It follows from the construction above that the obstruction theory $E_{\varphi_X} \to L_{\varphi_X}$ is equal to the pullback of $E_{\varphi_\PP^N} \to L_{\varphi_{\PP^N}}$ along $\eta$.

[JUSTIFY THIS!]

So:
\begin{equation*} (\varphi_X)_{\text{v}}^! = (\varphi_{\PP^N})_{\text{v}}^! : A_*(\om{Q}(X)) \to A_*(\mathcal{K}(X)) \end{equation*}
Then commutativity of virtual pull-backs gives
\begin{align*} k_{\text{v}}^! [\mathcal{K}(\PP^N)] & = k_{\text{v}}^! (\varphi_{\PP^N})_{\text{v}}^! [\om{Q}(\PP^N)] \\
& = (\varphi_{\PP^N})_{\text{v}}^! k_{\text{v}}^! [ \om{Q}(\PP^N) ] \\
& = (\varphi_{\PP^N})_{\text{v}}^! \virt{\om{Q}(X)} \\
& = (\varphi_X)_{\text{v}}^! \virt{\om{Q}(X)} \\
& = \virt{\mathcal{K}(X)}
\end{align*}
where we have used Lemma \ref{Lemma product class equals pullback class} twice. \end{proof}

\subsection{Comb loci pull back}
We are finally able to prove that the third term in the recursion formula pulls back along $k$.

\begin{lem} \label{Comb loci pull back} For any $\alpha$ we have:
\begin{equation*} k_{\text{v}}^! [D^\mathcal{Q}_{\alpha,k}(\PP^N|H,d)] = \virt{D^\mathcal{Q}_{\alpha,k}(X|Y,\beta)} \end{equation*} \end{lem}

Consider the cartesian diagram:
\bcd
D^\mathcal{Q}_{\alpha,k}(X|Y,\beta) \ar[r,"k"] \ar[d] \ar[rd,phantom,"\square"] & D^\mathcal{Q}_{\alpha,k}(\PP^N|H,d) \ar[d] \\
\Q{0}{n}{X}{\beta} \ar[r,"k"] & \Q{0}{n}{\PP^N}{d}
\ecd
We can write $D^\mathcal{Q}_{\alpha,k}(X|Y,\beta)$ as the disjoint union of spaces
\begin{equation*} D^{\mathcal{Q}}(X|Y,A,B,M) = \Q{0}{A_0 \cup \{q_1, \ldots, q_r\}}{Y}{\beta_0} \times_{Y^r} \prod_{i=1}^r \Q{0}{\alpha^{(i)}\cup (m_i)}{X|Y}{\beta_i} \end{equation*}
where $A$ and $B$ are partitions of the marked points and curve class as before, and $M=(m_1,\ldots,m_r)$ records the intersection multiplicities at the nodes which connect the internal component to the external components (the spine of the comb to the teeth).

As usual this space has a virtual class induced by pulling back the virtual class from the total product:
\begin{equation*} E^{\mathcal{Q}}(X|Y,A,B,M) = \Q{0}{A_0 \cup \{q_1, \ldots, q_r\}}{Y}{\beta_0} \times \prod_{i=1}^r \Q{0}{\alpha^{(i)}\cup (m_i)}{X|Y}{\beta_i} \end{equation*}

\begin{remark} \label{GIT comparison remark} There is a subtelty here; since $Y$ is not toric, it is not immediately obvious what we mean by the quasimap space:
\begin{equation*} \om{Q}(Y) = \Q{0}{A_0 \cup \{ q_1, \ldots, q_n \}}{Y}{\beta_0} \end{equation*}
There are two possibilities here: one is to \emph{define} this space as the cartesian product:
\bcd
\om{Q}(Y) \ar[r] \ar[d] \ar[rd,phantom,"\square"] & \om{Q}(H) \ar[d] \\
\om{Q}(X) \ar[r] & \om{Q}(\PP^N)
\ecd
and equip it with the pullback virtual class (using the fact that the base is smooth).

This has obvious advantages from the point of view of our computations, but is conceptually unsatisfying. On the other hand, $Y \subseteq X$ defines a $(\CC^*)^r$-invariant subvariety in the prequotient of $X$, which we refer to (by analogy with the case $X=\PP^r$) as the \ildef{cone of Y}:
\begin{equation*} C(Y) \subseteq \Aaff_k^{\Sigma_X(1)} \end{equation*}
Then $Y$ is equal to the GIT quotient
\begin{equation*} Y = C(Y) \sslash (\CC^*)^r \end{equation*}
and so we may use the more general theory of quasimaps to GIT quotients (\cite{CFKM}) to define $\om{Q}(Y)$ and its virtual class.

We should then check that these two definitions of $\om{Q}(Y)$ agree (i.e. that there exists an isomorphism between these moduli spaces which preserves the virtual class). This is carried out in Appendix \ref{Section comparison with GIT construction}.
\end{remark}

Having dealt with this issue, we can now move towards the proof of Lemma \ref{Comb loci pull back}. In order to make the discussion readable, we will need to introduce some shorthand notation. We suppose that the data of $A$,$B$ and $M$ has been fixed, and set:
\begin{align*}
\mathcal{K}(X|Y) & := D^{\mathcal{Q}}(X|Y,A,B,M) \\
\mathcal{L}(X|Y) & := E^{\mathcal{Q}}(X|Y,A,B,M)
\end{align*}
We then have a cartesian diagram:
\bcd
\mathcal{K}(X|Y) \ar[r] \ar[d] \ar[rd,phantom,"\square"] & \mathcal{L}(X|Y) \ar[r] \ar[d] \ar[rd,phantom,"\square"] & \mathcal{L}(\PP^N|H) \ar[d,"\theta"] \\
\mathcal{K}(X) \ar[r] \ar[d] \ar[rd,phantom,"\square"] & \mathcal{L}(X) \ar[r] \ar[d] & \mathcal{L}(\PP^N) \\
X^r \ar[r,"\Delta"] & X^r \times X^r
\ecd

The morphism $\theta$ does not have a perfect relative obstruction theory. Nevertheless, we can define virtual pull-back morphisms $\theta_{\text{v}}^!$ in the sense of \S \ref{Intersection theory lemma subsection} because there is a natural class on $\mathcal{L}(\PP^N|H)$. 

\begin{lemma} \label{K(X) pulls back to K(X|Y)} $\virt{\mathcal{K}(X|Y)} = \theta_{\text{v}}^!\virt{\mathcal{K}(X)}$ \end{lemma}
\begin{proof}
By the very definition we have
\begin{equation*} \virt{\mathcal{L}(X|Y)} = \theta_{\text{v}}^! \virt{\mathcal{L}(X)} \end{equation*}
and hence we obtain:
\begin{align*} \virt{\mathcal{K}(X|Y)} & = \Delta^! [\mathcal{L}(X|Y)]\\
& = \Delta^! \theta_{\text{v}}^! \virt{\mathcal{L}(X)} \\
& = \theta_{\text{v}}^! \Delta^! \virt{\mathcal{L}(X)} \\
& = \theta_{\text{v}}^! \virt{\mathcal{K}(X)} \end{align*}
as required.
\end{proof}

\begin{proof}[Proof of Lemma \ref{Comb loci pull back}]
Finally, consider the diagram:
\bcd
\mathcal{K}(X|Y) \ar[r] \ar[d] \ar[rd,phantom,"\square"] & \mathcal{K}(\PP^N|H) \ar[r] \ar[d] \ar[rd,phantom,"\square"] & \mathcal{L}(\PP^N|H) \ar[d,"\theta"] \\
\mathcal{K}(X) \ar[r] \ar[d] \ar[rd,phantom,"\square"] & \mathcal{K}(\PP^N) \ar[r] \ar[d] & \mathcal{L}(\PP^N) \\
\om{Q}(X) \ar[r,"k"] & \om{Q}(\PP^N)
\ecd
Then we have:
\begin{align*} k_{\text{v}}^! [ \mathcal{K}(\PP^N|H) ] & = k_{\text{v}}^! \theta_{\text{v}}^! [ \mathcal{K}(\PP^N) ] \text{\qquad \ by Lemma \ref{K(X) pulls back to K(X|Y)}} \\
& = \theta_{\text{v}}^! k_{\text{v}}^! [ \mathcal{K}(\PP^N) ] \\
& = \theta_{\text{v}}^! \virt{\mathcal{K}(X)} \text{\ \ \qquad by Lemma \ref{Centipede loci pull back}} \\
& = \virt{\mathcal{K}(X|Y)} \text{\qquad \ \, by Lemma \ref{K(X) pulls back to K(X|Y)}}
\end{align*}
The lemma follows because $D^\mathcal{Q}_{\alpha,k}(X|Y,\beta)$ and $D^\mathcal{Q}_{\alpha,k}(\PP^N|H,d)$ are the disjoint union of the $\mathcal{K}(X|Y)$ and $\mathcal{K}(\PP^N|H)$. \end{proof}

\begin{thm} Let $X$ be a smooth and proper toric variety and let $Y \subseteq X$ be a very ample hypersurface (not necessarily toric). Then, with the set-up as in the preceding discussion, we have an equality
\begin{equation*} (\alpha_k \psi_k + ev_k^* [Y]) \virt{\Q{0}{\alpha}{X|Y}{\beta}} = \virt{\Q{0}{\alpha+e_k}{X|Y}{\beta}} + \virt{D^\mathcal{Q}_{\alpha,k}(X|Y,\beta)} \end{equation*}
in the Chow group of $\Q{0}{n}{X}{\beta}$. \end{thm}
\begin{proof} Pull-back the equation for $\PP^N$ relative $H$ using the map $k_{\text{v}}^!$, applying Lemmas \ref{Relative spaces pull back} and \ref{Comb loci pull back}. \end{proof}